%************************************************
\chapter{Problematic Interpretations}
\label{chapter:problematic_interpretations}
%************************************************

\section{The Mistake of Subjective Idealism}
\label{section:the_mistake_of_subjective_idealism}

Readers might confuse my model of mind with the philosophy of
Subjective Idealism or Immaterialism \citep{berkeley:1734}, since both
use the term ``mind'' to refer to everything that exists.  However, it
would be a mistake to confuse this semantic detail in my explanation
of my model of mind being equivalent to Subjective Idealism.  The
philosophy of Subjective Idealism states that everything is
axiomatically divided into subject and object parts.  Then,
subsequently, the subject part is defined to be everything.  In other
words, Subjective Idealists first believe that the world is divided
into two parts, the subject and objects, and then subsequently, and
illogically claim, that the subject is all that exists.  Obviously,
there is a tautological cycle in this model.  The mistake can be found
in Subjective Idealism's definition of ``everything''.  Everything
functions as the origin of the model as well as the ``part''.  There
is a causal circularity in even attempting to imagine this everything
that is being divided into these parts in the first place.

For example, here's another way to realize the contradiction.  If one
is aware of having a subjective perspective on objects, one must have
a vantage point from which to be aware of this.  This vantage point is
necessarily outside of the subjective perspective.  If the mind is the
subjective perspective, then one could never become aware of being
subjective, since there would be nothing in existence to contrast
``everything''.  My theory explicitly avoids a subjective perspective
on objects.  Therefore, when I define the mind to be everything that
exists, this limitation avoids the flaw that would prevent the
modelling of an awareness of subjective points of view.

