%************************************************
\chapter{Introducing Computation}
\label{chapter:introducing_computation}
%************************************************

\section{Digital Abstraction}

Computer simulations are based on the \emph{digital abstraction}, an
assumption that provides the most basic symbols of a computational
model, ``$1$'' and ``$0$''.  I will refer to the capacity to arrange
symbols in Space as \emph{computer memory}.  I will refer to a
specific arrangement of symbols in computer memory as \emph{data}.
While computer memory is often thought of as arising from physical
objects, such as transistors, the specific physical derivation of
these symbols is not important, given the modelling assumption of
digital abstraction.  For example, computer memory can be written down
on paper and manipulated by hand, once the idea is understood, it is a
tool for thinking, separately from any physical implementation.  This
assumption is necessary when using computers in modelling work.
Computer memory is a static arrangement of symbols in Space.

\section{The Combinational Device}

The digital abstraction is not all that is assumed when using a
computer to simulate a model.  The digital abstraction provides the
static symbols in Space, so let us now focus on the dynamic activities
in Duration.  In simulating a model with a computer, there is a
predetermined discrete transition from the past to the future.  In the
past, there is an arrangement of symbols in Space and there is a
predetermined activity in Duration that causes these symbols in Space
to be arranged in a different specific configuration in the future.
This discrete activity is designed to be exactly the same for every
transition from the past to the future.  This logical predetermined
transition is referred to as \emph{the combinational device}.  The
combinational device maps any specific combination of symbols
Spatially arranged in the past to a specific Spatial arrangement in
the future.

\section{Digital Reflection}
\label{section:digital_reflection}

Because the combinational device is always understood to be the exact
same activity, there is not a non-tautological reason to symbolically
reference this activity in the computer memory.  Because this
transitional activity is always understood to be exactly the same, the
symbolic cause that refers to the activity during the transition from
the past to the future would be predefined and thus always the same,
making no distinctions.  I will refer to the ability to symbolize the
ongoing dynamic activity of the digital transition from the past to
the future as \emph{digital reflection}.

\section{The Programmer}

Symbols have grounding only in reference to the ongoing activities in
Duration.  So, then, the question becomes: to what dynamic activities
of Duration do digital symbols refer?  The symbols in computer memory
are programmed by a programmer that uses these symbols as references
to the activities in Duration that are actively ongoing and currently
being experienced by the programmer.  Computer memory is a static tool
of thought for the programmer, who provides the grounded reference to
the dynamic for the $1$'s and $0$'s in the computer memory.  Again, a
computer does not create symbols that reference the dynamic.  Given
symbols that have meaning to the programmer, these symbols are used to
create models that are simulated according to the dynamic activities
of the combinational device.  The model that I have programmed in this
way simulates the reflective creation of simulated symbols, or
$\text{symbols}^*$.  These $\text{symbols}^*$ can then be dynamically
displayed on a computer monitor, symbolically perceived, and given
meaning outside of the computational process by the programmer or
other observer.

