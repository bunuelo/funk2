%*****************************************
\chapter{A System}\label{ch:a_system}
%*****************************************

\section{Reflectively Traced Frame Memory}


I've written an entire multi-core operating system, including a
compiler, on top of this traceable and distributed memory layer.  It
doesn't run as fast as a highly optimized operating system, but this
project is about building models that are reflectively traced.
Everything is open-source, can be downloaded from my webpage, and
compiled by simply typing "./configure; make".  Gerry sees this as a
complete waste of two or three years of my PhD, which is frustrating.

I've built the system with the sole intention of building a massive
combination of many artificial intelligence techniques.  I can run
thousands of parallel processes concurrently on my 8-core machine.  They
can control and watch one another execute.  I feel that the field of AI
is not separate from the low-level details of software engineering, and
my project embodies that philosophy.

Many people see AI as being a theoretical and mathematical field, and I
strongly disagree.  We need good software engineers, not mathematicians
to solve most of the problems we face in getting these massive software
systems to work together.

That said, I have built a layered cognitive architecture on top of my
custom operating system, programming language, and compiler.  The
cognitive architecture is what I gave you a demo of.

Joe Paradiso understands what he calls the "dynamic range" of my thesis,
but I fear that my advisors are unsatisfied or simply uninterested in
the technical solutions to the software engineering that I've spent most
of my time doing.  The goal of my project is to build a demo of part of
Marvin's architecture, and the domain of commonsense reasoning in a
kitchen is I feel the right way to develop those high-level theories.

If you have references for the kitchen as a good problem solving domain,
those would be very helpful.  I know you've done a lot of work with
kitchens and models of mind.  Kitchens are ubiquitous across cultures.
They have a clear production goal, food.  They involve many many mental
realms: math, physics, chemistry, thermodynamics, natural language,
social, family, imprimer learning, children, parents, concurrent
planning, etc.


\section{An Operating System}

\section{A Programming Language}



\subsection{Why not use Lisp?}

Lisp is a great programming language.  We wrote a custom programming
language for the project and didn't use lisp.  Lisp simply isn't fast
enough, and isn't very well supported; when you find a bug in a lisp
compiler, it is difficult to find the support to fix the bug.  We
wrote the first version of the reflectively traced memory system in
lisp and realized that Steele Bourne Common Lisp had memory bugs when
the system grew beyond 600 megabytes of RAM.  Allegro Lisp is a
commercial solution, but it costs many hundreds of dollars for their
commercial compiler, and we feel strongly against having that
commercial requirement for building academically intentioned
open-source software.  The main problem with lisp is it's lack of
speed and lack of support, so we found ourselves writing a lot of C
extensions even when programming in Lisp.  C is good for speeding up
inner loops of algorithms as well as necessary for interfacing with
the Linux, Mac, or Windows operating systems, which are all written in
C.

\section{A Layered Cognitive Architecture}

Further, we have developed a cognitive architecture within our
language that provides structures for layering reflective processes,
resulting in a hierarchy of control algorithms that respond to
failures in the layers below.

