%************************************************
\chapter{An Example}
\label{chapter:an_example}
%************************************************

In this chapter, I will describe a scenario where learning occurs at
multiple reflective layers.

\section{The \emph{in media res}}

There is a table.  The table has some blocks on it.  The blocks are
different shapes and colors.  One of the blocks is a green triangle.
Another block is a red square.  There is a hand that can move around,
picking up, and putting down blocks.  One can think of plans for the
hand to follow to arrange the blocks into different types of goal
configurations.  For example, one can think of trying to stack two of
the blocks.  More specifically, one can think of trying to stack the
red square on the green triangle.  If one knows how triangles and
squares interact, then one would know that triangles do not provide
good support surfaces, and squares will fall off if attempts are made
to stack them on top.  If one had the knowledge of the physical
interactions, then one would not make this plan.  But, in this
example, this plan exists in ones mind because one does not have this
knowledge of how triangles and squares physically interact.  In this
case, one thinks that this plan should succeed.  If one were to
execute this plan, moving the hand, picking up and putting down
blocks, one would find that the red square falls off of the green
triangle.  This would be an unexpected outcome that would cause the
plan to fail.  When the plan fails, there is physical knowledge to be
learned: putting the red square on the green triangle results in the
red square being on the table.  There is also knowledge to be learned
about how one uses plans.  For example, imagine that one has been told
two plans for how to stack blocks; each of these plans is learned from
a different source, a different friend.  One of the plans is for
stacking a red square on top of a green triangle, while the other plan
has the opposite approach of stacking the triangle on the square.  One
of these plans will fail and the other will succeed.  Which one to use
is a matter of being good at thinking, not physically acting.








For example, one can think of trying to stack two of the blocks.  More
specifically, one can think of trying to stack the red square on the
green triangle.  If one knows how triangles and squares interact, then
one would know that triangles do not provide good support surfaces;
squares will fall off if attempts are made to stack them on top.  If
there was an awareness of this knowledge, this plan would not be
further considered.  But, in this example, there is not an awareness
of this knowledge, and this plan is considered as potentially useful
for stacking the red square block on top of the green triangle.  In
this case, one thinks that this plan should succeed.  If one were to
follow this plan, moving the hand, picking up and putting down blocks,
one would find that the red square falls off of the green triangle.
This would be an unexpected outcome that would cause the plan to fail.
When the plan fails, there is physical knowledge to be learned:
putting the red square on the green triangle results in the red square
being on the table.  There is also knowledge to be learned about how
one uses plans.  For example, imagine that one has been told two plans
for how to stack blocks; each of these plans is learned from a
different source, a different friend.  One of the plans is for
stacking a red square on top of a green triangle, while the other plan
has the opposite approach of stacking the triangle on the square.  One
of these plans will fail and the other will succeed.  Which one to use
is a matter of being good at thinking, not physically acting.
