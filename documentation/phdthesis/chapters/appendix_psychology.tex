\chapter{Related Psychology}

Between the ages of 1-3 years old, children display primary emotions,
such as joy, disappointment, and surprise.  These emotional processes
have been hypothesized to be related to the process of failing or
succeeding to accomplish a goal.  Around age 4, children begin to
display emotions that involve the self, such as guilt and shame.  It
has been hypothesized that these emotions relate to another person's
evaluation of the child's goals as good or bad.

We approach modelling this developmental process by applying Marvin
Minsky's theory of the child-imprimer relationship.  According to
Minsky's theory, at a young age, a human child becomes attached to a
person that functions as a teacher.  The imprimer could be a parent or
a caregiver or another person in the child's life, but the function of
the imprimer is to provide feedback to the child in terms of what
goals are good or bad for the child to pursue.

\section{Simulation Theory of Mind versus Theory Theory of Mind}


\section{Emotion or affect versus goal-oriented cognition}


\section {Embarrassment, Guilt, and Shame}



