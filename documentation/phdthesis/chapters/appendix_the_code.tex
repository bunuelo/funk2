\chapter{The Code}\label{appendix:the_code}

\section{Open-Source Download}

Everything is open-source, can be downloaded from my webpage, and
compiled by simply typing "./configure; make".

\section{The Hacker Philosophy of Code}

The problems of artificial intelligence and greater control of more
complicated and interconnected computational systems are not easy
domains to approach from a programming perspective.  Many of these
problems seem to require not only new programming abstractions,
methodologies and philosophies, but also bug-free implementations of
these potentially paradigm shifting ideas.

In other words, the only people that are going to be able to solve
these types of control problems that are on the forefront of science
fiction in research are going to be expert computer programmers, or
``hackers.''  In general the ideal hacker is a problem solver that
accomplishes their own goals in given complicated systems, not
necessarily computer systems.

In terms of computational science, hackers are often confronted by
given hardware and software problems that must be overcome in order to
implement solutions to their posed problems.  Hacking can be adopted
as a fun philosophy for accomplishing ones own goals in life, but
also, I think that hacking is a required prerequisite for any form of
scientific progress.  A joy for tinkering that leads to playing, which
subsequently leaves the tinkerer as an expert hacker is what is needed
for approaching undocumented domains, such as the forefront of
artificial intelligence and control theory in general.

\section{What is a Computer?}
\label{sec:what_is_a_computer}

Good question.  We've presented a simple model in
Section~\ref{sec:computational_process_model}, but this abstract model
leaves a lot to be desired for modelling the more realistic computers
that most programmers enjoy.  For example, hardware platforms with
many processing cores per CPU, multiple CPUs per motherboard, and
multiple computers per networked cluster of motherboards.  In order to
reflectively control computational systems that are organized
according to these modern engineering constraints, I have organized my
memory infrastructure to span these areas of future reflective control
research.  For example, basic pointer in the Funk2 programming
language includes 17 bits to represent the cluster machine ID, and 9
bits to represent a processor core specific memory allocation pool.
We see Funk2 as a platform supporting an open research community of
hackers that share computational resources in order to build
demonstrations of larger reflective artificial intelligence control
systems.

