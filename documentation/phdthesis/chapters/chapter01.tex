%************************************************
\chapter{Introduction}\label{ch:introduction}
%************************************************

\begin{quote}
Problem-solvers must find relevant data.  How does the human mind
retrieve what it needs from among so many millions of knowledge items?
Different AI systems have attempted to use a variety of different
methods for this.  Some assign keywords, attributes, or descriptors to
each item and then locate data by feature-matching or by using more
sophisticated associative data-base methods.  Others use
graph-matching or analogical case-based adaptation.  Yet others try to
find relevant information by threading their ways through systematic,
usually hierarchical classifications of knowledge---sometimes called
``ontologies''.  But, to me, all such ideas seem deficient because it
is not enough to classify items of information simply in terms of the
features or structures of those items themselves.  This is because we
rarely use a representation in an intentional vacuum, but we always
have goals---and two objects may seem similar for one purpose but
different for another purpose.
\end{quote}
\begin{flushright}
 --- \defcitealias{minsky:1991}{Marvin Minsky}\citetalias{minsky:1991} \citep{minsky:1991}
\end{flushright}

\section{Commonsense Reasoning}

