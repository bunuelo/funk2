\chapter{Related Philosophy}\label{appendix:philosophy}

\subsection{The Objective Modelling Assumption}

\begin{figure}[bth]
  \center
  \includegraphics[width=4cm]{gfx/objective_description}
  \caption[The objective-subjective modelling assumption]{The objective-subjective modelling assumption.}
  \label{fig:objective_description}
\end{figure}

We assume that the phenomenon that we are trying to model, namely
human intelligence, is an objective process that we can describe.
This is the objective-subjective philosophical assumption that is
inherent in any objective scientific hypothesis.  We make this
assumption in order to avoid logical problems of circular causality
that occur when trying to find a non-objective description of
reflective thinking.  Figure~\ref{fig:objective_description} shows
how, given the objective assumption, the subjective scientist is part
of the real world, while she is studying an objective phenomenon.  We
refer the reader to Appendix~\ref{appendix:philosophy} for a further
discussion of the problems of mistaking an objective model for reality
itself.  We will now continue to describe our model of human
intelligence in an objective way, aware of the utility and
artificiality of our objective assumption.

\section{Being, Time, and the Verb-Gerund Relationship}

\section{The intensional stance}

\section{Reflective Representations}

\cite{perner:1991}

