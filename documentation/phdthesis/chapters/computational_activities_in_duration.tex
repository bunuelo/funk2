%************************************************
\chapter{Computational Activities in Duration}
\label{chapter:computational_activities_in_duration}
%************************************************

%\newcommand{\FibR}{$\text{Fib}^\text{R }$}
%\newcommand{\SALS}{SALS}

My implementation is based on a virtual machine that allows parallel
and concurrent scheduling of Lisp-like programs.  I refer to this
virtual machine as \emph{SALS}, the substrate for accountable layered
systems, the focus of this dissertation.  The purpose of SALS is to
provide a simulation of the actual dynamic activities in Duration, the
activities in Duration that actually exist.

In this dissertation, I am focusing on how SALS demonstrates
reflective learning in two distinct causal classes of knowledge from a
single physical expectation failure.  I must begin my description with
the dynamic activities in Duration.  Thus, first, I will describe how
SALS simulates the activities in Duration.  I will explain the
symbolization of these activities in the context of a block building
physical simulation.  I will describe how causal models are learned.
I will give examples of how causal models are put together into plan
objects that are simple serial and parallel SALS programs.  I will
explain how a plan may begin executing and fail.  Finally, I will
describe how the expectation failure response updates causal models in
both reflective layers.  This, in total, shows how learning how to
plan is simulated by SALS.

\section{The Funk Virtual Operating System}

The virtual concurrent machine and operating system underlying SALS is
called \emph{Funk} \cite[]{morgan:2009} for two reasons: (1) the
``slap'' style of bass guitar playing \cite[]{graham:1969,
  johnson:1984}, which emphasizes the fundamentals of the dynamic
actual ongoing, and (2) the focus on reflective tracing of concurrent
procedural ``funktions'' as opposed to purely declarative or logical
approaches, which have no means of optimizing the underlying search
problem.  In this dissertation, I focus on the implementation of
second-order reflective thinking within this virtual operating system.
In order to come to this point, a very general underlying substrate
has been programmed, which will largely not be discussed.  To briefly
give an idea of the complexity of the implementation, the following
table gives quantification of the size of the SALS codebase:

\vspace{5mm}
\begin{tabular}{lr}
Lines of C code:             &    141,138 \\
Lines of Funk code:          &     50,346 \\
Total lines of code:         &    192,868 \\
Total characters of code:    & 10,117,671
\end{tabular}
\vspace{5mm}

The project is small compared to other virtual operating systems and
programming languages, but represents a non-trivial effort toward a
fully reflective parallel and concurrent virtual operating system for
implementing large-scale practical applications, such as word
processors, web browsers, as well as scientific data processing tools.

\section{The Global Environment}

In SALS, the simulation state graph, $S$, becomes the \emph{global
  environment}.  The global environment is SALS' static graph
representation of the dynamic activities in Duration.  The previous
part described the simulation state as the mathematical representation
for the mind.  I previously defined this simulation state graph to
represent all subgraphs that actually exist in
Equation\ \ref{equation:define_exists}, reproduced here:
\begin{equation*}
\text{exists}(G) \longleftrightarrow (G ~{\subseteq}~ S)
\end{equation*}
Environments are organized into hierarchies that have two parts: (1) a
frame containing slots containing variable definitions, and (2) a
reference to a parent environment.  The global environment is a
special environment because it is the only environment that has no
parent environment.  The entire SALS computational implementation is
contained within the static global environment or its directed
references.

\section{The Global Scheduler}

SALS is able to take advantage of multi-core and multi-processor
hardware, which makes it a concurrent processing system.  SALS
explicitly represents these hardware processors in an object called
the \emph{global scheduler}.  The global scheduler is contained in the
global environment frame under the slot name {\tt global-scheduler}.
The global scheduler object contains references to a number of
processors that reflect the number of actual hardware processors in
the current system.  For example, the implementation has been
successfully tested on 1 through 32 processors, although most of the
demonstration simulations were computed using a machine with two
processors, each with 4 cores, which appears as 8 concurrent
processors within SALS.

\section{Fibers Simulate Activities}

SALS introduces an object that represents a virtual machine that can
run user programs, the \emph{fiber}.  Fibers are the fundamental
element for simulating parallel user programs in SALS.  Each
concurrent processor contains references to a collection of fibers
that it is responsible for simulating.  Fibers execute bytecode
programs.  In each concurrent processor's execution cycle, it runs a
given finite number of bytecodes for each of its fibers, simulating
parallel execution.

SALS begins the simulation with only one fiber that serves a
predefined ``boot-up'' function that starts all of the other
concurrent fibers that are necessary for simulating the mind.  Fibers
are the discrete elements that I have used as a model of the
concurrent, actually inseparable, activities in Duration.

My simulation of mind uses hundreds of thousands of fibers in order to
demonstrate two layers of reflective learning.  Fibers are often very
short-lived processes; for example, not execution, but simply
compiling a single Lisp-like expression in SALS can lead to the
creation and destruction of hundreds of fibers of activity.  Fibers
are generally useful in programs that could use an extra perspective
on a problem.  {\mbox{\autoref{figure:example_global_environment}}}
shows an example global environment, containing the global scheduler,
concurrent processors, and parallel executing user-programmable fiber
objects.
\begin{figure}
\center
\includegraphics[width=12cm]{gfx/example_global_environment}
\caption[An example of a global environment with concurrent processors
  and parallel fibers.]{An example of a global environment with its
  frame object and a definition of the global scheduler object.  The
  global scheduler contains references to concurrent processors, which
  organize the parallel fibers, virtual machines that execute user
  programs.}
\label{figure:example_global_environment}
\end{figure}

\section{Stepping the Simulation State}

For the purposes of description and evaluation of the complexity of my
implementation of the reflective simulation, I assume that each
concurrent processor executes at the same speed and steps the
simulation forward at the same time.  Given a simulation state graph,
$S$, all of the concurrent processors step the simulation forward to
the next simulation state graph.
Equations\ \ref{equation:define_simulation_step_first}
and\ \ref{equation:define_simulation_step_last} define the initial
simulation state and recursive definition of the step function that
computationally creates the next simulation state graph.
\begin{align}
\label{equation:define_simulation_step_first}
    S[0] &\equiv \text{\emph{Initial simulation state graph}} \\
\label{equation:define_simulation_step_last}
  S[n+1] &= \text{step}(S[n])
\end{align}
Note that until this point, I have not previously introduced a
computational step function for the simulation state.  I have
previously defined the temporal transitions that compose $n$ layers of
distinct temporal orderings, one for each reflective layer in the
model, but these layers of temporal transition representations are
within the static state of the mind.  This computational step function
allows all of the previously described static structures to change to
new static representations.  The computational step function does not
exist within the mind, and the mind does not have any access to
symbolizing these different simulation state graphs that are now
introduced in the computational implementation.

Causally traced procedural reflection is the key component that allows
SALS to monitor changes to its own simulation state graph
representation.  Before discussing how procedural reflection must be
embedded into the definition of SALS' fundamental memory
representation, I must first briefly describe how fibers are
programmed in order to simulate simple non-reflective programs.

\section{Programming Language}

SALS includes a Lisp-like programming language, which is programmed
by typing statements that are called \emph{expressions}.  If
expression \ref{expression:print_hello} were typed into SALS, the
symbol ``green'' would be printed to the user's terminal screen.
\begin{equation}
\label{expression:print_hello}
\text{\tt [print `green]}
\end{equation}
The {\tt print} command is a useful debugging tool that can report
status messages to the programmer as the fiber reaches a specific
point in the program.

\section{Sequential and Parallel Programs}

SALS includes expressions for describing serial and parallel
programs.  For example, the expression
\begin{equation*}
\begin{array}{l}
\text{\tt [prog [print 1]} \\
\text{\tt ~~~~~~[print 2]} \\
\text{\tt ~~~~~~[print 3]]}
\end{array}
\end{equation*}
results in the output trace
\begin{equation*}
\begin{array}{l}
\text{\tt 1} \\
\text{\tt 2} \\
\text{\tt 3.}
\end{array}
\end{equation*}
The command {\tt prog} is a way for expressing a sequence of commands
to be executed in serial order.  SALS also includes the {\tt parog}
command for executing a list of commands in parallel, waiting for them
all to complete, and then continuing.
\begin{equation*}
\begin{array}{l}
\text{\tt [parog [print 1]} \\
\text{\tt ~~~~~~~[print 2]} \\
\text{\tt ~~~~~~~[print 3]]}
\end{array}
\end{equation*}
When {\tt parog} is used, it is unclear what command will complete
first because they are all running concurrently, in parallel, starting
at slightly different times.  Here is an example output trace from the
{\tt parog} expression:
\begin{equation*}
\begin{array}{l}
\text{\tt 3} \\
\text{\tt 1} \\
\text{\tt 2.}
\end{array}
\end{equation*}
The {\tt parog} expression is one way to easily start a number of
parallel fibers to simultaneously execute a number of different tasks
and wait for these tasks to complete.

\section{Fibers Create Objects}

SALS includes an object type system.  Every expression in SALS,
besides {\tt nil}, has an object type.  Objects in SALS are in most
cases based on a frame knowledge representation.  Objects have three
main classes of functionality: {\tt get}, {\tt set}, and {\tt have}.
The uses of these types of functionality are as follows:
\begin{itemize}
\item {\tt [get <object> <slot-name>]}

The object-oriented {\tt get} command retrieves the value from the
named slot of the object's frame.
\item {\tt [set <object> <slot-name> <new-value>]}

The object-oriented {\tt set} command mutates the value at the named
slot of the object's frame to be the given new value.
\item {\tt [have <object> <slot-name> <arguments>]}

The object-oriented {\tt have} command performs other types of
activities that are not simple frame slot accessors and mutators.  The
{\tt have} commands sometimes involve complex mutations or other
side-effects.
\end{itemize}

\section{Creating Parallel Fibers}

The {\tt apply} operator is the normal way for a SALS program to
evaluate a given function with arguments:
\begin{equation*}
\text{\tt [apply <function> <arguments>]}
\end{equation*}
The fundamental operator of SALS's parallel programming language is
the {\tt fiber} operator:
\begin{equation*}
\text{\tt [fiber <function> <arguments>]}
\end{equation*}
The {\tt fiber} operator does not evaluate the given function with
arguments.  The {\tt fiber} operator starts a new parallel fiber that
will evaluate the given function and arguments; the new fiber object
is returned.  The intermediate state or final result can be monitored
through the fiber object.

\section{Monitoring Simulated Activities}

SALS includes a tool named \emph{FiberMon} that is written in the SALS
programming language.  FiberMon helps the programmer to introspect on
all fibers currently in the simulation scheduler and is shown in
\autoref{figure:fibermon_many_fibers}.  Fibers may be easily removed
or added to the scheduler, which enables efficient scheduler
optimizations to be implemented at the highest levels of the language.
If any bugs arise in any parallel fiber, that fiber shows up in red in
the monitoring application, so that it may be inspected and debugged
by hand.
\begin{figure}[bth]
  \center
  \includegraphics[width=11cm]{gfx/fibermon_many_fibers}
  \caption[FiberMon application monitoring many fibers]{FiberMon
    application monitoring hundreds of parallel fibers, simulated
    activities in Duration, executing on eight different physical
    processor cores.}
  \label{figure:fibermon_many_fibers}
\end{figure}

\section{Pausing and Resuming Activities}

SALS includes a {\tt pause} command to pause the current fiber:
\begin{equation*}
\text{\tt [pause]}
\end{equation*}
The {\tt pause} command causes the executing fiber to remove itself
from the scheduler, after removing itself, it yields the rest of the
scheduler cycle.  A fiber that is paused in this sense uses none of
SALS's processor resources because it is completely removed from
the scheduler.

If a fiber has paused, it cannot restart itself.  In order for a fiber
to resume execution, another executing fiber must use the
{\tt resume} command with the fiber object as an argument:
\begin{equation*}
\text{\tt [resume <fiber>]}
\end{equation*}

The {\tt pause} and {\tt resume} commands are extremely efficient for
managing large simulations in which most fibers will be inactive at
any given time, but they are a little cumbersome, so I have written a
lightweight and helpful \emph{fiber trigger} object, which I will
explain after explaining one more object that is used to program fiber
triggers.

\section{Mutual Exclusion}

SALS provides a primitive object called a \emph{mutex} for handling
the mutually exclusive access to resources.  A mutex object has two
possible states: locked and unlocked.  If the mutex object is unlocked
then when a fiber tries to locks the mutex, the mutex switches to the
locked state and the fiber continues execution.  If the mutex object
is already locked when a fiber tries to lock the mutex, the fiber will
pause until the mutex is unlocked, at which point the mutex will be
locked and the waiting fiber will resume execution.  Mutexes are used
for protecting shared resources from being used by more than one fiber
at a time.

Mutexes are a special type of object that must be supported by the
hardware of a concurrent computer.  Almost every parallel programming
language has a mutex construct that derives from this hardware mutex.
So, mutexes are common to parallel programming, but they are
notoriously difficult to debug, resulting in bugs affectionately
referred to as ``race conditions'' or ``deadlocks''.  In order to help
with debugging these types of problems, mutexes in SALS keep track of
which fibers are waiting for the lock or holding the lock.  I have
found this extra information invaluable in debugging complicated mutex
related bugs.

\section{Fiber Triggers}

The fiber trigger organizes sets of paused fibers so that they can be
woken up at the same time.  Basically, the fiber trigger is an object
that provides a useful abstraction for controlling fiber execution
that combines a mutex with the {\tt pause} and {\tt resume} commands.

A fiber can be added to the resume set of a fiber trigger by using the
{\tt wait-for-trigger} command as in the following example:
\begin{equation*}
\text{\tt [wait-for-trigger <fiber-trigger>]}
\end{equation*}
The {\tt wait-for-trigger} command atomically pauses the current fiber
and then adds the fiber to resume queue of the fiber trigger object.

\section{Fiber Complete and Bug-Found Triggers}

Fibers have two primary ways of completing their execution: by (1)
successfully reaching the end of their execution task, or by (2)
encountering a bug.  Each fiber has a special fiber trigger for each
of these two possible events: (1) a {\tt complete} trigger, and (2) a
{\tt bug-found} trigger.  When a parallel fiber is created, further
parallel fibers can be created in order to listen for one or both of
these triggers.  Within the internals of the Funk operating system, an
object called a \emph{larva} is created when a C function encounters
an error.  A larva object is similar to an exception in other
programming languages such as C++ or Java.  The larva object
propagates up the C execution stack until it reaches the {\tt value}
register of the Funk virtual operating system fiber object.  The Funk
virtual operating system never sees the larva object because when the
C code detects a larva object in the {\tt value} register of the
virtual machine represented by the fiber object, the larva object is
immediately converted to a bug object and the fiber is also
immediately paused by the C code.  When a bug is encountered the {\tt
  bug-found} trigger is triggered, and a reflective parallel fiber can
respond and inspect the paused fiber that contains the bug in its {\tt
  value} register.  Larva and bugs are very versatile frame-based
objects, so they can pass an arbitrary amount of information from the
depths of the C code internals of the virtual machine to the very
abstract reflective thinking capabilities of the SALS first-order
reflective learning-to-act and second-order reflective
learning-to-plan systems.

%\section{leftovers...}

%\section{Symbols and Simulated Symbols}

%It becomes important at this point to keep track of the layer of
%modelling that results when one simulates thinking.  The danger is in
%losing an awareness of the difference between what one is thinking and
%what a simulation is simulating as thought.  For example, a programmer
%types symbols to express commands to the computer in a programming
%language.  These symbols have grounding for the programmer.  The
%computer continually repeats the same symbolic manipulation function,
%the combinational device.  So, when the simulation appears to create
%new symbols by manipulating the symbols contained within its
%programming, this is an illusion.  The observer of output from the
%simulation may create new symbols, but the computational process only
%has grounding in terms of the symbols of its programming.  Therefore,
%I will use the previously introduced asterisk notation for referring
%to different classes of simulated modelling.

%Now that I have described the difference between the symbols the
%programmer types and the symbols* that the AI simulates as being
%created, the first goal of my description of the implementation is to
%describe how SALS simulates the creation of a symbol.  The rest of
%this chapter will cover the creation of a symbol*, but first, I must
%discuss how the activities in Duration begin simulating by programming
%lists of symbols into SALS' interactive programming language.

%get fiber triggers from fiber objects.
%
%execution complete versus bug found.
%
%
%\section{{\tt par-fib}}
%
%\begin{equation*}
%\begin{array}{l}
%\text{\tt [defunk par-fib [n]} \\
%\text{\tt ~~[if [== n 0]} \\
%\text{\tt ~~~~~~0} \\
%\text{\tt ~~~~[if [== n 1]} \\
%\text{\tt ~~~~~~~~1} \\
%\text{\tt ~~~~~~[let [[x []]} \\
%\text{\tt ~~~~~~~~~~~~[y []]]} \\
%\text{\tt ~~~~~~~~[parog [= x [par-fib [- n 2]]]} \\
%\text{\tt ~~~~~~~~~~~~~~~[= y [par-fib [- n 1]]]]} \\
%\text{\tt ~~~~~~~~[+ x y]]]]]}
%\end{array}
%\end{equation*}
%
%
%\section{Symbolic Statements in SALS}
%
%\section{Three Categories of Symbol}
%
%SALS programming expressions are combinations of symbols in lists.
%It is important to keep all of these symbols straight.  I have so far
%introduced three slightly different conceptions of symbols: (1) the
%symbols that are reflectively symbolized from the activities of
%Duration, (2) the symbols that a programmer types into a computer, and
%(3) the simulated symbols that are generated by the simulated
%first-order reflective layer of thinking.
%
%\section{Concurrent Memory Allocation Pools}
%
%SALS works off of a multiple pool memory allocation system that
%allows separate pools for each concurrent virtual processor,
%eliminating the need for many lock situations that occur with shared
%memory pools.
%
%\section{Garbage Collection}
%
%
%
%\section{Layered Cognitive Architecture}
%
%On this computational substrate, I have built a layered cognitive
%architecture that is inspired by Minsky's description of the bottom
%four layers of his Emotion Machine, or Model-6 architecture.  This
%includes, a physical world, a reactive mapping of the physical world
%to pre-symbolic activities, a first-order reflective thinking layer
%that creates symbols, causal models, and plans for accomplishing
%physical goals, as well as a second-order reflective thinking layer
%that creates symbols, causal models, and plans for accomplishing
%first-order thinking goals.  The model behind this implementation
%inductively explains how this implementation can be extended to an
%arbitrary number of reflective thinking layers.  While this part of
%the thesis is primarily focused on the simulation of a static,
%modelled representation of this model, I derive the fundamental basis
%of my model of mind in non-technical English in
%\autoref{part:the_model}.
%
%\section{Block Building Domain}
%
%My architecture exists in three main parts: the physical layer, the
%first-order reflective layer, and the second-order reflective layer.
%In order to explain how my model allows for learning at multiple
%levels, I use a simulation of a physical domain that is easy to
%understand.  I use this simulation primarily for communication of my
%working model by demonstrating learning at multiple reflective levels
%in response to a single physical failure.
%
%My simulation of the physical block building domain is meant to appear
%as similar to the canonical toy problem, \emph{Blocks World}, with one
%key exception: my model is meant to have a different interpretation
%than the original Blocks World.  The primary point to emphasize here
%is that my physical simulation is meant to represent a dynamic
%physical world as opposed to the logical and completely static
%reference for the Blocks World physical domain.
%
%
%
%\section{Reactive Layer}
%
%The physical layer in my model is implemented by combining the
%physical simulation with a reactive layer that maps the physical
%simulator perceptual and motor functions to ``sub-symbolic''
%activities that are available to first-order reflection.



