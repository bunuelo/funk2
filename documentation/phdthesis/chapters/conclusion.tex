%************************************************
\chapter{Conclusion}
\label{chapter:conclusion}
%************************************************

Building a model of general human intelligence is a massive
engineering effort that includes the shared expertise between all of
the disciplines of cognitive science, including: philosophy,
artificial intelligence, psychology, linguistics, anthropology, and
neuroscience.  While this dissertation is focused on building an
artificial intelligence that models novel forms of metacognition that
occur at higher layers of reflective intelligence than have before
been modeled, I hope that the future of the SALS architecture enjoys
contributions from all of the diverse sub-disciplines of cognitive
science.  The SALS AI is meant to be a user-friendly platform for
encoding many specific types of intelligences from each of these
domains that are usually researched separately.  The current strengths
and weaknesses of the SALS AI have been discussed throughout this
dissertation, but in this case, I hope that the following features of
the SALS architecture help toward this collaborative end:
\begin{packed_enumerate}
\item{Simple Syntax}
\item{Object-Oriented Frame-Based Representation}
\item{Cognitive Architectural Primitives, such as Resources, Agencies,
  and Layers.}
\item{Reflectively Traced Semantic Knowledge Bases}
\item{Causal Organization of Parallel and Concurrent Processes}
\item{Extension Package Manager}
\item{On-Line Documentation and Open-Source Community
  \cite[]{morgan:2012}}
\end{packed_enumerate}
Given that the current five-layered version of the SALS AI is designed
to take advantage of multicore processors with local cache, as
computer hardware trends toward CPUs with more cores on each chip with
each core being simpler and with more on-chip cache storing memory
local to computations, the SALS virtual machine is well placed to
benefit from this coming wave of distributed multithreaded hardware
platforms.  This dissertation has described the four contributions of
this thesis:
\begin{packed_enumerate}
\item \emph{Emotion Machine Cognitive Architecture}
\item \emph{Learning from Being Told Natural Language Plans}
\item \emph{Learning Asynchronously from Experience}
\item \emph{Virtual Machine and Programming Language}
\end{packed_enumerate}
The SALS AI has been evaluated to scale at a tractable linear increase
in time-complexity for each additional layer, which implies an $N$
layered metacognitive architecture that allows not only reasoning
about a problem domain but also any number of reflective layers of
planning and learning about the thinking processes themselves.

