%************************************************
\chapter{Education}
\label{chapter:education}
%************************************************

%\section{Skill and Understanding}

To understand an activity in Duration is to actively maintain a causal
model of its effects.  There are many activities in Duration that are
referred to as skills that do not require understanding.  For example,
an understanding of walking is not necessary for the skillful activity
of walking to occur.  Skills are symbolic references to activities
that may be refined through symbolic repetition without having any
causal model existing for the activity.  Also, one may have an
understanding without having any symbolic skill referring to the
activity.  Often, an understanding of the activity can guide a
practice routine for refining the skill.  Coaches use an understanding
of a skill to provide instructions for practicing skills.  When skills
are practiced, the activity in Duration changes.  The understanding of
the skill does not necessarily change during the practice.  Practicing
a skill can simply lead to sub-symbolic changes to the activity that
the symbolic skill is a referent.

\section{Coaching and Teaching}

Coaching and teaching play similar roles in changing the mind of the
disciple.  The purpose of coaching is to change sub-symbolic skillful
activities, while the purpose of teaching is to give the disciple an
understanding of an activity.  The roles of coach and teacher work
well together.  For example, the master may teach the disciple an
understanding of a symbolic activity before coaching how to practice
the skill.  Often, an understanding is used as a crutch for initially
learning to practice a skill, and once the disciple has learned to
practice the skill, the understanding is no longer necessary and is
sometimes even a distraction from its mastery.

\section{Understanding and Mechanical Simulation}

If an activity is understood, then a mechanical model of the activity
can be built, allowing for the symbolic simulation of the activity.
Computers allow for the mechanical simulation of the active
maintenance of symbolic relationships in Space; computers also allow
for the simulation of causal models, allowing transitions from the
past to the future.  Therefore, computers allow for the mechanical
simulation of the understanding of an activity.  In this sense, the
existence of a mechanical simulation of the understanding of an
activity is an existence proof for the understanding of the activity.
Further, for any understanding, a mechanical simulation can be built.
Therefore, a one-to-one relationship exists between an understanding
and a mechanical simulation.

\section{Education through Neuroscience}

Education is a field that is designed to coach and teach an
understanding of skillful thinking activities.  Current technology
uses cleverly designed forms of reading and writing as the primary
tool for determining whether or not a child has learned the target
skills.  As the skills become longer and more complex, it becomes
harder for this form of linguistic evaluation to be designed by the
teacher.  Modern brain scanning technologies have begun to explore
exciting new ways to approach education of sub-linguistic symbolic
thinking.  This opens the possibility for the coaching and teaching of
pre-linguistic forms of skillful thinking and understanding.

The design, assignment, and grading of homework for schoolchildren is
expensive, time consuming, and error prone, when humans perform these
tasks.  If parts of these tasks could be automated, more children
would become better educated, given the same resources.  Because of
growing interest in this potential from both an educational and mental
health perspective, I here explain the relationship between the
physical objective science and the model of mind, which are both
necessary for automating these tasks.

When physical phenomena are measured and correlated with a model of
mind, a potential exists for mechanically automating a program based
on educational goals, emphasizing augmentation and rehabilitation of
mental behavior.  Valuing mental activities in correlation with
physical phenomena, such as neural networks as measured by brain
scanning technologies, will lead to a new approach to more efficient
and more exact educational tools.  Current reliance on symbolic
written output from complex thinking tasks is a critical bottleneck in
all forms of education and a critical impasse for teaching
pre-linguistic forms of thinking.

Children learn in many ways.  One way is through their personal
experience playing with the physical world.  Another important way is
through understanding language and inheriting knowledge directly from
parents and teachers in a language form.  Having a model of mind that
explains how causal models are learned and debugged in both of these
cases is important for an embracing neuroscientific approach to
education.

Having a model of mind that provides explanatory descriptions for many
types of learning is important to allow for a wide range of children
to be approached sensitively and idiosyncratically with an educational
program that helps them at their individual stages of learning.
Developing a program that enables students to usefully create,
incorporate, and debug knowledge from many different sources is
important for developing a sound inherited culture and adaptable
individuals.

