%************************************************
\chapter{Evaluation}
\label{chapter:evaulation}
%************************************************

The focus of this thesis has been the implementation of a
computational model of the dynamic activity of reflective thinking.  I
began in {\mbox{\autoref{part:the_model}}} with a basic model.  In
order to put this basic model into mathematical terms,
{\mbox{\autoref{part:the_simulation}}} introduced the assumption that
the activities in duration can be modelled as an arrangement of
symbols in graph theoretic relationships.  Further, while maintaining
an awareness of the primary modelling focus on actual reflective
thinking, {\mbox{\autoref{part:the_implementation}}} made the further
assumption that the dynamic activity of simulation is a static logical
combinational device.  In this chapter, I evaluate the thesis in terms
of the following three guiding questions:
\begin{enumerate}
\item How well have the goals of the model described in
  {\mbox{\autoref{part:the_model}}} been accomplished?
\item How well have the goals of the mathematical simulation model
  described in {\mbox{\autoref{part:the_simulation}}} been
  accomplished?
\item How well have the goals of the computational implementation
  described in {\mbox{\autoref{part:the_implementation}}} been
  accomplished?
\end{enumerate}
Next I will review the goals of each of the contributions of this
thesis, followed by a discussion and evaluation of the more specific
goals of each of the three larger contributions.

\section{The Model}

The model described in {\mbox{\autoref{part:the_model}}} serves a
number of purposes:
\begin{enumerate}
\item To communicate an understanding of the distinction between
  actual continuous dynamic activity as opposed to the static symbolic
  references and models that reflective thinking manipulates, guiding
  thinking about reflective thinking away from ``loopy'' models and
  toward a layered theory of mind that avoids the tautologies inherent
  in such ``loopy'' theories as described by \cite{perlis:2008}.
\item To provide a foundational model that does not assume a
  subjective view of objects, so that this model may lead to models of
  the analogical abstraction of subjective views of objects from
  symbolic goal-oriented plans.
\item To explain how actual logical problem solving may benefit from
  the refinement of symbolic perceptions and action resources in order
  to create better hypothetical models for accomplishing symbolic
  goals.
\item To explain how probabilistic causal models are rationally
  constructed from the more fundamental symbolic perceptions and
  action resources, and therefore, must keep track of these more basic
  symbolic models, so that they may be debugged when they end up
  failing.
\item To explain how neural networks and other objective models of
  ``sub-symbolic'' perception and thinking are at least constructions
  of second-order reflective thinking, more abstract and symbolic than
  the perceptions of the first-order reflective layer that they are
  created to describe.
\item To explain that a model that is grounded in the actual dynamic
  activities that exist can not only represent factual perceptual
  memories of physical activities but also factual perceptual memories
  of thinking activities, such as actual plan failures of the
  reflective thinking layers.
\end{enumerate}
While I have attempted to make {\mbox{\autoref{part:the_model}}}
accessible in non-technical language, an evaluation of the
communicative purposes of this part can only be evaluated in
retrospect.  The evaluation of how well this part serves as a strong
and basic foundation for the subsequent contributions of this thesis
is next evaluated in the terms of those contributions, respectively.

\section{The Simulation}

To review, the simulation model described in
{\mbox{\autoref{part:the_simulation}}} serves the following purposes:
\begin{enumerate}
\item To explicitly represent the dynamic continuous activities
  otherwise left undescribed in {\mbox{\autoref{part:the_model}}} by
  using a discrete static symbolic mathematical representation.
\item To provide a graph theoretic definition of symbolic reference
  that maintains explicit distinctions between the existence of
  symbolic references and symbolic referents, enabling a grounded,
  recursively applicable, and non-tautological model of symbolic
  perception.
\item To mathematically clarify four of the most important yet
  potentially confusing distinctions in the reflective theory: (1)
  continuous dynamic activities, (2) discrete static symbolic
  references, (3) $n$ layers of distinct ordered sets of temporal
  relationships, and (4) the steps of computational simulation
  introduced in {\mbox{\autoref{part:the_implementation}}}.  In
  non-reflective theories, only one of these concepts may be needed,
  but these concepts are important to distinguish in this recursive
  model of thinking in order to maintain the grounded,
  non-tautological organization.
\end{enumerate}
This part serves the purpose of communicating the technical foundation
of the implementation.  The communication goals of this part cannot be
evaluated except in retrospect, but the foundational goals of this
part will be next evaluated in terms of the implementation goals.

\section{The Implementation}

The computational implementation described in
{\mbox{\autoref{part:the_implementation}}} serves the following
purposes:
\begin{enumerate}
\item To describe $n$ layers of procedurally reflective heuristic
  learning that do not affect the ``big O'' time complexity of the
  algorithm under focus.
\item To describe $n$ layers of planning machines that operate by
  executing a number of simple constant time, $O(1)$, planning
  activities that are combined in order to perform all search-like
  operations within the model, avoiding any primitives that would
  imply declarative or logical searches ``behind the scenes'', so that
  all search-like activities may be reflectively optimized.
\item To describe $n$ layers of efficient causal hypothesis learning
  and knowledge maintenance from the failure of counterfactual
  decisions back to bases in factual perceptual memories.  In this
  way, planning heuristics are defined to be hypotheses of planning
  machine operations by the reflective thinking layers of second-order
  and above.
\end{enumerate}
There are two key features that must be evaluated in the
implementation: (1) reflecting over a basic planning algorithm does
not slow down the basic planning algorithm, and (2) reflective
learning improves the efficiency of the basic planning algorithm.  I
will discuss how and when these two points are true in general.  Also,
there is a question as to increased space complexity required by each
layer of reflection.  Evaluations of these issues will be covered in
the following sections.

\section{No Theoretical Slowdown of Original Algorithm}

In order to assume that there is no theoretical slowdown of the
original planning algorithm when reflective heuristic learning is
applied, it must be assumed that the reflective implementation is an
ideal concurrent shared memory architecture.  In practice, the
underlying Funk virtual operating system slows down when more
concurrent and parallel fibers are executing.  This is beside the
theoretical point of this thesis, but the following table shows a
real-time test of the actual slowdown experienced by the Funk
operating system as different numbers of parallel fiber tasks are
executed to perform a simple numerical processing task.  The test was
done on a four core Pentium processor with hyper-threading, which
appear as eight processors to the Linux operating system underlying
the Funk virtual operating system:

\vspace{5mm}
\begin{tabular}{ll}
Tasks & Real-Time (s) \\
1 & 29\\
2 & 36\\
3 & 46\\
4 & 44\\
5 & 68\\
6 & 67\\
7 & 73\\
8 & 110\\
\end{tabular}
\vspace{5mm}

If a non-reflective learning algorithm uses one fiber and each
reflective learning algorithm uses an additional fiber, the Funk
virtual operating system underlying SALS will experience slowdown as
shown in this table.

\section{All Activities are Performed in Constant Time}

In order for all activities in my model to be reflectively optimized
by the $n$ layers of procedural reflective learning and planning, all
activities must be composed of primitive activities that are known to
complete in constant time, $O(1)$.  I have clearly described a
planning machine that operates based on constant time operations,
similar to a virtual machine.

In order to maintain this invariant in every aspect of the
implementation, I have needed to make some limiting assumptions in
order to avoid the general NP-complete problem of subgraph isomorphism
as a means of implementing symbolic perception.  If symbolic
perception is implemented as a subgraph isomorphism, then the
reflective optimization benefits of my model of reflective thinking no
longer apply.  In order to avoid this dangerous problem in my
implementation, I have made two temporary proof-of-concept simplifying
assumptions: (1) restricting learned symbolic perceptions to subgraphs
of 4 nodes and 4 edges or less of a constant upper-bound in
complexity, and (2) by hand-coding a small number of procedural
critics in the second-order reflective thinking layer that observe
specific properties of plans, such as similarly restricted subgraph
combinations of expected transframe additions and removals of plans in
different planning machine registers.

The general solution to the problem of symbolic perception in my model
is the planning-to-perceive problem as described by
\cite{pryorcollins:1995}.  In this way, the first-order planning
machine is responsible for creating plans that can be compiled into
perceptual resources that are composed of constant time, $O(1)$, steps
that respond to the perceptual event stream originating from the
layers below.  These plans must be composed of actions that traverse
and compare node and edge labels of the event stream as well as the
reflectively reconstructed representation of the layers below in order
to implement different planned ways to recognize the same symbol
depending on the order of the events in the stream.

It is imperative to not confuse the symbolic perception problem in my
model with the general NP-complete subgraph isomorphism problem.  In
this way, every aspect of my model, including symbolic perception,
recognizing isomorphic subgraphs in the layers below, becomes a
problem that can be optimized through procedural reflection, depending
on the order of reflective events.

\section{Efficiency Gains of Heuristic Learning}

\cite{mitchell:1997} defines the \emph{inductive learning hypothesis}
as follows:
\begin{definition}\emph{
\emph{The inductive learning hypothesis.} Any hypothesis found to
approximate the target function well over a sufficiently large set of
training examples will also approximate the target function well over
other unobserved examples.  }\end{definition} As the first-order
reflective learning algorithm learns hypotheses that generalize the
effects of actions from symbolic perceptions, the inductive learning
hypothesis is assumed of the target $\text{reflective}^0$ or physical
layer under the reflective focus.  In order to see efficiency gains in
the second-order reflective learning layer, I define here the
\emph{reflective inductive learning hypothesis}:
\begin{definition}\emph{
\emph{The reflective inductive learning hypothesis.} Any heuristic
found to approximate the target planning failure well over a
sufficiently large set of plan execution examples will also
approximate the target execution failure well over other unobserved
plan executions.}\end{definition} As long as the reflective inductive
learning hypothesis holds for a given planning domain, arbitrarily
greater planning efficiencies can be expected for each reflective
layer where this hypothesis holds true.

\section{Space Complexity}

\section{Conclusion}



