%************************************************
\chapter{Evaluation}
\label{chapter:evaulation}
%************************************************


\section{Streaming Constant-Time Symbolization}

In {\mbox{\autoref{part:the_simulation}}}, I define a symbolic
represenation to be a subgraph of the dynamic activity in the layers
below the static symbolic reference.  In general, this definition
becomes a subgraph isomorphism problem.  An algorithmic solution to
subgraph isomorphism is known to be an NP-complete problem, requiring
search.  Because my approach requires every activity in the
implementation to take a constant, $O(1)$, time, the symbolic
perception activity must also be composed of constant time components.
The way that this problem is handled in my architecture is by
considering the symbolic perception problem as a
\emph{planning-to-perceive} problem \cite[]{pryorcollins:1995}.  In my
proof-of-concept implementation, I only use graphs that are of a low
enough complexity in order to be bounded by a known constant upper
time boundary.  My implementation only creates new symbols for
those subgraphs that are below the size of four nodes and four edges.
I have hard-coded one two types of symbolic perceptions at the
second-order reflective level that focuses on plans in the focus and
execution registers.

In general, the subgraph isomorphism for specific symbolic perceptions
from the layers below becomes a planning problem for each newly
created symbol.  This allows the planning machine to create plans that
are compiled into parallel resources that can note the perception of
each newly created symbol, based on the reflective copy of the
physical knowledge-base.

