%************************************************
\chapter{Experiment Data}
\label{chapter:experiment_data}
%************************************************

This appendix chapter includes empirical data gathered using the
reflective tracing features of the underlying Funk garbage collected
memory system, upon which the virtual operating system is built.  This
low level memory tracing functionality allows easily tracing events at
any level of abstraction in the entire system using the Funk causal
reflective tracing features.  \autoref{table:bootup} shows a list of
figures showing data from an experiment where the AI boots up
and then perceives the world over time.  This experiment is a control
that shows that the AI memory usage and execution is stable if it does
not have any goals and the physical world is not changing.
\autoref{table:no_reflective_learning} shows a list of figures showing
data from an experiment with the reflective learning disabled.
\autoref{table:reflective_learning} shows a list of figures showing
data from an experiment with the AI reflectively tracing, and building
models of the deliberative planning process.  Comparing the execution
of the experiment with no reflective learning with that of reflective
learning shows a rough estimate of how much more memory is used when
the reflective learning about the deliberative planning machine is
enabled.  The evaluation chapter, \autoref{chapter:evaluation}, has a
more detailed analysis of the data in this appendix, comparing the
three experiments.

\experimentdatatable[bootup]{an experiment that boots up the AI and
  allows it to perceive the physical world and not stimulating it to
  achieve any goals} {This experiment serves as a control for the
  subsequent two experiments.  This experiment shows that when the AI
  is not given any deliberative goals, it learns from its initial
  perceptions, and when these do not change, it has nothing to learn
  and memory usage is stable.}

\experimentdatatable[no_reflective_learning]{an experiment without the
  AI reflectively tracing, and building models of the deliberative
  planning process} {This experiment serves to show the memory
  performance of the system when the reflective perception, learning,
  and prediction of deliberative actions is disabled.}

\experimentdatatable[reflective_learning]{an experiment with the AI
  reflectively tracing, and building models of the deliberative
  planning process} {This experiment serves as a demonstration of the
  complete AI system running, learning from its physical perceptions,
  trying to accomplish the given deliberative goal of stacking two
  blocks, and also reflectively perceiving and learning to predict the
  effects of its deliberative actions, including how planning can lead
  to execution failures.}


\clearpage
\section{Figures from Mind Bootup Experiment}

\experimentdatafigures[60]{bootup}{an experiment that boots up
  the AI, allowing it to continue to perceive the physical world over
  time without stimulating it to achieve any goals.  This experiment
  shows that the AI learns from its initial perceptions, and when
  these do not change, it has nothing to learn and memory usage is
  stable.}

\clearpage
\section{Figures from Learning Experiment without Reflection}

\experimentdatafigures[75]{no_reflective_learning}{an experiment with
  the reflective tracing of the deliberative process disabled.  This
  experiment can be compared with run-time behavior of the full AI
  cognitive architecture, which can learn from the failure and
  initiate a second attempt at plan selection and execution.}

\clearpage
\section{Figures from Reflective Learning Experiment}

\experimentdatafigures[180]{reflective_learning}{an experiment testing
  the full AI cognitive architecture, including the reflective tracing
  of the deliberative planning process, imagining the potential
  failures of the deliberative planning machine as well as the
  physical effects of physical actions.}

