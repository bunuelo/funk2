%************************************************
\chapter{Experiment Data}
\label{chapter:experiment_data}
%************************************************

This appendix chapter includes empirical data gathered using the
reflective tracing features of the underlying Funk garbage collected
memory system, upon which the virtual operating system is built.  This
low level memory tracing functionality allows easily tracing events at
any level of abstraction in the entire system using the Funk causal
reflective tracing features.  \autoref{table:reflective_learning}
shows a list of figures showing data from an experiment with the AI
reflectively tracing, and building models of the deliberative planning
process.  \autoref{table:no_reflective_learning} shows a list of
figures showing data from an experiment with the reflective learning
disabled.

\experimentdatatable[bootup]{an experiment that simply boots up the AI
  and simply allows it to perceive the physical world and not
  stimulating it to achieve any goals} {This experiment serves as a
  control for the subsequent two experiments.  This experiment shows
  that when the AI is not given any deliberaive goals, it simply
  learns from its initial perceptions, and when these do not change,
  it has nothing to learn and memory usage is stable.}

\experimentdatatable[reflective_learning]{an experiment with the AI
  reflectively tracing, and building models of the deliberative
  planning process} {This experiment serves as a demonstration of the
  complete AI system running, learning from its physical perceptions,
  trying to accomplish the given deliberative goal of stacking two
  blocks, and also reflectively perceiving and learning to predict the
  effects of its deliberative actions, including how planning can lead
  to execution failures.}

\experimentdatatable[no_reflective_learning]{an experiment without the
  AI reflectively tracing, and building models of the deliberative
  planning process} {This experiment serves to show the memory
  performance of the system when the reflective perception, learning,
  and prediction of deliberative actions is disabled.}


\clearpage
\section{Figures from Mind Bootup Experiment}

%\experimentdatafigures[18]{bootup}{an experiment that simply boots up
%  the AI, allowing it to perceive the physical world without
%  stimulating it to achieve any goals.  This experiment shows that the
%  AI learns from its initial perceptions, and when these do not
%  change, it has nothing to learn and memory usage is stable.}

\clearpage
\section{Figures from Reflective Learning Experiment}

\experimentdatafigures[90]{reflective_learning}{an experiment with the
  AI reflectively tracing, and building models of the deliberative
  planning process}

\clearpage
\section{Figures from Learning Experiment without Reflection}

%\experimentdatafigures[80]{no_reflective_learning}{an experiment
%  without the AI reflectively tracing, and building models of the
%  deliberative planning process}

