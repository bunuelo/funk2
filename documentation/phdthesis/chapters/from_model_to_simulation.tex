%************************************************
\chapter{From Model to Simulation}
\label{chapter:from_model_to_simulation}
%************************************************

\section{The End of the Model}

My description of the model of mind in plain non-technical English is
now complete.  The rest of this part will extend the model that I have
completed describing in English into the assumptions of simulation
that I will need for beginning a mathematical description in
\autoref{part:simulation}.

There is a difference between a model of mind and a simulation of a
model of mind.  I have presented my model of mind in plain English in
the first part of this dissertation.  Simulating the model does not
change the model.  At this point, I am assuming that I have
communicated my model in an understandable way.  I will now augment
this understanding with a separate auxiliary understanding of how to
simulate it.  It is important to explain how the model can be
simulated, so that I can subsequently quantify metrics for evaluation.
Further, understanding how to simulate the model is imperative to
understanding the computational implementation.

\section{Simulation Model}

I will use the term \emph{simulation model} to refer to the model that
changes during the activity of simulation.  It is imperative to
understand that the simulation model is not the same as the model that
I have previously described.  The model of mind does not change in
this dissertation.  The simulation model is an arrangement of symbols
that can be manipulated in order to simulate the model of mind.  The
simulation model is meant to change as the simulation proceeds.

The simulation model explicitly represents the changing aspects of the
model of mind as a model itself.  I have carefully left the activities
in my model of mind as the undescribed dynamic referent of a symbol.
The activities in my model are prior to their symbolization, so
assumptions are necessary for referring to these activities
symbolically in the simulation model.  Each assumption made in
symbolizing the ongoing activities in Duration restricts the
simulation from modelling those assumptions.

\section{Sets of Activities}

It is useful to assume that the activities in Duration exist as a
subset of things among all of the possible things that could possibly
exist.  In order to create a set of activities for the simulation
model, symbolic references must be enumerated that refer to the
activities in Duration that are to be simulated.



\section{Leftovers...}

\section{Examples of Activity in the Physical Layer}

In my English description, I have been careful to not include any
symbols in the physical layer at all; I have used descriptions that
include symbols to refer to example situations, but these example
situations are for the reader's understanding, these example
situations are not included in my model; these are descriptions of the
activity in the physical layer of my model, which does not have
symbols in it.
