%************************************************
\chapter{From Simulation to Implementation}
\label{chapter:from_simulation_to_implementation}
%************************************************

Now that I've described the theory of mind in
\autoref{part:theory_of_mind} of this dissertation, in
\autoref{part:the_implementation} I will extend this theory to the
models used for computational simulation.  I will begin with a
description of computation as an artificial tool of thought.  I will
discuss the relationship between a theory of computation and the
physical computer that is used to ease the use of the computational
model as a tool for thinking.  This chapter serves to transition the
focus from understanding the theory to understanding how the theory
relates to the absolute artificiality and, thus, potential utility of
computational models.  After this transitional chapter, I will have
established the theoretical language for describing my computational
implementation of the theory.

All of the software that is referenced in this part is an original
creation of the author and the source code is available on-line in a
form that can be downloaded, compiled, installed, and freely
manipulated.  See \autoref{appendix:the_code} for instructions on how
to access and use the code.

\section{Simulating Qualities of Duration}

Now, in moving from the theory to the computational implementation,
there are many dangers that must be avoided in order to not conflate
important theoretical distinctions in the implementation process.  The
primary danger is in confusing the distinction between the static
artificial Spatial arrangements of symbols and the dynamic ongoing
actual qualities of Duration.

\section{Digital Abstraction}

Computer simulations are based on the \emph{digital abstraction}, an
assumption that provides the most basic symbols of a computational
model, ``$1$'' and ``$0$''.  I will refer to the capacity to arrange
symbols in Space as \emph{computer memory}.  I will refer to a
specific arrangement of symbols in computer memory as \emph{data}.
While computer memory is often thought of as arising from physical
objects, such as transistors, the specific physical derivation of
these symbols is not important, given the theoretical assumption of
digital abstraction.  For example, computer memory can be written down
on paper and manipulated by hand, once the idea is understood, it is a
tool for thinking, separately from any physical implementation.  This
assumption is necessary when using computers in theoretical work.
Computer memory is an artificial static arrangement of symbols in
Space.

\section{The Combinational Device}

The digital abstraction is not all that is assumed when using a
computer to simulate a theory.  The digital abstraction provides the
static symbols in Space, so let us now focus on the dynamic qualities
of Duration.  In simulating a theory with a computer, there is a
predetermined discrete transition from the past to the future.  In the
past, there is an arrangement of symbols in Space and there is a
predetermined activity in Duration that causes these symbols in Space
to be arranged in a different specific configuration in the future.
This discrete activity is designed to be theoretically exactly the
same for every transition from the past to the future.  This logical
predetermined transition is referred to as \emph{the combinational
  device}.  The combinational device maps any specific combination of
symbols Spatially arranged in the past to a specific Spatial
arrangement in the future.

\section{Digital Reflection}
\label{section:digital_reflection}

Because the combinational device is always understood to be the exact
same theoretical activity, there is not a non-tautological reason to
symbolically reference this activity in the computer memory.  Because
this transitional activity is always understood to be exactly the
same, the symbolic cause that refers to the activity during the
transition from the past to the future would be theoretically
predefined and thus always the same, making no distinctions.  I will
refer to the ability to symbolize the ongoing dynamic quality of the
digital transition from the past to the future as \emph{digital
  reflection}.  As I've described, it is a theoretically meaningless
and tautological pursuit for computers to exhibit digital reflection.
Digital reflection is one of two forms of actual reflection that are
available to a computer.

\section{The Programmer}

Symbols have meaning only in reference to the ongoing qualities of
Duration.  So, then, the question becomes: if digital reflection is a
meaningless tautological pursuit, to what qualities of Duration do
digital symbols meaningfully and non-tautologically refer?  The
symbols in computer memory are programmed by a programmer that uses
these symbols as references to the qualities of Duration that are
actively ongoing and currently being experienced.  The computer is an
artificial tool of thought for the programmer, who provides the
meaningful reference to reality for the symbols in the computer
memory.  Again, a computer does not create symbols that reference
reality; the programmer is the reference to reality, while the
computer is an artificial tool of thought.
