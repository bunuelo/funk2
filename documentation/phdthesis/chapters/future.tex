%*****************************************
\chapter{Future}\label{chapter:future}
%*****************************************

%% \section{from related models (EM-ONE)}

%% The EM-ONE architecture does not have an explicit self-reflective
%% layer that contains these types of self-reflective thinking.  The SALS
%% AI makes an attempt to keep a clear distinction between the types of
%% knowledge that are available to each reflective layer of mind.  The
%% SALS AI does not contain self-reflective knowledge in its deliberative
%% or reflective layers of thinking because if this knowledge is allowed
%% to exist in these lower layers, this introduces ``loopy'' knowledge
%% references that confuse the purely hierarchical control structure that
%% currently exists in the SALS AI.  For example, the deliberative layer
%% of the SALS AI can only manipulate physical knowledge as the
%% reflective layer in the SALS AI can only reference deliberative
%% knowledge and indirectly reference the physical knowledge that the
%% deliberative knowledge references.  The fact that the reflective layer
%% in the SALS AI cannot reference higher layers of reflective knowledge,
%% such as self-reflective knowledge, is consistent with the SALS
%% architecture's strictly hierarchical knowledge organization between
%% reflective layers of control.  The SALS AI is currently limited to the
%% knowledge of the bottom four layers of the Emotion Machine
%% architecture.

%% \section{from related models (metacognition)}

%% The Emotion Machine theory makes a distinction between layers of
%% reflective knowledge and self-reflective knowledge that is not clearly
%% distinguished in the metacognitive theory.  The implemented SALS
%% architecture is currently only an implementation of reflective
%% thinking but implementing self-reflective thinking, what the cognitive
%% science literature would refer to as ``theory of mind,'' will be
%% described as a future extension of the SALS architecture in
%% {\mbox{\autoref{chapter:future}}}.  The Emotion Machine theory of
%% self-reflective thinking is a higher-level form of thinking than
%% simply reflecting and controlling a deliberative or object level
%% thought process.  The Emotion Machine theory of self-reflective
%% thinking describes how humans think of themselves and others in terms
%% of abstract models of their ``selves,'' different subpersonalities
%% that depend on the current problem solving or social context.
%% \cite{minsky:2006} begins his description of ``selves'' as analogous
%% to how one might think of others:
%% \begin{quote}
%% How do people construct their Self-models?  We'll start by asking
%% simpler questions about how we describe our acquaintances.  Thus, when
%% Charles tries to think about his friend Joan, he might begin by
%% describing some of her characteristics.  These could include his ideas
%% about: the appearance of Joan's body and face, the range and qualities
%% of her abilities, her motives, goals, aversions, and tastes, the ways
%% in which she is disposed to behave, her various roles in the social
%% world.
%% \end{quote}
%% {\mbox{\autoref{figure:emotion_machine_multiple_models_of_self}}}
%% shows a few examples of different subpersonalities or ``selves'' that
%% Minsky explains a person may use to think in different contexts.
%% \begin{figure}
%% \centering
%% \includegraphics[width=8cm]{gfx/emotion_machine_multiple_models_of_self}
%% \caption{Multiple models of self.}
%% \label{figure:emotion_machine_multiple_models_of_self}
%% \end{figure}
%% The metacognitive meta-level could be said to map to both reflective
%% and self-reflective layers of thinking in the Emotion Machine.


%% \section{old beginning}


%% The Emotion Machine cognitive architecture has interesting
%% implications for the future of AI.  The ability to reflectively learn
%% in $n$ layers has not yet been shown, but the linear slowdown of this
%% implementation with the addition of reflection, along with the
%% potential for no slowdown on ideal concurrent shared memory
%% architectures point toward learning much more useful information in
%% one-shot learning situations, where learning is costly or dangerous.

%% \section{Learning Object and Subject Distinctions}
%% \label{section:objective_reflection}

%% Although this model is based upon an assumption of objects and their
%% relationships, the ability of the AI to learn new object type
%% abstractions and subjective perspectives is future research for the
%% architecture.  Having the architecture abstract its own types of
%% objects and ways of using these objects would be useful in the
%% deliberative layer for learning physical object types, but the ability
%% to abstract object types and subjective perspectives becomes more
%% interesting in the metacognitive layers because this may lead to
%% concepts of problems as types of objects that may be solved through
%% different representational perspectives.

%% The ability to abstract object and subject distinctions within the
%% mind is a key that may lead to the abstraction of a concept of
%% ``self'' objects within the AI, as it makes distinctions between those
%% physical objects that it thinks of as its body versus those physical
%% objects that it thinks about as an environment.  The most important
%% and the first subject and object distinctions may be those that help
%% accomplish or avoid physical goals.  The separation between the
%% physical body and the physical world is an objective form of thinking
%% that could partially be caused by goals that emphasize distinguishing
%% bodily and worldly goals in the causal structures of the physical type
%% knowledge.  For example, \cite{bongard:2006} demonstrates an approach
%% referred to as ``motor babbling'' that uses a non-reflective model in
%% order to learn a physical self-model.

%% \section{The Self-Conscious Layer and Social Goals}

%% If we are going to develop a theory of social goals, how do we begin
%% to model the cooperative and non-cooperative thought processes that
%% would constitute working toward or against other people? It has been
%% shown that the neural circuits that are involved in how someone thinks
%% about what someone else knows are largely distinct from the circuits
%% involved in perceptual capabilities \cite[]{bedny:2009}. There have
%% also been distinctions shown between the neural circuits involved in
%% how someone thinks about what someone else knows and executive control
%% of actions \cite[]{saxe:2006}. There is good reason to believe that
%% the learning of social behavior is bootstrapped from learning causal
%% physical models \cite[]{perner:1991}.

%% \section{A Physically Grounded Social Problem Domain}

%% The IsisWorld simulation \cite[]{smith:2010} is a rigid-body physics
%% simulation that is capable of handling semantic commonsense objects
%% and actions involved in kitchen cooking tasks.  In order to help us
%% define a grounded problem domain, I have chosen to focus on a
%% rigid-body physical simulation of parents and children in the context
%% of basic cooking tasks in a kitchen environment. I chose parents and
%% children because I feel that major parts of social learning develops
%% during early stages involving these familiar social relationships. I
%% chose the domain of basic kitchen cooking tasks because this is a
%% non-trivial social commonsense reasoning domain that exists in some
%% form in all human cultures.

%% \section{The Self/Other Distinction and Self-Reflective Thinking}

%% I have described \cite[]{morgan:2011} a theory of how self-reflective
%% states are communicated or inferred in other agents by correlating a
%% physical perceptual stream with a reflective perceptual stream,
%% allowing the agent to use these correlations to predict the reflective
%% states of other agents based purely on their physical actions. For
%% example, if one agent performs a series of physical actions in order
%% to accomplish a goal, such as preparing a slice of buttered toast
%% using a knife and a loaf of bread and some butter in a kitchen, then
%% these actions have been correlated with this agent's reflective
%% states, including the current goals that the agent is pursuing. If
%% this agent then sees the physical world changing, and he knows that he
%% is not causing those changes, then he can attempt to infer the state
%% of mind of another agent. This type of inference process, based on the
%% previous correlations of physical and reflective perceptual streams,
%% is what could be referred to as a self-reflective thought process. A
%% self-reflective thought process introduces the social distinction
%% between Self and Other reflective state knowledge that is inferred by
%% using the previously learned correlations between physical perceptions
%% and reflective perceptions.  This is a type of self-reflective
%% knowledge that is knowledge that allows thinking about what someone
%% else is thinking about.  Sometimes this self-reflective knowledge is
%% referred to as "theory of mind" knowledge in the cognitive science
%% literature.

%% With self-reflective knowlege, an agent can choose to act
%% altruistically by helping another agent accomplish their goals or
%% maliciously by working against their goals. However, there is the
%% possibility that an agent is not always running a self-reflective
%% process that pays attention to another agent's physical actions in
%% order to infer what their goals might be. In many cases, such as in
%% the IsisWorld physical simulation of parents and children in a
%% kitchen, some agents might focus single-mindedly on a novel task
%% without thinking about what the goals of surrounding agents might
%% be. This type of model allows for an agent that works against the
%% goals of another agent without knowing the goals of the other
%% agent. One might argue that this type of situation where one agent
%% clobbers the goals of another agent is due to a form of
%% self-reflective laziness and not due to malicious intentions.

%% \section{Guilt, Pride, and Self-Conscious Thinking}

%% It is interesting to consider how children learn from parents as well
%% as how parents guide a child's cooperative behavior, but to study this
%% type of thinking, the AI must be extended in order to allow for
%% modelling what one agent thinks about what another agent thinks about
%% her.  This form of double reflective recursion is self-conscious
%% knowledge and the problems that are represented in this type of
%% knowledge are handled by what is called self-conscious thought
%% processes. For example, a little boy may be cooking something in the
%% kitchen and he may carelessly work against the goals of his sister;
%% this situation may be recognized by an onlooking parent, and the
%% parent may inform the boy of a self-reflective mistake he has made:
%% "Ralph, your sister was going to use some of that butter that you just
%% finished." The little boy, Ralph in this case, thinks that his mother
%% thinks that he made a self-reflective mistake. I hypothesize that
%% modelling this form of double reflective recursion in conjunction with
%% respectful social relationships may be useful for understanding
%% emotions such as Guilt and Pride.  I have described
%% \cite[]{morgan:2010} a theory of children could learn cooperative
%% behavior in the context of a supervisory parent that makes corrections
%% to different layers of knowledge within the child's mind, up to
%% self-reflective knowledge.

%% \section{Imprimers}

%% \cite{minsky:2006} explains that terms like Guilt and Pride can refer
%% to self-conscious ways of thinking.  Minsky hypothesizes that there
%% are powerful ways of thinking that result when someone whom one
%% respects, like a parent, values or devalues their goals.  This type of
%% relationship between object models is named an \emph{imprimer
%%   relationship} by Minsky.  In this relationship, the parent plays the
%% role of what Minsky has named an \emph{imprimer}, in reference to the
%% ability of ducklings to learn, relatively arbitrarily, to trust and
%% follow a parent-like figure.  He hypothesizes that imprimers, such as
%% parents and caregivers, play an important role in unquestioned
%% knowledge inheritance in children.  This type of thinking about what
%% someone else is thinking about their goals may require at least two
%% recursions of self-reflective abstraction and thinking.

