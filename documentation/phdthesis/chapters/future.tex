%*****************************************
\chapter{Future}\label{chapter:future}
%*****************************************

The reflective model has interesting implications for the future of
symbolic approaches to AI.  An explicit acknowledgment of symbols
being references to the dynamic ongoing activity implies this layered
model that I have described.

\section{Propogating Knowledge Dependencies}

Propagators, described by \cite{radul_and_sussman:2009}, could be
organized into reflective layers that would allow propagator networks
to learn better propagate and merge functions.  The main obstacle in
combining the propagator model with this reflective model is that the
dynamic components of the propagator model must be explicitly
represented as knowledge within another propagator model.  In a
propagator network, this would require that cells contain knowledge
learned from watching the execution of propagate and merge functions
of other cells.  Hypotheses can be learned from watching the propagate
and merge functions execute, and these hypotheses can then be
propagated in a layer above.  Layers of propagator cells allows
debugging the representations for how these functions could be changed
in order to lead to better ground network performance.  In a
reflective model, there is a necessary explicit separation between the
static and dynamic components of the model.  Without strict reflective
layers of knowledge, the model becomes tautological as soon as a
representation is made of the dynamic component of itself.

\section{Self Reflection}
\label{section:objective_reflection}

Viewing the physical activities as physical objects is future research
for the model.  Thinking about subjective perspectives on physical
objects and their relationships is future research in the model.

The ability to abstract object and subject distinctions within the
mind is key to how I see the creation of selves.  The most important
and the first subject and object distinctions are those that help
accomplish or avoid physical goals.  In creating the distinction that
allows the mind to see the physical body as subjected to the objects
of a physical world, the mind has created one of the first subjective
perspectives, the physical body.  An object has subjective
perspectives and extents in Space.  Further, every subjective
perspective would usefully have a set of causal hypotheses that lead
to and away from the object's more central subjective perspectives.
Objects, therefore, become collections of subjective perspectives that
can be easily traversed given the included causal models.  The extent
of an object in Space gives a useful Spatial arrangement for causal
hypotheses that help to create controllable boundaries around similar
perceptual and goal activities.

\section{Circular Objects}

Causal hypotheses are used in order to predict the future occurrence
of symbolic perceptions and goals.  Through reflective thinking, plans
are constructed from causal hypotheses consistent with the past,
elaborating the necessities in the past and the results in the future.
Analogies between consistent plans can be abstracted into models of
objects.  For example, a plan that begins and ends with the same
symbolic perception could be considered an example of a ``circular''
object; thus, an analogical plan abstraction would represent a type of
object, that has multiple sides to perceive depending on the subject's
position in the plan.  Circular objects may be an important type of
object to analogically recognize because a circular object allows one
to perform actions while being able to get back to a known perceptual
symbol.  If one is in a known circular object, then one is not lost in
the sense that one can always follow the circle in order to get back
to any perceptual symbol contained within the circular plan object.
Although, the implementation includes tools for performing analogical
abstractions and constructions of plans, this machinery is not key to
the thesis of explaining causal reflective learning to accomplish
goals.  This section is included in the dissertation to eliminate a
potential confusion that would conflate the concept of symbol with
that of object.  Symbols are the most primitive elements available to
thinking, while objects are more complicated in that they are composed
of analogical consistencies between plans that are themselves composed
of many symbols.  Objects have multiple subjective perspectives.
Objects are static creations of a mind, based on static symbols
representing the ongoing activity in Duration.

\section{Body and World}

An intelligent mind builds models that predict the occurrence of
perceptual symbols.  Analogical thinking is used to abstract objects
and simultaneously the implied subjective perspectives on these
objects as they are manipulated.  The separation between the physical
body and the physical world is an objective form of thinking that is
fundamentally caused by goals that emphasize distinguishing bodily and
worldly goals in the causal structures of the physical perceptions.
For example, physical actions that change physical perceptions in
predictable ways are often bodily physical perceptions, i.e. moving
ones hand in front of ones face causes one to consistently see a hand.
Also, physical pain perceptions could be related to physical goals
that emphasize the static separation between the body and the world.
Understanding the static separation of the body and the world allows
for an objective form of scientific study that places the object of
study in the world outside of the effects of the body.

\section{$m^\text{th}$-Stratum Self Reflection}

\label{section:model_6_future_research}

After discussing strong parallels between my model and Minsky's
six-layered reflective model of mind, called The Emotion Machine or
Model-6, in \autoref{backreference:self_reflective_self_conscious}, I
will continue a description of the top two layers of Minsky's model in
the terms of my model.

Above this ability to model multiple interacting objects is the
ability to model objects as containing objects of the self-same type.
The ability to abstract all of the layers of a reflective mind into
one type of object or another can be used to recursively model one
object's model of another object.  In learning to abstract subjective
perspectives on objects from the mind, the distinction between the
body and the world can be imagined to be one of the most important
fundamental subject and object distinctions that would be made by such
a system.  In my model, abstracting part of the mind to be an object
and another part to be a subject is the same as creating a ``self''
model.  In other words, the initial objective abstractions in this
model would be a distinction between a subjective self and an
objective physical world.  Therefore, although my model does not
include self reflection, I see this form of reflection as along a new
dimension that has, as its origin, all of the reflective layers that I
have so far described in my model.  I will refer to this new dimension
as ordered \emph{strata} of the \emph{self dimension of reflective
  thinking}.

\begin{align}
\label{equation:self_0_reflective_stratum}
\text{self}^0\text{ reflective stratum } &{\approx} \bigcup_{n=0}^{\infty}{\text{reflective}^n\text{ layer}} \\
\label{equation:objective_m_reflective_stratum}
\text{self}^m\text{ reflective stratum } &{\approx} \bigcup_{n=0}^{\infty}{\text{self}^m\text{ reflective}^n\text{ layer}}
\end{align}

Prior to the first objective thinking stratum existing, it must have a
prior part of the mind upon which to objectively reflect.  I will
refer to all of the reflective layers I've described so far as the
zeroth-stratum of objective thinking, the ``$\text{self}^0$''
stratum, since it contains no objective thinking at all.  Therefore,
the $m^\text{th}$-stratum of objective thinking in my model can be
described as ``$\text{self}^m$'' reflective thinking.  Given this
notation, the self-reflective layer of Minsky's model would be the
first-stratum of objective thinking, or $\text{self}^1$ reflection,
in my model.  Each $\text{self}^m$ reflective stratum in my model
contains an infinite number of $\text{self}^m\text{ reflective}^n$
layers.
Equations~\ref{equation:objective_0_reflective_stratum}~and~\ref{equation:objective_m_reflective_stratum},
roughly and non-mathematically show how each objective thinking
stratum is composed of an infinite number of reflective layers.
Again, this notation is \emph{not} set theoretic, and thus, set
theoretical mathematics do not apply.

\begin{figure}[bth]
  \begin{align*}
    \left.
    \begin{array}{l}
      \text{Minsky's Problem Domain }\\
      \text{Minsky's Built-in Reactive Layer }\\
      \text{Minsky's Learned Reactive Layer }\\
      \text{Minsky's Deliberative Layer }\\
      \text{Minsky's Reflective Layer }\\
    \end{array}
    \right\}                               &{\approx} \bigcup_{n=0}^{\infty}{\text{self}^0\text{ reflective}^n\text{ layer}} \\
                                           &{\approx} \text{ self}^0\text{ reflective stratum} \\
  \end{align*}
  \begin{align*}
    \text{Minsky's Self-Reflective Layer } &{\approx} \bigcup_{n=0}^{\infty}{\text{self}^1\text{ reflective}^n\text{ layer}} \\
                                           &{\approx} \text{ self}^1\text{ reflective stratum} \\
  \end{align*}
  \begin{align*}
    \text{Minsky's Self-Conscious Layer }  &{\approx} \bigcup_{n=0}^{\infty}{\text{self}^2\text{ reflective}^n\text{ layer}} \\
                                           &{\approx} \text{ self}^2\text{ reflective stratum}
  \end{align*}
\caption{The top layers of Model-6 roughly mapped to the suggested
  $\text{self}^m\text{ reflective}^n$ stratum notation.}
\label{figure:model_6_as_reflective_stratum_notation}
\end{figure}

Minsky's ``self-conscious'' layer would thereby roughly correspond
with the $\text{self}^2$ reflective stratum.  Having two levels of
recursion in an objective thinking sense, leads to the ability to
represent what one object references in terms of what another object
references in terms of the original object.
\autoref{figure:model_6_as_reflective_stratum_notation} shows roughly
how the levels of Minsky's model can be related to the suggested
stratum notation.

\section{Imprimers}

\cite{minsky:2006} explains that terms like ``guilt'' and ``pride''
can refer to self-conscious ways of thinking.  Minsky hypothesizes
that there are powerful ways of thinking that result when someone whom
one respects, like a parent, values or devalues their goals.  This
type of relationship between object models is named an \emph{imprimer
  relationship} by Minsky.  In this relationship, the parent plays the
role of what Minsky has named an \emph{imprimer}, in reference to the
ability of ducklings to learn, relatively arbitrarily, to trust and
follow a parent-like figure.  He hypothesizes that imprimers, such as
parents and caregivers, play an important role in unquestioned
knowledge inheritance in children.  This type of thinking about what
someone else is thinking about their goals requires at least two
strata of self reflection in my model.

