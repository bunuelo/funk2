%************************************************
\chapter{Introducing Simulation}
\label{chapter:introducting_simulation}
%************************************************

It is imperative to understand that the \emph{simulation model} is not
the same as the model of mind that was previously described.  The
simulation model is an extension of the model of mind that includes a
mathematical description of a discrete ``state'' of the model in a
discrete time.  Neither the model of mind nor the simulation model
changes over time.  The simulation model therefore makes reference to
a discretely stepped state that is used to simulate the necessarily
separate, static model of mind.  The term ``simulation'' is thus used
as a general reference to the dynamic activity in Duration that
manipulates and steps the state of the simulation model.  In
describing the simulation model, it is important to not confuse the
dynamic activity of simulation with the model or the state of the
model, which are both static at any given point in time.  The term
``simulation'' is the only reference to activities in Duration.  This
distinction is necessary for the simulation model to not be limited to
simulating a specific kind of activity.

For example, consider a simulation of topological proof.  A
mathematician can ``simulate'' the rules of topological proof by first
seeing a static arrangement of symbols on paper.  These symbols can be
manipulated in the process of proving or disproving the initial
topological statement.  There is no existing logical or mathematical
formulation of the activities of topological proof.  I describe how a
symbolic state space can be added to the model of mind for the
purposes of simulation.  How exactly this simulation is done is left
until the next part, where the activity of simulation is assumed to be
computational.

The activities in Duration in the model of mind exist prior to their
symbolization, and since it is obviously not possible to refer to
something prior to symbolizing it, the assumption that these
activities have already been symbolized is necessary in order to
mathematically describe the state of the simulation model.  Defining
the simulation model requires a description in terms of symbols for
the ``undescribed activities in Duration.''  Each assumption made in
symbolizing the ongoing activities in Duration restricts the
simulation model from being a model of those assumptions because only
the symbolic result of those assumptions is available for the
simulation of the activity of reflection.

