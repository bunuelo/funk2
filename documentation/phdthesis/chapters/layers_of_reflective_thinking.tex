%************************************************
\chapter{Layers of Reflective Thinking}
\label{chapter:layers_of_reflective_thinking}
%************************************************

\section{First-order Reflective Thinking}

After constructing causal hypotheses that predict the necessities and
results of activating resources, these causal models are used to
construct larger structures called plans.  Plans are combinations of
causal models that contain inferences of past necessities and future
results.  For example, one could imagine a sequence of resources that
leads through a counterfactual future sequence of perceptions,
resources, and goals.  In the literature, these counterfactual
constructions are referred to by a number of names, including: case,
explanation, fiction, narrative, and story.  Stories and narratives
often include a social self-reflective representation, which I see as
future work in the development of my model, which I will discuss in
\autoref{chapter:future}.  I will simply use the term \emph{plan} to
refer to all of these constructions that combine causal models into
counterfactual views of the past and future.

First-order reflective thinking is the ongoing activity in Duration
that is limited to manipulating symbolic references to physical
activities.  First-order reflective thinking symbolizes physical
activities, creates physical causal hypotheses, and uses these
hypotheses to create plans toward or away from physical goals.
Further, first-order reflective thinking activities execute these
plans composed of physical resource causal symbolic references.

\section{$n^\text{th}$-Order Reflective Thinking}

I've previously described three layers of activity in my model: the
physical layer, the first-order reflective thinking layer, and the
second-order reflective thinking layer.  For brevity, and to emphasize
the layered nature of my model, I will sometimes use a superscript
notation ``$\text{reflective}^n$'' to refer to the $n^\text{th}$-order
of a layer in my model.  For example, the $\text{reflective}^1$ layer
will refer to first-order reflective activities, and the
$\text{reflective}^2$ layer will refer to second-order reflective
activities.  Although the physical layer is not a reflective thinking
layer at all, its place as the prior reference for the first-order
reflective thinking activities moves me to inductively extend this
notation to allow referring to the physical layer as the
$\text{reflective}^0$ layer or the \emph{zeroth-order reflective
  layer}.  Zero thus serves as the implicit origin of an infinite
regress of $n$ layers.  In this sense, the implementation of my model
of reflective thinking includes activities of reflective order zero
through reflective order two.

\section{Limitations of First-order Reflective Thinking}

The act of creating a symbolic reference to physical activity is not
accessible to the first-order reflective layer.  This is critical to
an understanding of my model.  A creative first-order reflective layer
action, such as creating a plan toward a physical goal, is limited to
creating and Spatially arranging symbols that refer to physical
activities.  The first-order reflective layer cannot manipulate
symbols that refer to its own activities without reducing itself to
something hopelessly meaningless and tautological.

Another critical point to understand is that first-order reflective
thinking does not directly manipulate physical activities.  Neither
symbols nor Space are physically actual; instead, symbolic references
to physical activity and the relationships that order them in Space
are effects and not causes.  As artificial creations of an independent
first-order reflective layer they are only derivative.  First-order
thinking is about physical activities and is limited to the terms of
its own created artificial symbolic references to the actual physical
layer.  In contrast, the physical layer does not include any of the
artificial constructions of thinking, neither symbols nor Spatial
arrangements.  This is because the physical layer does not include
thinking, which is fundamentally a reflective activity.

\section{Second-order Reflection on Symbolization}

First-order reflective thinking activities create, Spatially arrange,
and otherwise manipulate symbolic references to physical activities.
The activity of symbolization is a primary aspect of every reflective
thinking layer; however, given the logical limitation that reflective
thinking layers do not refer to their own activities with the
exception of active Spatial arrangements, a second-order reflective
thinking layer is necessarily required for creating symbolic
references to the activity of symbolization.  Note that the activity
of reflecting on symbolization appears in all layers above the
first-order reflective thinking layer, but this activity of reflecting
on the activity of symbolization necessarily first appears only in the
second-order reflective thinking layer.

\section{Non-existence of Symbolic References}

Reflecting on symbolization is one the primary activities of the
second-order reflective layer, resulting in the primary second-order
symbols that refer only retrospectively to first-order activity.
Having a symbolic reference to first-order symbolization allows the
consideration of this activity to be symbolized as a second-order
resource.  Considering first-order symbolization as a second-order
resource thus allows a second-order causal hypothesis to be created
with this resource as the ongoing cause of a transition.  This is key.
In other words, the transition is from the ``non-existence'' to the
purely retrospective necessity of the first-order symbolic reference
to physical activity.

Note that non-existence only fulfills the logical function of a
placeholder in the causal model, which refers only to the necessity of
an alternative to creation and existence itself.  Thus, I've begun my
description with the assumption that what exists is my reference, and
that what exists does not require symbols to refer to itself or
describe itself.  However, in reflecting on and symbolizing the
activity of symbolization, I have implicitly assumed this placeholder,
``non-existence'', in the causal model that allows second-order
reflective thinking about the existence of first-order symbolic
references.  Non-existence refers to the past slot in the second-order
causal model of first-order symbolization.  Causal hypotheses and
their symbolic parts are artificial constructions, so non-existence
can only be a reference to part of an artificial construction.
Non-existence thus becomes a necessary modelling tool for second-order
reflective thinking.  Non-existence does not refer to a primary
quality of Duration, non-existence refers to an artificial Spatial
arrangement of symbols in the second-order reflective layer.

\section{Planning Symbolic Refinement}

There is a very useful opportunity here to temporally extend the
causal model in the future in order to predict the creation of new
symbols based on previous symbolized perceptions or resources.  This
could lead to second-order reflective thinking creating plans for the
first-order thinking layer to create new and more refined symbols that
may be useful for either accomplishing or avoiding physical goals.
This is a useful and powerful type of second-order reflective
planning.

\section{The Distractive Nature of Non-existence}

Non-existence as a tool of thought is as dangerous as it is powerful.
The danger is that non-existence can be incorrectly considered to be
\emph{the preexisting reality}.  In my model, I have defined reality
to be everything that exists as ongoing in Duration, maintaining an
awareness of the absurdity of symbolizing something that is prior to
symbols.  Said another way, the danger is in forgetting that
non-existence is a reference to the past slot in an artificial causal
construction of at least order two.  This danger of mistaking
non-existence for the prior reality is insidious, distracting, and
tempting because of its false promise of an explanation for creation
and existence itself.

For example, the first-order reflective layer can create the
perceptual symbol ``green'' to refer the current undescribed physical
activities in Duration.  These ongoing physical activities do not
contain symbols and I have been very careful to not attempt to provide
a description of that which cannot be defined in terms of artificial
symbols, the activities ongoing in Duration, which are neither
artificial nor in artificial terms.  Therefore, the symbol ``green''
is an artificial creation that refers to these undefined activities,
which are the given qualities of Duration.  They require neither
symbolization nor thinking to exist.

By means of this new methodological notion of non-existence, a
second-order reflective thinking activity is now able to create a
causal model of this first-order symbolization, the creation of the
symbol ``green''.  This would not have been possible if non-existence
were allowed to function as a form of ``pre-existing reality''.  In
other words, this second-order activity implies that before the
creation of the symbol ``green'' there was a paradoxical existence of
the non-existence of the symbol ``green''.  The fact is that the
non-existence of the symbol is actually only the secondary result of
the creation of the symbol, so we must not be confused when the
second-order causal model places the non-existence of the symbol in
the past slot.  In this way the model remains cognizant of the fact
that non-existence of the symbol is only an artificial tool of
thought.  ``Active non-existence'' is equally as absurd as ``inactive
existence''; actual non-existence is a tempting but simple
contradiction.

\section{Third-order Reflection on Existence}

The second-order reflective layer creates causal models of the
creation of symbolic references.  These second-order artificial
constructions contain the primary references to existence and
non-existence.  Although my implementation does not include a
third-order reflective layer, my model includes the potential for an
arbitrary number of layers.  In order to briefly describe the
potential for third-order reflective thinking, let us consider a
causal model that would be the creation of the third-order reflective
layer.

For example, let the second-order activity of creating a causal
hypothesis be symbolized as a symbolic resource in the third-order
reflective layer.  This symbolic resource could refer to the
second-order creation of a causal hypothesis describing the
first-order creation of a symbolic reference to physical activity.
The symbolic resource that refers to the first-order creation of the
symbol is placed in the past slot of this third-order causal model.
In the future slot of this third-order causal model is a symbolic
resource that refers to the second-order creation of the causal model
that introduces the concept of existence being created out of
non-existence.  Note that the power of third-order reflection allows
the creation of the causal hypothesis that correctly places the
creation of the symbol prior to the simultaneous creation of the
knowledge of the existence and non-existence of the symbol.

\section{``Sub-symbolic Thinking''}

I have previously discussed the danger of having the second-order
reflective ability to create a causal model that predicts the
existence of symbolic perceptions from non-existence.  The danger is
in believing that the non-existence of a symbolic perception
represents some kind of actual pre-reflective mechanism that produces
symbols.  It doesn't.  There is no such thing as a pre-reflective or
pre-symbolic symbol or model.  Reflection creates symbols and models
that refer to reality.  In other words, models must necessarily model
something other than their own activity.  Thus, it is important to
understand that a causal model that predicts the creation of symbols
is the result of a second-order reflective activity, more abstract and
artificial than the first-order creation of the symbol.

Many causal models that have been developed to describe the creation
of perceptual symbols.  Many of these are based on the artificial
extension of physically objective assumptions, therefore allowing one
to dissect brains and build models of perceptions based on such things
as neural networks.  These neural network models are sometimes said to
be ``sub-symbolic''.  It is important to realize that the integration
is totally illusive, that the idea of a model being sub-symbolic is a
contradiction in terms because all models are built from symbols.
These so called sub-symbolic models are actually, at best, only at the
level of second-order reflection, which is the first level of
reflection that is able to model the creation of a symbolic
perception.  I think that the current trend in referring to these
types of models, that in fact only actually appear to the second-order
reflective thinking layer, and that further claim to be able to
predict perceptions as sub-symbolic, is a confusing and distracting
phrase in the field of AI.  How to think reflectively about neural
networks has not yet been described in the field of AI because neural
networks are very complex numerical models that require at least a
second-order level of reflective thinking, if they are to be
understood at all as a model of the creation of symbols.

