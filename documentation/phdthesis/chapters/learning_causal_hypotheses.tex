%************************************************
\chapter{Learning Causal Hypotheses}
\label{chapter:learning_causal_hypotheses}
%************************************************

\section{Resources}

Activities in Duration can be referred to symbolically by the
reflective thinking layers.  While all activities in all layers are
generally willful qualities of Duration, these activities can be
symbolized as potentially actionable parts of plans by the reflective
thinking layers.  I will refer to a symbolic reference to activities
that can be put into plans as a \emph{resource}.  Resources exist in
the thinking layers as symbolic references to activities in the layers
below.

\section{Activation and Suppression}

In my model, I've included a logical idea that I refer to as
\emph{suppression}.  Suppression is a symbolic relationship to a
resource that can be put into plans.  The idea of suppression is a
subtle point with respect to the qualities of Duration.  The basic
problem is that the qualities of Duration are the willful activities
that exist independently without thinking existing as any implicit,
co-existent necessity.  Further, the activities in Duration cannot be
inactive, by definition.  This is important.  In other words, a
symbolic reference to something inactive would imply a symbol that
refers to something that does not exist, or, in other words, only
exist as a contradiction.  Therefore, the logical idea of suppression
is only an artificial tool of thought, part of the thinking layers,
entirely separate from physical activities.  The subtle point here is
that suppression does not disable the potential for willful
activities.  Suppression is only a logical Spatial relationship
including the symbolic reference to the physical activity in question.
I will refer to the logical alternative to suppression as
\emph{activation}.  Activation and suppression refer to types of
assumed Spatial relationships that are maintained between symbolic
resources.  Therefore, it would not be correct to say that physical
activities have been activated or suppressed, but alternatively, it
would be correct to say that a resource is in an ongoing, unstoppable,
dynamic Spatial relationship that has activated or suppressed
qualities.  The activation and suppression of resources occurs in the
sense of actively creating Spatial relationships that represent the
logical qualities of activation or suppression, which have the
potential to be logically consistent or contradictory.

Therefore, in my model, a resource can be both either activated or
suppressed, which does not, by itself, imply a resultant activity in
the layer below; logically, if a resource is activated and is not
otherwise suppressed, then the resource is considered logically to be
activated; however, when a resource is both activated and suppressed
simultaneously, this is a logical failure that is cause for a plan to
halt execution.  Before discussing potential responses to plans
failing in this way, it will obviously be necessary to first discuss
the basics of the planning process.

\section{Cause and Effect}

Two symbols correlated in time are not enough to compose a causal
relationship.  Two symbols correlated in time are simply a coincident
transition from the past to the future.  A causal relationship
supposes an additional component, a necessary connective symbolic
reference to the activities that are ongoing during the transition.
Thus, the effect of the causal component becomes the transition
itself.  A causal relationship, therefore, has three parts: the
symbolized qualities of Duration active in the present, the symbolized
qualities of Duration in the past, and the symbolized qualities of
Duration in the future.  I will sometimes refer to these three parts
of the causal relationship more succinctly as (1) the cause, (2) the
necessity, and (3) the result.  When causal models are used for
planning, the symbolic reference that is the cause is referred to as a
resource.

\section{Goal Activities}

It is important to understand that, in my model, goals are not
derivatives of symbolized perceptions.  Symbolized goals refer to the
fundamental activities in Duration that give \emph{a priori} direction
to the activities of thinking.  Goals can thus be arranged in Spatial
orders, according to preferential qualities; i.e. goals can be
considered to be positive or negative, something to seek or something
to avoid.

Bergson refers to activities in Duration as generally \emph{willful}.
In my model, a goal is simply a reference to these activities that
symbolically exist in Duration.  Qualities of Duration can be
willfully reflective, and thus symbolized as goals, remaining
generalized without becoming anything specific or fundamental in
themselves.

The first-order reflective layer willfully symbolizes goals and this
creation of a symbolic reference to goal activities becomes the reason
for planning and acting toward or away from the associated
perceptions.  Note that because symbolic goals, perceptions, and
resources in my model are artificial and do not derive from one
another, a process of refining symbolic references to perceptions and
resources can be undertaken in the pursuit of creating more accurate
causal models that predict the activities in Duration that symbolic
goals reference.

\section{Causal Hypothesis}

When goal activities are symbolized, causal hypotheses can be created
for predicting the goal activities in the future.  For example, if
there are ongoing activities happening simultaneously with the
symbolization of the goal, these can be hypothesized as causes of a
future symbolized goal.  Remember that a causal model has three parts:
present cause, past, and future.  In this case, the goal is placed in
a future context, the symbolized current activity is in the present,
and there must also be a past symbol to fill the last slot in the
model.

\section{Limitations of Logical Goals}

Logical approaches to AI have seen the value in considering goals to
be relative arrangements of perceptual symbols.  These sorts of
symbolic relationships in my model are thought of as occurring
simultaneously with, before or after goal activities.  Artificial
constructions that are actively maintained in Spatial arrangement can
be reified as symbolic perceptions, but fundamentally the goal does
not refer to an artificial construction, despite the possible
correlation of goals with such constructions.  In this way, my model
gains the ability to think about and refine the symbolization of
perceptions because symbolic goals refer to the actual qualities of
Duration, instead of the inversely limiting assumption that goals are
specified in terms of artificial constructions of perceptual symbols.
Artificial constructions are useful tools for accomplishing goal
activities; however, reflectively, all artificial constructions,
including symbols, are caused by the AI itself and the correspondingly
responsible activities can change.  Thus, all symbolic constructions
are explicitly available as artificial references that are available
for inspection by the AI itself.  In other words, in purely logical
approaches to AI, symbols are not understood to be artificial
creations of the AI itself and, thus, cannot be reflectively debugged
when they are wrong.

In order to allow these logical approaches to adapt, I see no
alternative but to fundamentally reinvent all logical approaches to
problem solving in reflective terms that acknowledge a dynamic reality
and the static artificiality of symbols, allowing the potential for
debugging meaningful and useful symbolic constructions.  AI systems
must use symbols in full awareness of their creation as pliable
artificially static references to a dynamic unstated actual existence.

\section{Physical Goals}

The primary class of goals that can be symbolized is called the
physical goal.  Physical goals refer to dynamic activities in Duration
that are the cause for refinement and initial creation of static
symbolic perceptions and resources.  These initial symbolic
perceptions and resources are thus physical as well.  The primary and
most fundamental class of causal hypothesis is the physical causal
hypothesis, composed of physical perceptions, resources, and goals.
All of this initial knowledge stems from the most fundamental class of
goal, the physical goal.

\section{Reflective Classes of Causal Models}

The first-order reflective layer creates causal models from physical
perceptions, resources, and goals.  I've described previously how
goals can cause reflective thinking to create causal hypotheses.
Another way to state this more succinctly is as follows: \emph{goals
  are the cause of causal hypotheses}.  Note two different meanings of
cause in the previous sentence.  In the latter case, I'm referring to
causal models relating physical perceptions and resources, and in the
former case, I'm referring to the first-order reflective activity that
causes the creation of the physical model.

Allowing activities of the first-order reflection to be available for
inspection by the AI itself allows my model to represent the
transitions caused by first-order reflective activities.  These
transitions are knowledge level changes.  By using this reflective
technique, a new class of causal model is created, categorically
different from the physical causal models that the first-order
reflective layer manipulates.  When the activities of the first-order
reflective layer are reflectively symbolized, my model can then be
motivated to learn to accomplish or avoid knowledge level goals, using
knowledge level causal hypotheses.

\section{Probabilistic Causal Models}

Probabilistic models are an advanced and important thinking tool.
Building a probabilistic causal model involves counting symbolic
perceptions, resources, and goals.  For example, let us consider that
two causal hypotheses have been created that reference the same
symbolic cause and the same symbolic perception in the past; let us
say that the only difference between these two hypotheses is that they
refer to two different symbolic goals in the future.  This situation
gives my model a reason to further distinguish symbolic
representations for perceptions and resources, introducing more
refined symbols for these perceptions and resources in order to lead
to causal models that are more useful for correctly predicting these
two different symbolic goals.  This would be the more appropriate
course of action if we wanted to correctly predict the symbolic goals;
however, there is an opportunity here to build a probabilistic model
that does not refine the symbolization of perceptions or resources.
For example, both of the two causal hypotheses could be considered
together to compose a probabilistic causal hypothesis.  Using this
probabilistic hypothesis, a future inference could be created that
includes both symbolic goals, each with one half of a potential
existence.  Probabilistic causal hypotheses are useful for predicting
the average number of times that a symbolic event will occur.  Note
that probabilistic causal hypotheses require counting and creating
ratios from the more fundamental non-probabilistic causal hypotheses
from which they are constructed.

