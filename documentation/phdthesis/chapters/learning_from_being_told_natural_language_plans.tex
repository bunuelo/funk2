%************************************************
\chapter{Learning from Being Told Natural Language Plans}
\label{chapter:learning_from_being_told_natural_language_plans}
%************************************************

Every planning layer in the SALS cognitive architecture, including the
deliberative, reflective and super-reflective layers, is capable of
learning in two different ways: (1) from ``being told'' and (2) from
``experience.''  Learning from being told occurs in terms of natural
language plans that are programmed into the different layers of the AI
by the user or potentially another AI.
{\mbox{\autoref{figure:learning_from_being_told_in_layers}}} shows the
information pathways in SALS that are involved in learning from being
told as well as learning from experience.
\begin{figure}
\centering
\includegraphics[width=6cm]{gfx/learning_from_being_told_in_layers}
\caption[Learning from ``being told'' and learning from ``experience''
  both occur in each of SALS' planning layers.]{Learning from ``being
  told'' and learning from ``experience'' both occur in each of SALS'
  planning layers.  When a layer of the AI learns from being told, a
  natural language plan is communicated to that layer from a source
  external to the AI, such as a human user.  When a layer of the AI
  learns from experience, two streams of trace events are received
  from the layer below that are asynchronously woven into hypothetical
  causal models of the effects of actions.}
\label{figure:learning_from_being_told_in_layers}
\end{figure}
In this chapter, I will focus on learning from being told natural
language plans.  I will describe the details of how each planning
layer in SALS asynchronously learns from experience in
{\mbox{\autoref{chapter:learning_asynchronously_from_experience}}}.
The important point that I will elaborate upon in this chapter is that
the SALS AI interprets natural language plans, while simultaneously
considering syntax, semantics, current environmental context, learned
hypothetical knowledge about the effects of actions as well as the
current positive and negative goals of the AI.  My approach is as
opposed to linguistic traditions that focus on only one or two of
these aspects of natural language understanding.

Natural language plans in SALS are in the form of a programming
language that has variables, conditionals, recursion, ambiguous
values, an imaginative compiler, and the ability to use analogical
patterns between collections of these plans to interpret natural
language sentences and phrases.  SALS natural language plans are
sequences of commands that can be created, mutated, and executed by a
planning layer in order to accomplish goals.  The following is an
example of a definition of one of the deliberative plans that the AI
in the story could consider executing:
\begin{samepage}
\begin{Verbatim}
  [defplan 'move left'
    [call-below 'move left']]
\end{Verbatim}
\end{samepage}
This expression defines a new deliberative plan.  The ``defplan''
command is shorthand for ``define plan.''  The first argument to the
defplan expression is the name of the plan: ``move left.''  The body
of the plan is the remaining sequence of expressions.  The only
expression in the body of this plan is the ``call-below'' expression
with the ``move left'' argument.  This expression activates a resource
in the layer below, in this case, the ``move left'' resource, which is
in the built-in reactive layer of the AI.  The ``call-below''
expression not only activates a resource in the layer below but also
waits for that resource to complete execution or fail.  The ``move
left'' plan defines a possible natural language interpretation for the
``move left'' phrase, stating that this phrase refers to the
synchronous execution of the ``move left'' resource in the layer
below.

Consider this analogous plan to the ``move left'' plan that defines an
interpretation of the ``move right'' phrase:
\begin{samepage}
\begin{Verbatim}
  [defplan 'move right'
    [call-below 'move right']]
\end{Verbatim}
\end{samepage}
The analogous similarity between the ``move left'' and ``move right''
commands can be abstracted into a new ``move left'' natural language
plan that uses the following syntax:
\begin{samepage}
\begin{Verbatim}
  [defplan 'move left'
    :matches ['move [? direction]']
     :frame [[direction 'left']]
      [call-below 'move [? direction]']]
\end{Verbatim}
\end{samepage}
This generalized form of the original ``move left'' and ``move right''
plans uses a natural language variable, ``direction.''  Note that
there are two optional arguments to the defplan expression in this
example: (1) ``:matches'' and (2) ``:frame.''  The optional
``:matches'' argument specifies a list of potential natural language
patterns that this plan may match as it is being interpreted.  In this
case, the variable expression ``[? direction]'' is allowed to replace
the word ``left'' from the original name of the plan.  The optional
``:frame'' argument specifies the default natural language variable
bindings.  In this case, the ``direction'' variable is assigned the
natural language phrase ``left'' by default.  In the body of the
generalized form of the plan, all occurrences of ``left'' have been
replaced with the variable expression ``[? direction]''.  Given this
generalized form of the original plan, the planner can create a new
analogous plan as an interpretation of either of the natural language
phrases: ``move left'' or ``move right.''

\section{Conditionals and Partial States}

The SALS natural planning language includes conditional branches that
can change the course of plan execution based on the existence of
partial states in the knowledge base that it is trying to control.
For example, here is a more complicated SALS plan that shows a number
of new SALS primitives that will be discussed next:
\begin{samepage}
\begin{Verbatim}
  [defplan 'move toward a cube'
    [if [exists [relationship block property shape cube
	                      preposition left-of
	                      gripper property is-me true]]
        [call-below 'move left']
      [if [exists [relationship block property shape cube
                                preposition right-of
                                gripper property is-me true]]
          [call-below 'move right']
        [fail]]]]
\end{Verbatim}
\end{samepage}
This plan checks to see if there is a cube to the left of the gripper
that the AI is controlling.  If there is a cube to the left, this plan
will activate the ``move left'' resource in the layer below.  If there
is not a cube to the left, this plan then checks to see if there is a
cube to the right of the gripper that the AI is controlling.  If there
is a cube to the right, this plan will activate the ``move right''
resource in the layer below.  At this point, if there is not a cube to
the left or to the right, the plan fails.  There are a number of new
primitives that are introduced in this example of conditional
branches:
\begin{packed_itemize}
\item{``if''}
\item{``exists''}
\item{``relationship''}
\item{``fail''}
\end{packed_itemize}
The syntax for the SALS ``if'' expression is similar to the ``if''
expression in most lisp-like languages: the first argument is the
conditional, the second argument is the true branch, and the remaining
arguments are the optional body of the false branch.  Unlike most
lisp-like languages, the SALS ``if'' expression's conditional value
must be a Boolean type object and will fail with any other value.  The
``fail'' expression is a simple way for a plan to stop executing and
mark the plan with the knowledge of a failure object.  The ``exists''
expression accepts a partial state as its only argument and checks to
see if this partial state exists in the knowledge base that the
planning layer is trying to control.  When the effects of a plan are
being imagined, the return value of the ``exists'' expression are not
known, so multiple possible ambiguous values are returned: (1) the
result based on current hypothetical models of the effects of previous
actions or the current state of the knowledge base that this planning
layer is trying to control if no previous actions have been imagined,
(2) a true value based on the possibility that learned models are
incorrect, and (3) a false value based on the possibility that learned
models are incorrect.  The ``relationship'' expression is one of two
special expressions in SALS that return partial state objects.  The
``relationship'' expression accepts ten arguments, which map directly
to the internal semantic graph representation of the knowledge base
that the planning layer is trying to control.  The following are the
ten arguments to the ``relationship'' expression:
\begin{packed_enumerate}
\item{source-type}
\item{source-key-type}
\item{source-key}
\item{source-value}
\item{key-type}
\item{key}
\item{target-type}
\item{target-key-type}
\item{target-key}
\item{target-value}
\end{packed_enumerate}
{\mbox{\autoref{figure:relationship_partial_state_graph}}} shows how
the arguments to the ``relationship'' expression map to the frame
types, slot values, and properties of a frame-based knowledge base
that is represented as a graph.  When an argument to the
``relationship'' expression is a symbol, this symbol is checked
against a list of known symbols within the SALS AI.  If an argument to
the ``relationship'' expression is not a known symbol, this results in
an interpretation failure, limiting the number of possible
interpretations.
\begin{figure}
\centering
\includegraphics[width=12cm]{gfx/relationship_partial_state_graph}
\caption[The SALS ``relationship'' expression returns a partial state
  object.]{The SALS ``relationship'' expression returns a partial
  state object.  The graph on the left shows the ten argument names
  for the ``relationship'' expression, while the graph on the right
  shows a potential partial state of the physical knowledge base that
  literally means, ``a cube shaped block to be to the left of a
  gripper that is me.''}
\label{figure:relationship_partial_state_graph}
\end{figure}

Now, let us consider a slightly different type of partial state
expression in the following example plan that attempts to control the
gripper to grab a block:
\begin{samepage}
\begin{Verbatim}
  [defplan 'attempt to grab block'
    [call-below 'grab']
    [wait-for [property gripper property is-me true
                        property movement-command
                        stop]]]
\end{Verbatim}
\end{samepage}
In this plan, two new types of SALS expressions are introduced:
\begin{packed_itemize}
\item{``wait-for''}
\item{``property''}
\end{packed_itemize}
The ``wait-for'' expression takes one argument, which similarly to the
``exists'' expression, must be a partial state object, such as that
returned by the ``relationship'' expression.  Functionally, the
``wait-for'' expression puts the plan to sleep until the specified
partial state exists in the knowledge base in the layer below that
this plan is trying to control.  The ``property'' expression is
similar to the ``relationship'' expression in that it does return a
partial state object, but the ``property'' expression only takes the
following seven arguments:
\begin{packed_enumerate}
\item{source-type}
\item{source-key-type}
\item{source-key}
\item{source-value}
\item{key-type}
\item{key}
\item{value}
\end{packed_enumerate}
{\mbox{\autoref{figure:relationship_partial_state_graph}}} shows how
the arguments to the ``property'' expression map to the frame types,
slot values, and properties of a frame-based knowledge base that is
represented as a graph.
\begin{figure}
\centering
\includegraphics[width=12cm]{gfx/property_partial_state_graph}
\caption[The SALS ``property'' expression returns a partial state
  object.]{The SALS ``property'' expression returns a partial state
  object.  The graph on the right shows the seven argument names for
  the ``property'' expression, while the graph on the left shows a
  potential partial state of the physical knowledge base that
  literally means, ``a gripper to be me and have a stop movement
  command.''}
\label{figure:property_partial_state_graph}
\end{figure}

While the deliberative layer may create plans that refer to partial
states in the physical knowledge base, the reflective layer may create
plans that refer to partial states in the deliberative plan knowledge
base, which may in turn also refer to partial states in the physical
knowledge base.  Consider the following example of a reflective plan
that includes this form of hierarchical partial state reference:
\begin{samepage}
\begin{Verbatim}
  [defplan 'a deliberative planner to have a positive goal
            for a cube to be on a pyramid'
    [property planner property layer deliberative
              property positive-goal
              [relationship block property shape cube
                            preposition on
                            block property shape pyramid]]]
\end{Verbatim}
\end{samepage}
In this reflective plan, the ``property'' expression describes a
partial state of the deliberative plan knowledge base, while the
``relationship'' expression describes a partial state of the physical
knowledge base.  In order for SALS to convert this expression to a
purely deliberative form of knowledge, hierarchical partial states are
reified in SALS so that they become a simple partial state that is
purely of one layer's type of knowledge.  When a partial state object
is passed as an argument to another partial state object, the first
partial state is converted to a symbolic form, so that the entire
structure can continue to exist as a simple frame-based graph
structure.  {\mbox{\autoref{figure:hierarchical_partial_state_graph}}}
shows an example of a hierarchical embedding of ``relationship'' and
``property'' partial states that may occur in any planning layer above
the deliberative layer.
\begin{figure}
\centering
\includegraphics[width=12cm]{gfx/hierarchical_partial_state_graph}
\caption[The SALS ``relationship'' and ``property'' expressions can be
  hierarchically combined in planning layers above the
  deliberative.]{The SALS ``relationship'' and ``property''
  expressions can be hierarchically combined in planning layers above
  the deliberative.  Note that any partial states that are
  sub-expressions of other partial states become symbolically reified
  in order to maintain a frame-based graph structure for all
  knowledge.}
\label{figure:hierarchical_partial_state_graph}
\end{figure}

\section{Natural Language Plans Interpreting Natural Language Plans}

The most powerful capability of the SALS natural language programming
language is the ability to find correct interpretations of ambiguous
natural language plans.  Let us first define the following simple
natural language plan that returns a ``relationship'' partial state
object:
\begin{samepage}
\begin{Verbatim}
  [defplan 'a cube to be to my left'
    [relationship block property shape cube
                  preposition left-of
                  gripper property is-me true]]
\end{Verbatim}
\end{samepage}
This plan can be generalized to analogously work for any type of shape
as in the following example:
\begin{samepage}
\begin{Verbatim}
  [defplan 'a cube to be to my left'
    :matches ['a [? shape] to be to my left']
     :frame [[shape 'cube']]
      [relationship block property shape [? shape]
                    preposition left-of
                    gripper property is-me true]]
\end{Verbatim}
\end{samepage}
Now, consider the following plan that makes use of this previous plan
definition and introduces two new SALS expression types for evaluating
natural language plans:
\begin{samepage}
\begin{Verbatim}
  [defplan 'a cube is to my left'
    [exists [plan-call [plan 'a cube to be to my left']]]]
\end{Verbatim}
\end{samepage}
This last plan returns a true or false value depending on whether or
not the partial state returned by the plan, ``a cube to be to my
left,'' exists in the knowledge base that the planning layer is trying
to control.  Two new types of SALS natural language programming
expressions are introduced in the last plan:
\begin{packed_enumerate}
\item{``plan-call''}
\item{``plan''}
\end{packed_enumerate}
The ``plan'' expression takes one argument, a natural language phrase.
The ``plan'' expression returns a plan from the current planning layer
that matches the given natural language phrase.  If there is no
matching natural language plan, SALS will attempt to find an plan that
has an analogous match to the given natural language phrase.  In most
cases, there are multiple possible matches for any given natural
language phrase.  In these cases, the SALS natural language plan
compiler is responsible for imagining the effects of different
interpretations on the knowledge base that the planning layer is
trying to control, while avoiding natural language plan interpretation
failures.  The ``plan-call'' expression accepts one argument, a
natural language subplan to compile into this location of the current
plan that is being defined by the ``defplan'' expression.

Now, consider the following redefinition of the previous ``move toward
a cube'' plan that I defined previously:
\begin{samepage}
\begin{Verbatim}
  [defplan 'move toward a cube'
    [if [plan-call [plan 'a cube is to my left']]
        [call-below 'move left']
      [if [plan-call [plan 'a cube is to my right']]
          [call-below 'move right']
        [fail]]]]
\end{Verbatim}
\end{samepage}
This version of the ``move toward a cube'' natural language plan is
simpler because it only indirectly references the ``relationship'' and
``exists'' expressions through the ``plan'' and ``plan-call''
expressions that refer to the appropriate analogies to other natural
language plans.  Now, consider the following expression that defines
an analogy for any natural language plans that use the ``if''
expression:
\begin{samepage}
\begin{Verbatim}
  [defplan 'if a cube is to my left, move left, otherwise move right'
    :matches ['if [? condition], [? true-branch], otherwise [? false-branch]']
     :frame [[condition    'a cube is to my left']
             [true-branch  'move left']
             [false-branch 'move right']]
      [if [plan-call [plan [? condition]]]
          [plan-call [plan [? true-branch]]]
        [plan-call [plan [? false-branch]]]]]
\end{Verbatim}
\end{samepage}
Using this definition of an analogy for the ``if'' expression, the
original ``move toward a cube'' natural language plan can be rewritten
as follows:
\begin{samepage}
\begin{Verbatim}
  [defplan 'move toward a cube'
    [plan-call [plan 'if a cube is to my left, move
                      left, otherwise if a cube is
                      to my right, move right,
                      otherwise fail']]]
\end{Verbatim}
\end{samepage}
Note that this last plan uses two ``if'' statements, the second is in
the false branch of the first.

Before getting to the details of how ambiguity in searched through and
eliminated in the SALS natural language plan compiler, consider the
following definitions that include the SALS ``not'' expression:
\begin{samepage}
\begin{Verbatim}
  [defplan 'a cube is not to my left'
    [not [plan-call [plan 'a cube is to my left']]]]
\end{Verbatim}
\end{samepage}
The SALS ``not'' expression is similar to the ``not'' expression in
most lisp-like programming languages in that it takes one argument.
Unlike most lisp-like languages, the SALS ``not'' expression only
accepts a Boolean type of object.  The ``not'' expression returns a
new Boolean type object that represents the opposite value of the
argument.  The following is an analogous plan that can be used to
generalize this natural language usage of the ``not'' expression:
\begin{samepage}
\begin{Verbatim}
  [defplan 'a cube is not to my left'
    :matches ['[? subject] is not [? preposition]']
     :frame [[subject     'a cube']
             [preposition 'to my left']]
      [not [plan-call [plan '[? subject] is [? preposition]']]]]
\end{Verbatim}
\end{samepage}
This plan allows many negative expressions to analogously have a
correct interpretation, such as ``if a pyramid is not to my right,
move left, otherwise fail.''

Another powerful component of the SALS natural programming language is
the ability to compile natural language plans that include recursive
references, which enable plans to describe looping functionality.
SALS also has a primitive capability to imagine the possible effects
of loops by imaginatively unrolling the loop only once.  The following
example is a definition of a reflective natural language plan that
searches for a deliberative plan whose effects have not yet been
imagined:
\begin{samepage}
\begin{Verbatim}
  [defplan 'find next unimagined plan'
    [call-below 'focus on next object']
    [plan-call [plan 'if a planner is focusing on
                      a plan that has been imagined,
                      find next unimagined plan']]]
\end{Verbatim}
\end{samepage}
Notice that the name of this plan is ``find next unimagined plan,''
which is the same as the true branch of the natural language ``if''
statement in the body of the plan.  This plan checks to see if the
plan currently in the focus of the deliberative planner has been
imagined.  If the plan in deliberative focus has been imagined, this
plan calls itself recursively until the deliberative planner is
focusing on a plan that has not been imagined.

As a final example of a natural language plan interpreting a natural
language plan, consider again the following hierarchical partial state
construction from a reflective natural language plan from earlier in
this chapter:
\begin{samepage}
\begin{Verbatim}
  [defplan 'a deliberative planner to have a positive goal
            for a cube to be on a pyramid'
    [property planner property layer deliberative
              property positive-goal
              [relationship block property shape cube
                            preposition on
                            block property shape pyramid]]]
\end{Verbatim}
\end{samepage}
This reflective natural language plan can be abstracted with ``plan''
and ``plan-call'' expressions as in the following example:
\begin{samepage}
\begin{Verbatim}
  [defplan 'a deliberative planner to have a positive goal
            for a cube to be on a pyramid'
    :matches ['a deliberative planner to have a positive
               goal for [? partial-state]']
     :frame [[partial-state 'a cube to be on a pyramid']]
      [property planner property layer deliberative
                property positive-goal
                [plan-call [plan [? partial-state]]]]]
\end{Verbatim}
\end{samepage}
In this way, analogous reflective plans can be created in order to
allow the interpretation of any natural language physical partial
state being a positive goal of the deliberative planning layer.  A
similar technique can be used to create analogous reflective natural
language plans that work for negative goals as well as plans that are
hypothesized to cause partial states to exist.

\section{Analogous Natural Language Plan Interpretation}

As previously described, the ``plan'' expression in a SALS natural
language plan returns a plan object that either matches the name of a
plan previously defined via the ``defplan'' expression, or if an
analogy can be found to an existing plan, a new analogous plan object
is created and returned as the result of the ``plan'' expression.
Because of the possibility that multiple plans may match a given
``plan'' expression, it is the task of the SALS compiler to imagine
the effects of the different interpretations and decide upon one for
execution.  The SALS natural language plan compiler must handle
multiple ambiguous return values for each ``plan'' expression.  Let us
consider again the following natural language plan that must be
imagined and interpreted, which requires the compiler to sort through
multiple possible interpretations:
\begin{Verbatim}
  [plan-call [plan 'if a cube is not on a pyramid, stack a
                    cube on a pyramid']]
\end{Verbatim}
A reasonable way to expect this natural language phrase to
be interpreted is as a plan analogous to the natural language plan for
the ``if'' expression similar to the one previously defined, as in the
following:
\begin{samepage}
\begin{Verbatim}
  [defplan 'if a cube is not on a pyramid, stack a cube on a
            pyramid'
    :matches ['if [? condition], [? true-branch]']
     :frame [[condition    'a cube is not on a pyramid']
             [true-branch  'stack a cube on a pyramid']]
      [if [plan-call [plan [? condition]]]
          [plan-call [plan [? true-branch]]]]]
\end{Verbatim}
\end{samepage}
Although this plan makes sense, there are many other possible
problematic analogies to previously defined natural language plans
that do not make any sense at all.  The following problematic
interpretation is one example:
\begin{samepage}
\begin{Verbatim}
  [defplan 'if a cube is not on a pyramid, stack a cube on a
            pyramid'
    :matches ['[? subject] is not [? preposition]']
     :frame [[subject     'if a cube']
             [preposition 'on a pyramid, stack a cube on a
                           pyramid']]
      [not [plan-call [plan '[? subject] is
                             [? preposition]']]]]
\end{Verbatim}
\end{samepage}
Notice that the natural language value of the ``subject'' variable in
the previous problematic interpretation is equal to ``if a cube.''
The following is the result of one step in compiling this problematic
interpretation:
\begin{Verbatim}
  [not [plan-call [plan 'if a cube is on a pyramid, stack a
                         cube on a pyramid']]]
\end{Verbatim}
Notice that the ``not'' expression has been moved to the
front of this expression after it has been partially compiled.  The
following is the result of another couple steps of further
interpreting the remaining natural language in this expression:
\begin{Verbatim}
  [not [if [exists [plan-call [plan 'a cube to be on a
                                     pyramid']]]
           [plan-call [plan 'stack a cube on a pyramid']]]]
\end{Verbatim}
When this problematic interpretation is imagined, the result from the
``exists'' expression is a Boolean value, so this satisfies the ``if''
conditional type requirement.  However, the result of the ``if''
expression is the special type ``nil,'' which is not a Boolean value
in the SALS natural programming language.  Because the ``not''
expression strictly requires a Boolean type value, the imaginative
interpretation of this plan fails when the nil value from the ``if''
statement reaches the ``not'' expression.  Using strict typing in the
low-level details of the SALS natural programming language allows
ambiguous high-level expressions with many possible interpretations to
be narrowed down to a few that make programmatic sense.  By
introducing more constraints to the low-level details of the SALS
natural programming language, many types of plan failures can be
imagined and avoided, even while using this very loose type of natural
language analogical matching technique.  Not all errors can be
imagined and avoided through imaginative compiling, but many types of
failures are avoided in this way.  However, some failures, such as
expectation failures, can only be realized during the actual execution
of the plan.

\section{Imagining the Effects of Ambiguous Natural Language Plans}

As natural language plans are interpreted, the hypothetical effects of
any resource activations in the layer below are also simultaneously
imagined in the counterfactual knowledge base of that planning layer.
Imagining the effects of resource activations first requires that
hypothetical models of the effects of resource activations exist.
Hypothetical models of the effects of resource activations in SALS
provide a rule-based mapping from the preconditions of the resource
activation to the potential transframes for the resource activation.
The preconditions and transframes are in terms of the abstract partial
states that have previously existed in the knowledge base that the
planning layer is trying to control.  For example, ``a gripper that is
me being above a cube shaped block'' could be one of many
preconditions for an action.  All partial states that can be returned
by the ``relationship'' and ``property'' expressions in the SALS
natural programming language are efficiently abstracted asynchronously
from the knowledge base that the planning layer is trying to control.
I will describe the details of the abstract asynchronous learning
algorithm for each planning layer in
{\mbox{\autoref{chapter:learning_asynchronously_from_experience}}}.
For now, know that abstract hypothetical models of resource
activations are learned and can be used for imagining the effects of
resource activations during the natural language plan interpretation
process.

Because some expressions in the SALS natural planning language can
return multiple possible ambiguous values, one task of the planning
layer is to decide which of the multiple possible interpretations is
complete, accomplishes positive goals and avoids negative goals.  This
means that each expression in the SALS natural programming language
may have one or more possible different outcomes, depending on the
interpretation path and which one of potentially multiple ambiguous
values is chosen to be executed when a decision must be made.  In
order to keep track of each of these different interpretations, each
plan expression is allocated an ``execution node'' object and each
decision among multiple ambiguous argument values is allocated an
``argument decision node'' object in the plan knowledge base of that
planning layer.  The ``execution node'' objects correspond to the
functional hierarchy of the imagined natural language plan execution,
while the ``argument decision node'' objects represent any points in
the imagined execution where two potentially different sub-trees of
the functional execution could occur based on different choices
between multiple possible ambiguous return values from sub-expressions
of the current expression being imaginatively executed.
\begin{figure}
\hspace*{-1cm}\includegraphics[width=14cm]{gfx/plan_execution_node_graph}
\caption[A graph representation of the deliberative plan knowledge
  base while a simple plan with multiple ambiguous interpretations is
  being imaginatively interpreted and evaluated.]{A graph
  representation of the deliberative plan knowledge base while a
  simple plan with multiple ambiguous interpretations is being
  imaginatively interpreted and evaluated.  The deliberative
  ``planner'' object is focusing on a ``plan'' object that has been
  partially evaluated.  The first ``execution-node'' object of this
  ``plan'' object represents the partial interpretation of a ``not''
  expression that has a ``plan-call'' sub-expression.  The ``plan''
  expression returns two possible values, which are interpreted
  separately under ``argument-decision-node'' objects.}
\label{figure:plan_execution_node_graph}
\end{figure}
{\mbox{\autoref{figure:plan_execution_node_graph}}} shows a graph
representation of the frame-based deliberative plan knowledge base as
it is in the process of imaginatively evaluating the following simple
plan:
\begin{samepage}
\begin{Verbatim}
  [defplan 'a cube is not on a pyramid'
    :matches ['[? subject] is not [? preposition]']
     :frame [[subject     'a cube']
             [preposition 'on a pyramid']]
      [not [plan-call [plan '[? subject] is
                             [? preposition]']]]]
\end{Verbatim}
\end{samepage}
When this natural language plan is imaginatively interpreted, there
are multiple possible values returned from the ``plan'' expression,
which returns analogous plans that match the natural language phrase,
``a cube is on a pyramid.''  In this case, there are the two following
plans that are created as analogies to other known plans:
\begin{packed_enumerate}
\item{
\begin{samepage}
\begin{Verbatim}
  [defplan 'a pyramid is on a cube'
    :matches ['a [? top-shape] is on a [? bottom-shape]']
     :frame [[top-shape    'pyramid']
             [bottom-shape 'cube']]
      [exists [relationship block property shape [? top-shape]
                            preposition on
                            block property shape [? bottom-shape]]]]
\end{Verbatim}
\end{samepage}
}
\item{
\begin{samepage}
\begin{Verbatim}
  [defplan 'a pyramid is on a cube'
    :matches ['a [? top-color] is on a [? bottom-color]']
     :frame [[top-color    'pyramid']
             [bottom-color 'cube']]
      [exists [relationship block property color [? top-color]
                            preposition on
                            block property color [? bottom-color]]]]
\end{Verbatim}
\end{samepage}
}
\end{packed_enumerate}
The first of these two analogous interpretations is what one would
expect: the natural language phrases ``pyramid'' and ``cube'' are
interpreted to be shapes of blocks that are on top of one another.
The second of these interpretations is less obvious and incorrectly
interprets ``pyramid'' and ``cube'' to be colors.  This problematic
interpretation is an analogy to the following plan that the SALS AI
already knows:
\begin{samepage}
\begin{Verbatim}
  [defplan 'a red is on a blue'
    :matches ['a [? top-color] is on a [? bottom-color]']
     :frame [[top-color    'red']
             [bottom-color 'blue']]
      [exists [relationship block property color [? top-color]
                            preposition on
                            block property color [? bottom-color]]]]
\end{Verbatim}
\end{samepage}
The SALS AI does not immediately know that ``red'' and ``blue'' are
colors, while ``cube'' and ``pyramid'' are shapes.  While types of
partial states could be programmed into the SALS AI so that these
specific words could be assigned specific symbolic types, this is not
the approach taken in the SALS AI.  Instead, both of these
interpretations are plausible in the SALS AI.  However, when the SALS
AI does finish imagining all possible interpretations of a natural
language phrase, each different resulting analogous plan has a set of
associated hypothetical states that this plan may or may not
accomplish.  If the possible effects of a plan include a ``pyramid''
color, which does not make sense, the SALS AI sees that this is not
one of its goals, so it ignores this interpretation for that reason
alone.  The SALS AI is a goal-oriented natural language understanding
system in this sense---finding those natural language plan
interpretations that it thinks will accomplish its positive goals.  On
the other hand, the SALS AI considers its negative goals in the
opposite sense: when a natural language plan interpretation is
hypothesized to accomplish a negative goal, that plan interpretation
is ignored.  The SALS AI can be considered to be an ``optimistic'' or
``pessimistic'' natural language understanding system in these cases.
The important point here is that the SALS AI interprets natural
language plans, while simultaneously considering syntax, semantics,
current environmental context, learned hypothetical knowledge about
the effects of actions as well as the current positive and negative
goals of the AI.

There are two ways that a natural language plan is hypothesized to
cause a specific partial state: (1) learned hypothetical models are
used to predict the effects of actions, and (2) existence checks for
partial states during the imaginative interpretation of the plan are
used as secondary evidence that a plan may or may not be expecting a
specific partial state to exist during its execution.  For example, if
a natural language plan checks for the existence of the partial state,
``a cube shaped block to be on top of a pyramid shaped block,'' this
existence check provides secondary evidence that this plan could be
expecting this state to exist during its execution.  This secondary
evidence of the possible intentions of a plan is an example of
knowledge learned during the imaginative interpretation of natural
language plans in the SALS AI.

