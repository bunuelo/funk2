%************************************************
\chapter{Learning from Success and Failure}
\label{chapter:learning_from_success_and_failure}
%************************************************

\section{Plan Execution}

First-order reflective thinking creates causal hypotheses that
describe how physical resources may cause transitions between physical
perceptions and goals.  Additional first-order reflective thinking
activities combine these causal hypotheses into plans that are
supposed to accomplish or avoid goals, given their preferential
arrangement.  After a plan is created, its sequential instructions for
activating or suppressing the resources contained in the causal slots
of the hypotheses can be traversed and the causal resource that has
been referenced can be put into the appropriate activated or
suppressed Spatial arrangement.  I will refer to this sequential
process of traversing through a plan and following its activation or
suppression instructions as \emph{execution}.

\section{Plan Instructions}

I've previously discussed the necessity of using activate and suppress
instructions that relate resources.  One instruction that has only
been implicitly introduced at this point in my description is the
model's ability to Spatially arrange a sequence of instructions, which
I will refer to as the \emph{program} instruction.  There is also an
instruction that allows the Spatial arrangement of concurrent
instructions in the model, which I will refer to as the \emph{parallel
  program} instruction.  The program and parallel program instructions
allow for the arrangement of hierarchies of planned sequential and
parallel program executions.

I will give a more detailed description of plan instructions in
\autoref{part:the_implementation}, but for now I will focus on
describing how failures can lead not only to learning better
first-order causal hypotheses for predicting physical goals from
physical resources and perceptions but also to learning better
second-order causal hypotheses for predicting first-order goals, such
as the creation of a plan that will not fail when it is executed.

\section{Failures and Success}

There are many types of failures that can occur during the execution
of plans.  Each different type of failure that occurs may have
different types of activities that, reflectively, will appear as the
causal result of the failure; actively halting the plan execution
allows for a debugging response activity to occur at the layer above
the failed plan.  I will refer to the execution of a plan that results
in no failures as a \emph{successful execution}.  Plans may be
considered to exhibit various types of success only in the sense that
they have avoided the various types of execution failures.

\section{Activation Failure}

If we imagine that two plans are executing concurrently, I will refer
to the simultaneous activation and suppression of a resource as an
\emph{activation failure}.  An activation failure can also occur if
two plans attempt to simultaneously activate the same resource.  When
an \emph{activation failure} occurs, one possible response is to
create a new plan to accomplish the combined goals of the two plans,
but which includes the two plans as executing in sequential order.

\section{Compiling Plans to Resources}

I've discussed previously how the instructions in plans can be
sequentially interpreted as one form of execution.  When a plan is
found to fail when executed concurrently with other plans, it is
sometimes helpful to consider the execution of a specific plan to be a
resource in itself.  I will refer to the process of turning the
execution of a specific plan into a new resource as \emph{compiling}.
With the ability to compile plans to resources, we have another
possible solution to activation failures.  For example, if one of the
conflicting plans is shorter or less critical than the other, the
shorter plan can be compiled into a resource that is inhibited
whenever the longer plan is executing.

\section{Expectation Failure}

As a plan executes, the results of resource activations can be found
to not match currently symbolized perceptions.  This failure of a
causal hypothesis to correctly predict the transitional results of a
resource activation I will refer to as an \emph{expectation failure}.
The most obvious response to an expectation failure is to correct the
causal hypotheses in the first-order reflective layer that predict
physical perceptions.  The correction of physical causal hypotheses
involves both eliminating the causal hypotheses that made the
incorrect prediction and creating new causal hypotheses that correctly
predict in light of the new transition between physical perceptions.

\section{Learning Below and Above the Failure}

Correcting the physical causal hypotheses that lead to expectation
failures will ensure that the same type of failure will never be
planned again.  This is useful.  Nevertheless, given reflective
layers, the implementation can still be improved in at least two ways.
Assuming that the causal hypotheses in the first-order reflective
layer have failed, we can also consider corrections that can be made
both ``below'' and ``above'' the failure.

For example, the causal hypotheses in the first-order reflective layer
are based on symbols that refer to activities in the physical layer.
Since, as already discussed, there necessarily are no symbols in the
physical layer, the only allowed descriptions of physical activities
must be created by first-order symbolization activities.  Thus, one
way to create more refined causal hypotheses that describe physical
activities is to create more refined symbols that refer these physical
activities.  This parenthetical ability to refine symbolic references
of an otherwise undefined activity is a critical feature of a
reflective model of mind.  Despite the fact that the refinement of
symbolic references occurs in the same layer as the failed causal
hypothesis, I will refer to this form of learning as being
\emph{below} the failure because it changes references to the layer
below the failure.

Alternatively, we can focus on learning in the layers above the
failure.  The second-order reflective layer can think about the
planning activities in the first-order reflective layer that created
the plan that failed.  Causal models of planning activities that lead
to successful or failed plans can be used by the second-order
reflective thinking layer to better plan sequences of planning
activities.  The second-order reflective layer considers first-order
activities as resources that create plans that, when executed, result
in either successfully or unsuccessfully accomplishing physical goals.
I will refer to correcting causal hypotheses in the layers above the
failure as learning \emph{above} the failure.

Learning above the failure is logistically difficult because of the
arbitrary amount of time that may have passed between the creative
activities that result in plans and the sometimes much later execution
of the plan that results in failure.  This thesis and its
implementation are exclusively focused on this type of learning.  In
other words, the failure becomes above the failure in response to an
expectation failure.

\section{First-order Reflective Thinking Activities}

I've previously described physical goals that are symbolic references
to physical activities that can be preferentially ordered.  The
first-order reflective layer activities create causal hypotheses and
put these together into plans.  This leads me to a critical point: I
have been purposefully vague in describing the types of plans that are
created by the first-order reflective layer.  In my model of mind, I
have statically divided the ongoing activities in Duration into
layers.  Despite my division, which allows talking about orders of
reflective thinking that build layers of abstractions upon other
abstractions, in the dynamic, all of these symbols or Spatial
arrangements only exist actively, not specifically prioritized
statically as discrete and separate components of action.  Thus, all
layers of activity in my model are given the appearance of being
\emph{in~media~res} as ongoing, inseparable, and simultaneously active
activities in Duration.  So, while physical goals may be symbolized by
the first-order thinking activities, the first-order thinking
activities are also making plans.  Thus, the types of plans that this
activity creates are undefined by priority in this sense.
Reflectively, from the perspective of second-order reflective
thinking, we may see clear causal hypotheses that give a reasonable
understanding of the first-order reflective layer activities, but
these activities are seen as ``causal'' only in retrospect and not in
execution.  Thus the idea of causal can be understood as introducing a
separate and reflective point of view to the idea of execution.  Thus,
the results of planning activities in the first-order reflective
thinking layer can be reflectively learned by the second-order
reflective thinking layer.

\section{First-order Reflective Thinking Goals}

The second-order reflective layer learns causal hypotheses of the
given first-order reflective layer activities.  For example, the
second-order reflective layer may symbolize the existence of a plan
that later results in successful execution as a positive first-order
reflective thinking goal.  Similarly, the second-order reflective
layer may symbolize the existence of a plan that later results in
failure as a negative first-order reflective thinking goal.  It is
important to understand that first-order thinking activities are not
limited to only making a specific type of plan, such as a plan that is
expected to result in a successful execution.

To keep the modelling process from becoming ungrounded and circular,
it is necessary for it to allow first-order reflective thinking to
remain undefined as pre-reflective activities in Duration.  The
utility in this type of thinking appears necessarily because it allows
for the creation of some worst possible plan so that other plans may
be compared and made as different as possible from this worst case
scenario.  There are many creative and potentially useful planning
techniques that have been described in the AI literature.  Many of
these techniques are based on analogical abstractions that have
reasonable explanations of utility only in retrospect by reflectively
creating a causal hypothesis in which the result is the symbolic goal
of the existence of a successful plan.

My reflective model does not limit first-order planning activities to
being subsequently recognized by the second-order reflective layer as
one symbolic type or another.  This emphasizes the ability of the
second-order reflective layer to create causal hypotheses and create
first-order reflective thinking goals that inform the second-order
creation of plans for first-order thinking resources to create more
successful plans for accomplishing symbolized physical goals.

