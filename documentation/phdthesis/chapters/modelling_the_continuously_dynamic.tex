%************************************************
\chapter{Modelling the Continuously Dynamic}
\label{chapter:modelling_the_continuously_dynamic}
%************************************************

\vspace{5mm}
\begin{center}
\emph{You can't find love in a dictionary.} \\ \medskip
    --- \cite{dennett:2009}
\end{center}
\vspace{5mm}

In this dissertation the focus will be a description of thinking that
reflectively learns to accomplish goals.  \autoref{part:the_model}
begins with a clear distinction between dynamic continuity and static
arrangements of symbols, which are \emph{the} fundamental key
distinctions in my model.  The model of reflective thinking grows
quickly within this initial foundational conception, and leads
necessarily and naturally to the goals of this dissertation.

In \autoref{part:the_simulation}, the simulation, I will describe an
example of a mathematical description of the model, which I will use
to evaluate the model in terms of the performance of a search
algorithm implemented in a reflective simulation.

In \autoref{part:the_implementation}, the implementation, an example
of a computational implementation of reflective thinking will be
described.  The implementation is a layered cognitive architecture
with three working layers.  I will focus on a very simple physical
block stacking problem domain as a tool for explaining simply, through
examples, the layered classes of reflective learning.  In describing
the working model implementation, I will maintain an awareness of the
explicit distinction between a static model and the dynamic referent.

\section{Symbols and Continuity}

Symbols are static referential tools of thought.  Words are examples
of symbols.  Mathematical expressions are composed of arrangements of
symbols.  All written language and spoken language is composed of
sequences of symbols.  While various fonts, writing styles, and
speaking accents do influence the appearance of symbols, the fact
remains that without symbols there is no language.  A symbol is
discrete and distinctly separate from other symbols.  Symbols do not
blend into one another when they are put side by side.  Using symbols,
models can be created.  Therefore, the crux of my AI modelling problem
is in understanding the continuous activity of reflective thinking.  I
have found understanding the reflective focus to be centered around
understanding how to answer the following two questions:
\begin{enumerate}
\item What is continuous space?
\item What is the continuous activity that fills this space?
\end{enumerate}
Any written or spoken answer to any question must be composed of
discrete symbols.  Some questions have clearly discrete and satisfying
answers.  For example, when asked, ``What is $1+1$?'', a satisfying
answer is clearly ``$2$''.  In this case, the question asks for
another representation for the already symbolic focus of the question,
``$1+1$''.  When we ask, ``What is continuous space?'', this is a
question that refers to something that is not a discrete symbolic
representation.  Because any answer must be specified in discrete
symbols, the answer can never be equivalent to the focus of the
question.  The focus of the question is not the words, ``continuous
space'', but instead, the question is about the referent of these
words.  There is no exact discrete symbolic answer that is equivalent
to the focus of the question.  Answers, therefore, to a question about
continuous activity are necessarily approximated with discrete
symbolic language.

Answers to these questions abound, depending on the goals of the
responder.  My goal is to build an AI model that makes arbitrarily
greater improvements given \emph{any} non-reflective goal-oriented
learning algorithm.  In other words, my AI must be able to abstract
and use models of its own continuously dynamic reasoning activities.
The goal hinges on the AI's ability to deal with the inherent
contradiction in answering these two basic and simple questions for
itself.  If it is understood that any answer must be based on
language, then it becomes obvious that there is no general or final
answer that can be predetermined for as yet unknown goals.  Many
philosophers, scientists, and spiritualists have provided different
answers to these questions in order to accomplish different goals.
Here are some caricatured answers to these two questions that
apparently have very different explanatory goals:
\begin{itemize}
\item {\bf{Physicist:}} Continuous time can be approximated by an
  infinite number of symbols that are arbitrarily close together.  For
  example, if we have the integers $0$ and $1$, then we know that
  between these symbols, rationally, we can define the symbol $1/2$
  that exists between them.  We know that if we are describing
  continuity that there must be a discrete point between any two given
  discrete points in our model.  We use many dimensions of these
  symbolic numbers, in order to define laws describing the dynamic
  continuous potential and kinetic energies in multiple continuous
  spaces and times.
\item {\bf{Spiritualist:}} There is a continuous indivisible flux of
  energy that is the being of everything and nothing together as one.
  All beings are part of this energy that flows among and between us
  without any separations actually existing.  We call this God, the
  energy that is in each and every one of us that contains all time
  and space, life and death, in one experiential existence.
\item {\bf{Philosopher:}} The continuous amorphous perceptual
  experience is the real dynamic ongoing activity of being in the
  present, from which all discrete perceptions are abstracted.  From
  these artificial, man-made symbolic abstractions, all models of
  discrete activities in the discrete dimensions of time and space are
  constructed.
\end{itemize}
Each of these caricatures, presents a very different model of the
continuously dynamic activity in continuous space.  The key idea to
understand here is that each of these models \emph{describes but is
  not equivalent to} this common continuous space and the dynamic
activities that exist within it.  \cite{magritte:1919} visually
explains this point with ``The Treason of Images'' reproduced in
\autoref{figure:magritte_pipe}.  The different descriptions that are
used are good for different goals, but they are all similar in that
they are all based on symbols.  They are all necessarily discrete
descriptions of something that is not fundamentally discrete.  The
physicist uses an infinite number of symbols to describe this
continuity.  A model that includes an infinite number of symbols, like
the mathematician's real number line, does not magically make this
model not symbolic.
\begin{figure}
\center
\includegraphics[width=10cm]{gfx/magritte_pipe}
\caption[``The Treason of Images'' (``This is not a pipe'') by
  \cite{magritte:1919} is an image of a pipe.]{``The Treason of
  Images'' (``This is not a pipe'') by \cite{magritte:1919} is an
  image of a pipe.  The symbolic image is not equivalent to its actual
  referent, which is neither symbolic nor an image.}
\label{figure:magritte_pipe}
\end{figure}

Models are all intentionally created, based on discrete symbols, in
order to reference and describe this continuous space of continuously
dynamic activities.  A reflective AI, therefore, must be a model of
the relationship between the goals of the responder and the creation
of relevant models of continuously dynamic activity.  In other words,
a reflective AI, must explicitly be a programmed model of this
fundamental modelling problem, a model of learning when a given model
is good or bad for accomplishing its own goals.  A reflective AI would
be able to question the utility of these caricatured models of
physicists, spiritualists, and philosophers, based on whether or not
they help it to accomplish its own goals.

\section{Dynamically Grounded Description}

A grounded description references the dynamic.  Symbols, words and
language refer to undifferentiated amorphous perceptions or ongoing
general experience and have grounding only in this way and create the
tools with which the undifferentiated can be manipulated, understood,
or communicated.

The further ability of symbols to also reference other symbols allows
the additional possibility of creating a referential circularity in a
static model.  An illusion of a primitive grounding is possible in
such a cycle because all of the symbols have a referent; however,
because all references within such a closed cycle appear to be
exclusively kept within the static model and do not seem any longer to
refer to anything dynamic, there is a danger of there being no meaning
other than circularity to such a purely static construct.  Symbols are
tools for modelling and are useful only if the circularity is not
mistaken for a grounded reference.  Thus, I strictly avoid presenting
such illusions of grounding as primitive in this dissertation.  To
allow for grounded descriptions, I will first outline my conception of
the dynamic, to which my description ultimately must refer.  A
description of reflective thinking referring to an \emph{a~priori}
concept that has not been derived in terms of symbols allows for
something to remain outside of the static model as the dynamic
referent.  This last step avoids the creation of a purely
self-referential cycle and prevents illusions from becoming mistaken
for grounded parts of the model.  In the next section I will describe
the necessary conception of the dynamic to which my description of
reflective thinking will refer.

\section{Activities in Duration}

\cite{bergson:1910}, in his seminal \underline{Time and Free Will},
discusses the dynamic as heterogeneous activities in Duration.
Activities in Duration will serve as my conception of the dynamic for
my description of reflective thinking.  Activities in Duration are the
dynamic ongoing activity, heterogeneous activities that are neither
distinct nor separate.  There is a trick, a sort of slight of hand,
required here in describing Bergson's conception of the activities in
Duration because, even though referred to through symbolization, these
activities exist independently of symbolization.

For example, when I look at a tree, I may say that the tree has green
leaves and a brown trunk; the symbols, green and brown, refer to the
activities in Duration, which I am seeing.  When I focus carefully on
the dynamic activity of my situation, I see that the leaves and trunk
could be described by a greater combination of various greens,
yellows, grays, browns and reds.  Symbols limit perception to a
discrete sort of projective presupposition.  Therefore, as I introduce
more symbolic references in my increasingly limited description of the
dynamic, a growing danger of losing focus on something dynamic
emerges, creating a cycle of symbols seemingly referring exclusively
to symbols.

The motivation to tangentially define more clearly what I mean by the
symbol green or brown by using more symbols results in a circularity
from which there is no exit, and therefore, a false dilemma from which
I will never again be able to see or talk about any dynamic tree as
the referent.  In order to avoid the danger of purely cyclic symbolic
references in my model, I will continue to relate the parts of my
model to the dynamic activity of reflective thinking, which exists as
activities in Duration.

\section{Actual Reflection}

The activities in Duration are the given dynamic ongoing.  These
activities are the inseparable actions that are currently available to
reflection.  Reflection is the ability to symbolize this ongoing
activity.  For example, the activity of perceiving a table can be
symbolized as the name ``table''.  Similarly, the activity of walking,
can be symbolized as present participle ``walking''.  Note that my
model of reflective thinking assumes that there is ongoing activity
before any symbols exist for the actions of thinking to manipulate.
Thinking is an auxiliary activity that manipulates static symbolic
constructions that refer to the primary ongoing activity of existence.
Thus, the activity of perceiving a table requires neither that this
activity itself be reflectively symbolized nor that a secondary
thinking activity manipulates the resulting symbols.  I will use the
phrase \emph{actual reflection} to refer to the activity of
symbolizing ongoing activities.

\section{Order of Space}

Activities in Duration may be symbolized in a homogeneous medium
without activity.  \cite{bergson:1910} calls this medium Space,
allowing symbolic references to activities in Duration to be related.
This is the primary fundamental distinction in both the mathematical
simulation and computer implementation of reflective thinking.  Of
major importance is the fact that one does not derive the other, but
activity and order coexist with neither Duration being a derivative of
Space nor Space containing the fundamental activities in Duration.
Obviously, Bergson has gone through painstaking efforts to leave room
in his description for the dynamic as some independent, non-contingent
activity not derived from either Space or Duration.  He makes the
necessary concession that a description of something dynamic is only
grounded in that it leaves room for this dynamic to exist
independently of its own symbolization.  This modelling work is meant
to be similarly grounded in the necessarily separate dynamic activity
of reflectively learning to accomplish goals.

\section{Modelling the Dynamic}
\label{section:modelling_the_dynamic}

Duration and Space are the fundamental duality that forms the basis of
my model of the dynamic activity of reflectively thinking.  Thus this
duality exists pre-reflectively, prior to symbols being created from
the activities in Duration and prior to these symbolic references
being related in Space.  Note the fundamental contradiction of
intentionally describing the dynamic only in static symbolic terms.
For example, if Space has no activity and is not itself composed of
activities in Duration, to what is the symbol ``Space'' referring?
Symbol ``Space'' refers only to the logical potential for an absence
of activity, the potential for itself as not existing; thus, it is the
logical necessity of the negative implication of positive existence,
to which no alternative can be assigned.  Again, my model explicitly
does not provide a definition of this implicit, fundamental and
irresolvable contradiction.  Thus, my description will appear to begin
in the middle of something endlessly dynamic and ongoing, the \emph{in
  medias res} of an underived time and space, maintaining an explicit
awareness that static descriptions are only tools for dynamic
reflections.  Duration and Space are not placeholders for the dynamic;
they are intermediate programmed fictions in the implementation that I
use here to model the dynamic ongoing, the actual activity, which is
not a static arrangement of symbols.

\section{Spatial Reflection}

The activities in Duration are the dynamic activity of being in the
present.  Symbols, like the photographic, are static references to
this dynamic.  Again, there is a fundamental distinction here between
the dynamic and the static.  Because AI already refers to an AI model,
I use an asterisk notation to refer a model that the AI learns in
simulation.  In this way, Spatial arrangements of symbols can actually
exist in the dynamic as an actively maintained static construction, a
model* within the AI.  The ongoing activities that hold symbols in a
specific Spatial arrangement can be symbolically reified.  I will
refer to the activity of symbolizing the activities of a Spatial
arrangement of symbols as \emph{Spatial reflection} or
\emph{reification}.  Spatial reflection is a form of actual
reflection.

\section{The Physical Layer}

My model includes three layers of ongoing dynamic activity.  I will
use the phrase \emph{physical layer} to refer to the pre-reflective
layer of activities that exist necessarily prior to any subsequent
reflective symbolization activity.  The physical layer is the only
pre-reflective layer of activity in my model.  Therefore, a clear line
is drawn between the physical layer and those activities that
symbolize and Spatially arrange symbols.  Thus, they are not included
in what I will refer to as physical activity.  Note that in most AI
models, there is a distinction between the mind and the body or the
environment.  In my model, \emph{all} activities that exist are
referred to as \emph{the mind}.
\footnote{Because my model of mind is an absolute reference to the
  dynamic, it makes very few modelling assumptions.  Because my model
  has been previously mistaken for the philosophy of ``Subjective
  Idealism'', I have included a description of the mistake of
  Subjective Idealism, and a clarification that my model does not make
  this mistake, an important difference between my model and this
  problematic philosophy; I explain in
  \autoref{section:the_mistake_of_subjective_idealism}.}  Thus, the
dynamic mental activities that I am referring to in my model are not
an object that is viewed subjectively.  Critical to an understanding
of the goals of objective science is that the creation of object and
subject distinctions is allowed to be a part of the activities of the
mind that can be simulated with my AI; because this understanding is
so critical to an advancement of scientific understanding, I will
clarify the mental derivation of objective science in
\autoref{chapter:science}.  Because the abstract creation of object
and subject relationships is a relatively advanced mental activity and
not key to my thesis, I will discuss this as future research in
\autoref{chapter:future}, which would be a promising place to extend
my model to include ideas like physical bodies and other subjective
and objective environmental perspectives.

\section{Orders of Reflective Layers}

In addition to the pre-reflective physical layer of activity, my model
includes two additional layers of reflective thinking activity that
use two different classes of causal models in order to accomplish two
different classes of goals.  I will use the phrase \emph{first-order
  reflective layer} to refer to the reflective layer of thinking that
creates symbols and Spatial arrangements that refer to physical
activities.  I will refer to reflectively symbolizing physical
activities as \emph{physical reflection}.  Physical reflection is a
form of actual reflection and is one of the primary reflective
thinking activities; remember that physical activities are
pre-reflective, so physical reflection is not itself a physical
activity.

Obviously, the first-order reflective layer cannot create symbols or
Spatial arrangements that refer to its own activities.  The
first-order reflective layer has the \emph{a~priori} ability to
maintain its own actively reified Spatial relationships as existing
between symbols that refer to physical activities.  The lack of the
ability for the first-order reflective layer to refer to its own
symbolically creative or Spatially manipulative activities acts as a
sort of protective ``firewall'' that clarifies my model by keeping
clear distinctions between each class of causal knowledge in each
subsequent reflective layer of thinking.  Before describing
distinctions between classes of causal models, I will spend the next
few sections describing the composition of temporal transitions and
causal hypotheses.

Reflection over the activity of the first-order reflective layer
necessarily creates a second layer of reflection and thinking.  Thus,
clearly deriving and describing this second layer of reflective
thinking additionally necessitates an implicit and arbitrary number of
layers.  This condition is the focus of this dissertation.
\cite{minsky:2006} describes a six-layer reflective model of thinking
about thinking.  The first three layers of my model and
implementation, including these two layers of reflective thinking, can
be compared to the first four layers of Minsky's model.  There are
many close parallels between my model and Minsky's model, and I will
describe these modelling parallels in
\autoref{chapter:related_models}.

\section{Consciousness, Awareness, and Experience}

There are many terms that are used in the literature when referring to
the concept of the dynamic.  Each of these terms has a different
associated collection of necessary contingencies that, as isolated and
independent, tend to become absurdities.  For example, the general
term consciousness always begs the question \emph{of what} one is
specifically conscious.  There is an implicit independence of subject
and object when using this transitive term, which is not assumed when
the term is used as a reference to the dynamic.

The terms awareness and experience are used similarly and have the
same logical contradictions as the term consciousness when they are
considered as independent isolated references to the dynamic.  Each of
these terms has different meaningful uses when they are not used in
this independent sense.  For example, in this dissertation an
awareness of the potential confusions these terms introduce when they
are used singularly in abstract isolation as reference to the dynamic
will be maintained.

Thus, I will not argue that these conceptions of the dynamic are
incorrect.  Rather, they are limited to particular and less
generalized models that are restricted descriptions of the dynamic.
In this dissertation, however, in order to avoid confusion, I will
never use these terms in their limited and abstracted sense.  Instead,
I will conflate all of these uses of these terms as all being the same
as my conception of the dynamic, the ongoing activity that my model
accepts, symbolizes, and thinks about as given.

\section{Extensive and Intensive Quantities}
\label{section:extensive_and_intensive_quantities}

Sometimes it is hard to imagine how a fundamentally symbolic model can
ever refer to the ``intensities'' of perceptual stimuli.  Thus, in
this section, I make a distinction between two types of intensities,
the ``extensive'' and the ``intensive''.  This distinction is used to
explain how the basic symbolic component of the model can come to
represent various aspects of everyday thinking that, according to
common sense, are fundamentally of a certain intensity.

\cite{bergson:1910} makes a key distinction between comparisons of
``extensive'' quantities and comparisons of ``intensive'' quantities.
Extensive quantities are Spatial comparisons.  \emph{Extensive
  quantities} are between two symbolic references that share a
containment relationship in terms of their referents: one symbolic
referent is inclusively contained by another symbolic referent in
Space.  For example, one body may be said to be larger than another in
terms of its Spatial extent.  Intensive quantities are more subtle and
an example is in terms of pain, which can be said to be more or less
intense.  \emph{Intensive quantities} are similar to neither
numerically nor Spatially extensive containment relationships but,
instead, refer to different fundamental activities in Duration.  For
example, a pain that has no intensity is not a pain at all; a pain
that has a mild intensity may be equivalently referred to as an
irritation or an itch; a pain of high intensity may be equivalently
referred to as a sharp pain or a throbbing pain.  The point is that
these intensities of quality are actually very different from
extensive quantities in that they are not greater than or less than
one another in the sense of a containment relationship.  For example,
a very intense pain, such as a shooting pain or a throbbing pain, does
not in any sense contain a lesser mild pain, such as an irritation or
an itch.  This same pattern exists with detailed descriptions of other
qualities that are often thought of as having comparable intensities:
heat and cold; light; pressure; sound; pitch; aesthetic feelings, such
as grace and beauty in music, poetry and art; emotions, such as rage
and fear; moral feelings, such as pity; affective sensations,
including pleasure, pain and disgust.  Each of these examples points
out that the apparent relationships of greater than or less than with
respect to intensity are static Spatial arrangements of different
fundamental activities in Duration.

\section{Number}
\label{section:number}

Some models make the complicating assumption that numbers are an
additional component to every symbolic perception just for this
purpose of modelling intensity.  In this section, I describe how
numbers are related to the model: numbers are thought of in my model
as arrangements of symbols in Space.

\cite{bergson:1910} shows the derivation of all numerical
representations as requiring an accompanying extension in Space.  For
example, in referring to the individual sheep in a flock of sheep, he
explains counting:

% pg. 75-77
\begin{quote}
Number may be defined in general as a collection of units, or,
speaking more exactly, as the synthesis of the one and the
many\ldots

No doubt we can count the sheep in a flock and say that there are
fifty, although they are all different from one another and are easily
recognized by the shepherd: but the reason is that we agree in that
case to neglect their individual differences and to take into account
only what they have in common.  On the other hand, as soon as we fix
our attention on the particular features of objects or individuals, we
can of course make an enumeration of them, but not a total\ldots

Hence we may conclude that the idea of number implies the simple
intuition of a multiplicity of parts or units, which are absolutely
alike.

And yet they must be somehow distinct from one another, since
otherwise they would merge into a single unit.  Let us assume that all
the sheep in the flock are identical; they differ at least by the
position which they occupy in space, otherwise they would not form a
flock.
\end{quote}

The last point here illustrates the idea that numbers are always
derived from symbols situated in Space.  Therefore, a model for
thinking based directly on symbols cannot, by definition, be more
complicated than a model based on a numerical representation.  A
number is, after all, only the reified result of an axiomatic
construction of Spatial organization of symbolic references to the
dynamic.  In other words, they are specific sheep of the dynamic.  The
explicit implication is, therefore, that, in building an AI that is
capable of reflective thought, the model building process and its use
of numerical representations must be explicitly available for
inspection by the AI itself.  Here, I employ Occam's razor to simplify
the loop of reflective representation in order to ease building the
first proof-of-concept examples of AIs capable of reflective thinking.

\section{Time as Space}

Symbolic references can be ordered Spatially.  These symbolic
relationships can be reified and treated symbolically by further
ordering Spatial relationships.  Thus, Bergson explains time as a form
of Space:

% pg. 98
\begin{quote}
Now, if space is to be defined as the homogeneous, it seems that
inversely every homogeneous and unbounded medium will be space.  For,
homogeneity here consisting in the absence of every activity, it is
hard to see how two forms of the homogeneous could be distinguished
from one another.  Nevertheless it is generally agreed to regard time
as an unbounded medium, different from space but homogeneous like the
latter: the homogeneous is thus supposed to take two forms, according
as its contents co-exist or follow one another.  It is true that, when
we make time a homogeneous medium in which conscious states unfold
themselves, we take it to be given all at once, which amounts to
saying that we abstract it from duration.  This simple consideration
ought to warn us that we are thus unwittingly falling back upon space,
and really giving up time.
\end{quote}

Time exists as a static creation, a derivative arrangement of symbols
in Space.  Understanding time to be static is important because this
allows the dynamic heterogeneous activities of Duration and a static
homogeneous unqualified Space to remain, respectfully, neither
distinct nor separate.  By prohibiting this separation, and, thereby,
their independent existences, I avoid defining the dynamic activities
in Duration in terms of static symbols, which, in turn, avoids the
creation of a self-referential cycle, thus sidestepping the potential
illusion of a false dynamism.  Being clear in our understanding of
time used intentionally as a static tool of thought avoids confusing
the focus of the modelling process to focus on modelling the dynamic.
Again, a model that is purely of a model leads to an AI based on a
tautology because models are only grounded in that they reference the
dynamic.

\section{Temporal Reflection}

Spatial arrangements of symbols can be used to represent a past
through memory and subsequently project that distinction as an
analogous ``future'', through imaginative action, thus constructing a
derivative conception of time.  The past and the future, therefore,
remain static constructions that are actively maintained by the model
as Spatial relationships in the dynamic present.  I will refer to the
reification of Spatial relationships that order the past and the
future as \emph{temporal reflection}.  Sequential time is created
through temporal reflection.  Thus, temporal reflection is a form of
Spatial reflection.  Like Spatial reflection, temporal reflection is
also a form of actual reflection, symbolizing the active maintenance
of a temporal Spatial relationship between static symbolic references.

\section{Simultaneity and Transition}

Correlation requires two symbols to be related in Space.  As symbols
are correlated in the Space called time, these temporal correlations
are referred to as either simultaneities or transitions.
Simultaneities represent two symbols that are both actively symbolized
in Duration.  Transitions represent the change from the past to the
future.  Transitions can be extended in time in order to create
counterfactual temporal extensions, called inferences.  Inferences of
the past and the future can be created by matching and repeating these
transitions.  A transition implies the removal of one symbol and the
addition of another as a progression is made through a temporal
sequence.  If correlations are counted and compared in ratios, then
these are called probabilistic correlations, and the resulting
inference would be a probabilistic inference.

