%************************************************
\chapter{Objective Science}
\label{chapter:objective_science}
%************************************************

\section{Body and World}

An intelligent mind builds models that predict the occurrence of
perceptual symbols.  Analogical reasoning is used to abstract objects
and simultaneously the implied subjective perspectives on these
objects as they are manipulated.  The separation between the physical
body and the physical world is an objective form of reasoning that is
fundamentally caused by goals that emphasize distinguishing bodily and
worldly goals in the causal structures of the physical perceptions.
For example, physical actions that change physical perceptions in
predictable ways are often bodily physical perceptions, i.e. moving
ones hand in front of ones face causes one to consistently see a hand.
Also, physical pain perceptions could be related to physical goals
that emphasize the artificial separation between the body and the
world.  Understanding the artificial separation of the body and the
world allows for an objective form of scientific study that places the
object of study in the world outside of the effects of the body.

\section{Physics}

The physicist uses an objective form of reasoning in order to separate
his or her self as the subject from the object of their study.
Objectification of physical perceptions has led to the artificial
creation of universes, galaxies, stars, planets, humans, organs,
cells, molecules, atoms, and even the fascinating electrons, which can
be thought of as wave objects or particle objects, depending on the
goals of the subject.  In each of these studies, symbolic perceptions
are correlated into causal hypotheses and these causal hypotheses are
grouped into plans that are analogically abstracted into objects, each
different objective abstraction having different subjective
implications.  Note that objects and their implied subjective views
are artificial constructions caused by the goals of an intelligent
mind.  The fact that the position and momentum of an electron can be
considered objectively as probability distributions is fundamentally
derived from counting symbolic perceptions and putting the resulting
numbers into ratios.  The symbolic perceptions in this task are the
closest thing to the real activity of perception.  Once the real
activity of perception has been reflectively symbolized, all that
remains is to manipulate the artificial Spatial arrangements of these
symbols.

\section{Neuroscience}

In the field of AI, there is a tendency to confuse an objective
scientific model of the brain with a theory of mind, which is, among
other things, an explanation of the artificiality of the object and
subject distinction.  Neuroscience is fundamentally an objective
physical science.  Neurons are one of the most important objective
discoveries of neuroscience.  Like all objective physical sciences,
the objects discovered are caused by goals.  The objective study of
neuroscience has led to distinctions between the central and
peripheral nervous systems; forebrain, midbrain, and hindbrain
divisions; cortical columns and circuits; nerves and pathways; all
composed of neurons.  The objectification of the nervous system is led
by the goals of the scientist, which are usually medical.  Recently,
the study of neuroscience has combined with psychology and AI, which
allows for studying not just the physical structure, but how this
physical structure relates to a theory of mind.  How is reflective
reasoning, which requires symbols, done by the brain?  This is an
interesting question, but I want to make the point that in order to
study a question like this one must not confuse a theory of mind with
the physical objective model, both being necessary for posing a clear
question relating thinking and the brain.

\section{Neural Networks}

Neural networks are one of the objects created by the field of
neuroscience.  Given the physically objective assumptions of the
field, we have learned of the interactions between many different
objects related to neural networks: the brain is composed of
approximately one hundred billion neurons; chemical gradients guide
neuron growth and attrition; neurons use chemical binding and
electrical potential differences to communicate through axons to
dendrites.  There are many computational models that have been created
to explain different physical phenomena that arise from the
interactions of many neurons.  Models of biological neural networks
are generally referred to as \emph{artificial neural networks}.  The
study of neural networks is an objective physical science, like all of
neuroscience.

\section{Artificial Neural Networks}

Although artificial neural network models were originally created in
order to explain the behavior of physical neurons, a subset of these
models that were modelled as simple linear and non-linear algebraic
equations have become popular as a subset of control and optimization
theory.  These control theory models been found to be useful for
controlling robot arms, car braking systems, and even music
recommendation systems.  It is important to make a distinction between
three ideas here: neural networks, numerical control models, and
reasoning.  Reasoning is the activity that builds models of all of
these phenomena, including itself.  Algebraic control systems are
symbolic models that assume that one knows how to count, add, and
multiply.  Neural networks are objects composed of many symbols that
neuroscientists subjectively study.  Note that of these three ideas,
reasoning is the only one that is actual; the other two are static
models, which are created by the reasoning activity.

\section{Brain and Mind}

The brain is one of the key organs that keeps humans alive.  The
nervous system is a key factor in muscular movement of the body;
perceptions, including vision, smell, touch, etc.; speech production;
language understanding; reasoning, including counting, making plans,
and doing mathematical calculations.  When I say that the brain is
key, what I mean is that when the brain is damaged, these functions
are lost.  Understanding neural networks, like most neuroscientific
pursuits, is often driven by medical goals.  If a neural network is
understood in terms of how it leads to functionality, such as
reasoning, then we can use this physical understanding in correlation
with a theory of mind to design preventative, rehabilitative, and
augmentative approaches to neural medicine.

There is a danger at this point of becoming confused and accepting a
model in place of the reality of the ongoing activity in Duration,
which requires neither symbols, objects, nor any other form of
reasoning to exist.  A model of mind is a construction that is used to
think about this ongoing existence.  A model of the brain is a
construction that is based on the artificial physically objective
assumptions that divide bodies, organs, cells, etc.  Note that a model
of mind does not necessarily make any objective assumptions; in this
sense, a model of mind can model the exploration and creative activity
that leads to a model of the brain.  A model of the brain is a
physically objective model, and does not lead to models of thinking
because of the assumptions of purely physical objective focus.
However, regardless of focus, both examples of modelling must keep the
reality of the ongoing activity as the fundamental referent for any
symbolized perceptions or causal models.  Some models may be
derivatives of others, but all models are artificial.

