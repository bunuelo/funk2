%************************************************
\chapter{Problematic Interpretations}
\label{chapter:problematic_interpretations}
%************************************************

\section{The Category Mistake of Subjective Idealism}
\label{section:the_category_mistake_of_subjective_idealism}

\cite{berkeley:1734} describes a philosophy called ``Immaterialism''
or ``Subjective Idealism''.  The problem with Subjective Idealism is
simple.  Here, I provide an explanation.  Subjective Idealism has an
absolute goal of describing the mind as a subset of the entire world.
Thus, a decision is made to restrict the focus to a subjective
perspective on physical objects.  In Subjective Idealism, the term
``mind'' is used to refer to the subject.  Their description of the
subject then becomes a mental subject in relation to physical objects,
the material world.  In other words, by defining the mind to be a
subject, they have claimed that the mind is only a part of the
absolute totality of everything, which was their intent, so, success.
This restricts their following conclusions about the mind, the
subject.  Based on this division, they separate everything into the
subject and the physical objects as two primary assumed categories of
things.  This is one of the first assumptions of all objective
science.  Objective and subjective divisions of the world are
amazingly useful, as the physical sciences are living proof.  There is
nothing wrong with thinking about physical objects from a subjective
physical perspective.  My model is absolutely compatible with thinking
about physical objects from a subjective perspective.

The problem with Subjective Idealism has nothing to do with the fact
that they divide everything into subjects and objects.  The problem
with Subjective Idealism is what their theory says next: \emph{only
  minds and mental contents exist}.  This philosophy rests firmly and
exclusively on one side of a subjective division of reality.  Berkeley
claims that the subject exists and that \emph{the objective physical
  world does not exist}.  In other words, the philosophy begins by
dividing everything into a subject and objects, followed by, saying
that the subject is everything.  The mistake is this: everything
cannot be equivalent to a strict subset of itself.  The model has a
loopy contradiction in its definition.  This is what Gilbert Ryle
would call a ``category mistake'' \citep{ryle:1949}.  The idea is
absurd.

There is a danger of interpreting my model as superficially similar to
the philosophy of Subjective Idealism.  After all, \emph{my model of
  mind includes everything that exists;} however, my model does not
make the fatal flaw of first dividing everything that exists into a
subject, the mind, and physical objects, which subsequently are not
included in everything that exists.  \emph{My model does not make the
  category mistake of Subjective Idealism.}  My model includes
physical activities that exist; this is in fact included in the basis
of my reflective model in the $\text{reflective}^0$ or physical layer.
\autoref{section:objective_reflection} describes how objective
thinking would be based on learned abstractions in my model.

\section{Mind as Subjective to Objects}

My theory does not refer to an external physical simulation, a source
of perceptions and actions for the mind.  In my theory, the mind
creates its own symbolic perceptions, which it would have no reason to
do without a pre-existing symbolized goal, also of its own creation.
In other words, because my theory includes everything that actually
exists, my theory does not assume any objective or subjective
perspectives or the inherent devastating limitation of this modelling
assumption, to never be able to create symbols that are meaningful in
the sense that they reference the actual ongoing activity in Duration,
reality.

Under the subjective modelling assumption, all perceptions and actions
are artificial symbols that have been created implicitly in the
assumption that the mind does not include some things.  The implicit
assumption becomes explicit when an interface between the ``outside
world'' and the ``inside world'' must be made; at this point, the
symbols must be already defined for all perceptions and actions for a
subjective model of mind.

Considering minds to be in a subjective relationship with a world of
objects is in the traditional view, being implicit in the assumptions
of Model-6, HACKER, EM-TWO, and many other obviously useful theories.
The scientific method makes great use of the subject and object
distinction.  In this view, the mind is part of a larger existence
outside of itself.  This is a subtle but important distinction between
my model and these models: my model and these models are not equal in
scope.  My model of mind includes the creation of explicit references
to reality, everything that actually exists.

My model is not a subject in relation to objects, but instead inverts
this paradigm, being eventually responsible for creating object and
subject abstractions while viewing itself through these abstractions.
The explicit inclusion of references to the pre-symbolic activities in
my model allows my model to have an arbitrary number of layers that
reference the ongoing activity of the mind, while maintaining an
adaptive relationship with the given ongoing activities of itself,
everything that actually exists.

Seeing a model of mind as subjected to objects robs the AI from ever
modelling an awareness of the responsibility for this distinction as
its own creation.  In other words, reducing the mind to a subjective
role eliminates the potential for modelling the mind's reflection on
its own activities that are responsible for the creation of that very
primitive object and subject distinction, the self.
