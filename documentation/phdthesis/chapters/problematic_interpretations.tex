%************************************************
\chapter{Problematic Interpretations}
\label{chapter:problematic_interpretations}
%************************************************

\section{The Mistake of Subjective Idealism}
\label{section:the_mistake_of_subjective_idealism}

I have found that, readers will sometimes confuse my model of mind
with the philosophy of Subjective Idealism or Immaterialism
\citep{berkeley:1734}.  This is a mistake, and so, here, I will
explain, for the interested reader, the mistake of Subjective Idealism
and how my model explicitly avoids it.

In fact, my model shares similarities with the philosophy of
Subjective Idealism.  Specifically, my model uses the term ``mind'' to
refer to everything that exists.  This is true, also, of this
philosophy.  However, it would be a mistake to confuse this semantic
detail being somehow as equivalent to Subjective Idealism.  This
philosophy, in fact, makes a fundamental mistake.  First, the
philosophy of Subjective Idealism states that everything is
axiomatically divided into subject and object parts.  Then,
subsequently, the subject part is defined to be everything.  In other
words, Subjective Idealists first believe that the world is divided
into two parts, the subject and objects, and then subsequently, and
illogically claim, that the subject is all that exists.  Obvisously,
there is a tautological cycle in this model.  The mistake can be found
in Subjective Idealism's definition of ``everything''.  Everything
functions as the origin of the model as well as the ``part''.  Thus,
part is, by definition not everything.  There is a causal circularity
in even attempting to imagine this everything that is being divided
into these parts in the first place.

For example, here's another way to realize the contradiction.  If one
is aware of having a subjective perspective on objects, one must have
a vantage point from which to be aware of this.  This vantage point is
necessarily outside of the subjective perspective.  If the mind is the
subjective perspective, then one could never become aware of being
subjective, since there would be nothing in existence to contrast
``everything''.  My theory explicitly avoids this.

My model does not posit a subjective perspective on objects.
Therefore, when I define the mind to be everything that exists, this
limitation avoids the fatal flaw that would prevent the modelling of
an awareness of subjective points of view.

