%************************************************
\chapter{Problematic Interpretations}
\label{chapter:problematic_interpretations}
%************************************************

\section{The Mistake of Subjective Idealism}
\label{section:the_mistake_of_subjective_idealism}

I have found that, although this is just a philosophical point and has
nothing to do with my implementation, sometimes readers think of
Subjective Idealism or Immaterialism \citep{berkeley:1734} when they
read my theory of mind in \autoref{part:theory_of_mind}.  This is a
devastating mistake, and so, here, I have explained, for the
interested reader, the mistake of Subjective Idealism and how my
theory explicitly avoids it.

My model shares similarities with Subjective Idealism.  Specifically,
my theory uses the term mind to refer to everything that exists.  This
is true, also, of this philosophy.  However, it would be a tragic
mistake to confuse my theory as equivalent to Subjective Idealism
because this philosophy makes a mistake.  First, everything is divided
into subject and object parts.  Then, subsequently, the subject part
is defined to be everything.  Subjective Idealists believe that the
world is divided into two parts, the subject and objects, then they
say that the subject is all that exists.  There is a tautological
cycle in this model.  They use the term mind to refer to everything,
but there is a mistake in their definition of everything.  Everything
is both the root of the model as well as the part; there is a causal
circularity in even attempting to imagine this everything that is
being divided into these parts in the first place.

For example, here's another way to realize the contradiction.  If one
is aware of having a subjective perspective on objects, one must have
a vantage point from which to be aware of this.  This vantage point is
necessarily outside of the subjective perspective.  If the mind is the
subjective perspective, one could never become aware of being
subjective.  It is a fatal flaw in Subjective Idealism, which my
theory explicitly avoids.

My theory is not defined to be a subjective perspective on objects.
Therefore, when my theory defines the mind to be everything that
exists, I do not make the fatal flaw that would limit my model from
modelling an awareness of being subjective.

