%************************************************
\chapter{Problems to Solve}\label{ch:problems_to_solve}
%************************************************

\section{Build a Reflective Knowledge Substrate}

The assumption that we introduced in
Section~\ref{sec:introducing_reflection_early_in_the_process},
``Introducing Reflection Early in the Process'', requires that changes
in our knowledge representation can be traced and compiled into other
representations, such as the reflective event representations used in
learning to accomplish goals.

\subsection{General Parallelism and Concurrency}

\subsection{Automatic Collection of Audit Trails for All Processes}

\subsection{Program as Data}

The ability for a process to manipulate a program as data makes it
possible for that process to write and compile an efficient new
programming language.  A process that learns to accomplish goals is a
process that designs and uses its own new programming languages in
order to solve problems in different domains.  


%I have constructed just such a substrate.  It provides for general
%parallelism and concurrency, while supporting the automatic collection
%of audit trails for all processes, including the processes that
%analyze audit trails.  My system natively supports a Lisp-like
%language.  In such a language, as in machine language, a program is
%data that can be easily manipulated by a program.  This makes it
%easier for a user or an automatic procedure to read, edit, and write
%programs as they are debugged.


\section{}


\section{Learning by Credit Assignment}

\subsection{Use Reflective Representations for Better Models of Learning}

\subsection{Tracing Knowledge Provenance for Credit Assignment of Success or Failure}



