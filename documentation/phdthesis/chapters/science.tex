%************************************************
\chapter{Science}
\label{chapter:science}
%************************************************

Physically objectifying problems scientifically has proved useful.
The primary utility of this understanding is that the scientist can be
replaced by another scientist and the experimental results can be
duplicated.  Arbitrary replacement of the subjective scientist implies
the potential for mechanical simulation of the physical phenomenon,
reducing the subjective scientist to a purely perceptual role.
Mechanical duplication of human labor leads to increased efficiency
and productivity.  First, steam engines were a model of autonomous
human activity.  Next, computers became the physical model of choice.
Now, various humanoid robots have included electrical motors and
computers to become closer reproductions of human physical abilities.
Thus, understanding a model of mind has become important in
duplicating more complex human physical behaviors that require
thinking.  The mechanical duplication of human thinking is the goal of
the field of AI.  Only the simplest forms of human thinking have been
coherently mechanized thus far.

\section{Object and Subject}

Causal hypotheses are used in order to predict the future occurrence
of symbolic perceptions and goals.  Through reflective thinking, plans
are constructed from causal hypotheses consistent with the past,
elaborating the necessities in the past and the results in the future.
Analogies between consistent plans can be abstracted into models of
objects.  For example, a plan that begins and ends with the same
symbolic perception could be considered an example of a ``circular''
object; thus, an analogical plan abstraction would represent a type of
object, that has multiple sides to perceive depending on the subject's
position in the plan.  Circular objects may be an important type of
object to analogically recognize because a circular object allows one
to perform actions while being able to get back to a known perceptual
symbol.  If one is in a known circular object, then one is not lost in
the sense that one can always follow the circle in order to get back
to any perceptual symbol contained within the circular plan object.
My implementation includes tools for performing analogical
abstractions and constructions of plans, but this machinery is not key
to my thesis of explaining causal reflective learning to accomplish
goals.  I include this section in the dissertation to eliminate a
potential confusion that would conflate the concept of symbol with
that of object.  Symbols are the most primitive elements available to
thinking, while objects are more complicated in that they are composed
of analogical consistencies between plans that are themselves composed
of many symbols.  Objects have multiple subjective perspectives.
Objects are artificial creations of a mind, based on artificial static
symbols representing the ongoing activity in Duration.

\section{Body and World}

An intelligent mind builds models that predict the occurrence of
perceptual symbols.  Analogical thinking is used to abstract objects
and simultaneously the implied subjective perspectives on these
objects as they are manipulated.  The separation between the physical
body and the physical world is an objective form of thinking that is
fundamentally caused by goals that emphasize distinguishing bodily and
worldly goals in the causal structures of the physical perceptions.
For example, physical actions that change physical perceptions in
predictable ways are often bodily physical perceptions, i.e. moving
ones hand in front of ones face causes one to consistently see a hand.
Also, physical pain perceptions could be related to physical goals
that emphasize the artificial separation between the body and the
world.  Understanding the artificial separation of the body and the
world allows for an objective form of scientific study that places the
object of study in the world outside of the effects of the body.

\section{Reality as ``Out There''}

Scientists and others often view reality as ``out there''.  What this
means is that reality is viewed as a collection of objects, some of
which have been discovered, and many of which are still ``out there''
and ready to be discovered.  ``Out there'' generally means outside of
the physical body and in the physical world.  To me, this view of
reality as ``out there'' is grounded on a physically objective view,
which implies a subject; equivalently, one could state that this view
of reality being ``out there'' could be grounded on a subjective view,
which implies objects.  In either case, the view of objects is not
mentally grounded on anything less than objects or their implied
subjects.  The mind is not an object, and, logically, studying the
mind is not a subjective task.  The scientist studies objects to great
utility, but what is often not clear are the goals that have driven
the measurement of utility with these subjective objectifications.
The objects are taken as given to the mentality as if the mind did not
create the objects in order to think subjectively in the first place.
The idea of objective thinking being less prone to error than
subjective thinking is purely two sides of the same coin, and the
error in both cases is in respect to the goals of the creator of the
distinction.  With every object is an implied set of subjective
perspectives, and every object and subject distinction is useful with
respect to the goals that relate to its creation and experimentation.

One of the key distinctions between those who study the mind and
scientists is that scientists do not explicitly model the goals that
drive the creation of their objects of study, while those who model
the mind do not model an object from a subjective perspective because
the mind is not an object.  The mind is everything that exists.  A
model of the mind includes and explicitly states the goals that drive
it to create objects and their implied subjective perspectives.
Therefore, reality is not ``out there'' but reality is everything that
exists.  Seeing reality as ``out there'' is based on an artificial
object and subject distinction that often carries with it undisclosed
goals for its creation.  It is therefore a mistake to see this
artificial object and subject distinction as an unquestionable
fundamental quality of existence.

It is an option for scientists to be aware of the goals that drive
their subjective studies of objects.  A goal is not a physical object
that a scientist can study subjectively.  A goal is part of a model of
mind; so, therefore, in order for a scientist to be aware of their
goals, he or she must understand a model of mind that allows them to
understand their chosen goals in relation to everything else that also
exists.  Without this awareness, there is a critical danger of the
blind leading the blind with the scientist not being aware of their
choice of a subjective perspective on the world.

\marginpar{\emph{Understanding this distinction replaces the role of
    the subjected victim with that of the responsible creator.}}  Most
importantly, we are not fundamentally subjected to objects; we are
instead the creators of artificial objects, based on many artificial
symbolic references to the actual dynamic ongoing activity.  This is a
subtle distinction, but understanding this distinction replaces a role
of the subjected victim of fundamentally unquestionable objects with
the role of the personally responsible creator of artificial objects
with implied subjective perspectives.  Understanding that objects are
the artificial creations of an intelligent mind is something that
seems to have escaped the explicit statement of many great scientists.
The fact that the study of the mind is not a subjective field
separates it from all sciences.  The mind is not able to be studied
scientifically for this exact reason.  Science is based on objective
and subjective distinctions that are creations of a mind.  Studying
and understanding a model of mind is akin to understanding causality;
a prerequisite before posing a scientific hypothesis.  Therefore,
reality is not ``out there'' as many scientists currently believe;
reality is everything that exists, all activity that is currently
ongoing in Duration, the mind.

\section{Physics}

The physicist uses an objective form of thinking in order to separate
his or her self as the subject from the object of their study.
Objectification of physical perceptions has led to the artificial
creation of universes, galaxies, stars, planets, humans, organs,
cells, molecules, atoms, and even the fascinating electrons, which can
be thought of as wave objects or particle objects, depending on the
goals of the subject.  In each of these studies, symbolic perceptions
are correlated into causal hypotheses and these causal hypotheses are
grouped into plans that are analogically abstracted into objects, each
different objective abstraction having different subjective
implications.  Note that objects and their implied subjective views
are artificial constructions caused by the goals of an intelligent
mind.  The fact that the position and momentum of an electron can be
considered objectively as probability distributions is fundamentally
derived from counting symbolic perceptions and putting the resulting
numbers into ratios.  The symbolic perceptions in this task are the
closest thing to the real activity of perception.  Once the real
activity of perception has been reflectively symbolized, all that
remains is to manipulate the artificial Spatial arrangements of these
symbols.

\section{Neuroscience}

In the field of AI, there is a tendency to confuse an objective
scientific model of the brain with a model of mind, which is, among
other things, an explanation of the artificiality of the object and
subject distinction.  Neuroscience is fundamentally an objective
physical science.  Neurons are one of the most important objective
discoveries of neuroscience.  Like all objective physical sciences,
the objects discovered are caused by goals.  The objective study of
neuroscience has led to distinctions between the central and
peripheral nervous systems; forebrain, midbrain, and hindbrain
divisions; cortical columns and circuits; nerves and pathways; all
composed of neurons.  The objectification of the nervous system is led
by the goals of the scientist, which are usually medical.  Recently,
the study of neuroscience has combined with psychology and AI, which
allows for studying not just the physical structure, but how this
physical structure relates to a model of mind.  How is reflective
thinking, which requires symbols, done by the brain?  This is an
interesting question, but I want to make the point that in order to
study a question like this one must not confuse a model of mind with
the physical objective model, both being necessary for posing a clear
question relating thinking and the brain.

\section{Neural Networks}

Neural networks are one of the objects created by the field of
neuroscience.  Given the physically objective assumptions of the
field, we have learned of the interactions between many different
objects related to neural networks: the brain is composed of
approximately one hundred billion neurons; chemical gradients guide
neuron growth and attrition; neurons use chemical binding and
electrical potential differences to communicate through axons to
dendrites.  There are many computational models that have been created
to explain different physical phenomena that arise from the
interactions of many neurons.  The study of neural networks is an
objective physical science, like all of neuroscience.  Neural networks
are artificial objective creation of a mind.

\section{Computational Neural Networks}

Computational models of biological neural networks are generally
referred to as \emph{artificial neural networks}.  Note that there is
a danger of confusion here between the initial artificial construction
of the neural network object and subject distinction in the mind of
the scientist and the further artificial construction of a
computational description of the subjective view of this object.
Since both of these objects are artificial by definition, I will use
the term \emph{computational neural network} when I am specifically
referring to the simulation of the mathematical neural network object.

Computational neural network models were originally created in order
to explain the behavior of physical neurons; however, a subset of
these models that were modelled as simple linear and non-linear
algebraic equations have become popular as a subset of control and
optimization model.  These control models been found to be useful for
controlling robot arms, car braking systems, and even music
recommendation systems.

It is important to make a distinction between three ideas here: neural
networks, numerical control models, and thinking.  Thinking is the
activity that builds models of all of these phenomena, including
itself.  Algebraic control systems are symbolic models that assume
that one knows how to count, add, and multiply.  Neural networks are
an object created by physically objective neuroscience.  Because a
neural network is a physical object, it is an analogical abstraction
of plans composed of causal models that are themselves composed of
symbols that refer to physical actualities.  Note that of these three
ideas, thinking is the only one that is actual; the other two are
static models, which are created by the thinking activity.

\section{Brain and Mind}

The brain is one of the key organs that keeps humans alive.  The
nervous system is a key factor in muscular movement of the body;
perceptions, including vision, smell, touch, etc.; speech production;
language understanding; thinking, including counting, making plans,
and doing mathematical calculations.  When I say that the brain is
key, what I mean is that when the brain is damaged, these functions
are lost.  Understanding neural networks, like most neuroscientific
pursuits, is often driven by medical goals.  If a neural network is
understood in terms of how it leads to functionality, such as
thinking, then we can use this physical understanding in correlation
with a model of mind to design preventative, rehabilitative, and
augmentative approaches to neural medicine.

There is a danger at this point of becoming confused and accepting a
model in place of the reality of the ongoing activity in Duration,
which requires neither symbols, objects, nor any other form of
thinking to exist.  A model of mind is a construction that is used to
think about this ongoing existence.  A model of the brain is a
construction that is based on the artificial physically objective
assumptions that divide bodies, organs, cells, etc.  Note that a model
of mind does not necessarily make any objective assumptions; in this
sense, a model of mind can model the exploration and creative activity
that leads to a model of the brain.  A model of the brain is a
physically objective model, and does not lead to models of thinking
because of the assumptions of purely physical objective focus.
However, regardless of focus, both examples of modelling must keep the
reality of the ongoing activity as the fundamental referent for any
symbolized perceptions or causal models.  Some models may be
derivatives of others, but all models are artificial.

