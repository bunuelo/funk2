%************************************************
\chapter{The Computational Substrate}
\label{chapter:the_computational_substrate}
%************************************************

As my theory of mind fundamentally references the ongoing activities
in Duration, so too will my implementation need to actually be a
computational implementation of these activities.

In order to allow for artificially modelling the dynamic reality of
the mind, I have implemented an operating system that allows for
concurrent and parallel bytecode machines to be used for running
simulations of activities.  My implementation includes a Lisp-like
programming language for describing both serial and parallel processes
in a highly configurable macro language.



\section{Leftovers...}

\section{Data Reflection}



\section{The von Neumann Model}

\cite{von_neumann:1945} describes a model of computation that
introduces a concept called \emph{instructions} to the digital
abstraction.  Instructions are a predefined set of arrangements of
symbols that, with other data, determine the future arrangements of
symbols in memory.  Given the von Neumann model, the programming of
each new simulation can be done symbolically rather than by rewiring
the hardware every time.  The von Neumann model is a great technical
achievement that eases the manipulation of simulations, but
fundamentally, the same limitations of digital reflection exist for
the von Neumann model as exist for all computers.  This is because the
von Neumann model still assumes the static unchanging discrete
activity of the combinational device, which causes the transition from
the past to the future.  In the von Neumann model, the combinational
device is more complicated and allows the programmer to more easily
think more abstractly about programs as data.

\section{Instructional Reflection}

The von Neumann model introduces instruction sets that allow programs
to be stored in memory along with other data.

\section{Meaninglessness of the Digital Abstraction}



Because of the inability of a model based on the digital abstraction
to refer to the transition from the past to the future, a
computational model

\section{Leftovers...}

\section{Simulation of a Theory}

Creating a simulation of a theory of mind is useful for a variety of
reasons.  The primary use of simulation is to explore the mechanical
implications of different mechanical assumptions.

\section{Comparing Two Simulations}

Two simulations of two different theories is useful in finding
correlations between these theories.  In order for these correlations
to be meaningful, they must be contained within a more universal
theory that provides a reference to a universal reality.  The idea of
two realities is a contradiction in terms.


\section{Model-6}

My theory does not include Minsky's ``self-reflective'' or
``self-conscious'' layers.  I see objective models of self are
required for what Minsky discusses as self-reflective thinking,
including models of personality recognition and planning.  My model
does not yet have the capability to create or use subjective views of
objects, which I see as necessary not only for self-reflective
thinking, but also Minsky's concept of ``self-conscious'' thinking.
While singly recursive self-reflective statements, like ``Suzy wants
ice-cream'', require learning object and subject relationships, I see
doubly recursive descriptions of personality as necessary for Minsky's
conception of self-conscious thinking, e.g. ``Suzy wants Bob to want
ice-cream.''

