%************************************************
\chapter{The Focus}
\label{chapter:the_focus}
%************************************************

In this dissertation I will focus on a description of thinking that
reflectively learns to accomplish goals.  Reflectively learning to
accomplish goals includes not only accomplishing physical goals but
also an arbitrary number of layers of distinct classes of causal
knowledge goals.  I will describe my implementation, which is a
layered cognitive architecture, including a physical layer and two
layers of reflective thinking.  I will demonstrate learning from
physical feedback through the two reflective layers, updating two
necessarily distinct classes of causal knowledge.  In conclusion, I
will describe an inductive method as theoretical grounding for a
future implementation that includes an arbitrary number of optimizing
layers of thinking.

I will not attempt to give a complete representation for a general
problem domain nor a complete description of a general problem solver.
My work here leaves the unmodellable aspects of reality as known to be
just that.  Thinking uses models as useful tools with the clear
understanding that these models are not fundamentally real.  Thus, my
dissertation will focus on positive qualities of being, perceiving,
acting, reflectively creating causal models and plans, and
reflectively debugging multiple layers of causal models when failures
result from their use.

I will give clear examples in terms of a working model implementation,
aware of the explicit distinction between a model and reality.  I have
categorized my focus only as an example of the computational
description.  Further, and critically important for the advancement of
the field of AI, when describing reflective thinking, it is imperative
to have an explicit awareness of thinking being limited to
manipulating the artificial.  I will therefore refer to an AI model as
simply an ``AI'' with the shared understanding that I am referring to
a model.

\section{Meaningful Description}

A meaningful description references reality.  Words and language have
meaning only in this way.  The further ability of words to also
reference other words allows the additional possibility of creating a
referential circularity in language.  An illusion of a primitive
meaning is possible in such a cycle because all of the words have a
referent; however, because all references within such a closed cycle
appear to be exclusively kept within language and do not seem any
longer to refer to anything real, there is a danger of there being no
meaning other than circularity to such a purely artificial construct.
Again, symbols are tools for modelling and are useful only if the
tautology inherent in the circularity is not forgotten as being
artificial.  Thus, I avoid strictly presenting such illusions of
meaning as primitive in this dissertation.  To allow for meaningful
descriptions, I will first outline my conception of reality, to which
my description ultimately must refer.  A description of reflective
thinking referring to an \emph{a~priori} concept that has not been
derived in terms of symbols allows for something to remain outside of
language as the real referent.  This last step avoids the creation of
a cyclical, self-referential tautology as only an illusion of meaning
in the model.  In the next section I will describe the necessary
conception of reality to which my description of reflective thinking
will refer.

\section{Qualities of Duration}

\cite{bergson:1910}, in his seminal \underline{Time and Free Will},
discusses reality as heterogeneous qualities of Duration.  Qualities
of Duration will serve as my conception of reality for my description
of reflective thinking.  Qualities of Duration are the dynamic ongoing
activity, heterogeneous qualities that are neither distinct nor
separate.  There is a trick, a sort of slight of hand, required here
in describing Bergson's conception of the qualities of Duration
because these qualities exist independently of symbolization.  For
example, when I look at a tree, I may say that the tree has green
leaves and a brown trunk; the symbols, green and brown, refer to the
qualities of Duration, which I am really seeing.  When I focus
carefully on the reality of my situation, I see that the leaves and
trunk could be described by a greater combination of various greens,
yellows, grays, browns and reds.  Symbols limit perception to a
discrete sort of projective presupposition.  Therefore, as I introduce
more symbolic references in my increasingly limited description of
reality, a growing danger of losing focus on something real emerges,
creating a cycle of symbols seemingly referring exclusively to
symbols.  For example, the motivation to tangentially define more
clearly what I mean by the symbol green or brown by using more symbols
results in a circularity from which there is no exit, and therefore,
within which I will never again be able to see or talk about any real
tree as the referent.  In order to avoid the danger of purely cyclic
symbolic reference in my model, I will continue to relate the parts of
my model to real reflective thinking, which exists as qualities of
Duration.

\section{Consciousness, Awareness, and Experience}

There are many terms that are used in the literature when referring to
the concept of reality.  Each of these terms has a different
associated collection of necessary absurdities.  For example, the term
consciousness begs the question \emph{of what} one is conscious.
There is an implied subject and object relationship when using this
transitive term, which is not meant when the term is used as an
isolated absolute reference to reality.  The terms awareness and
experience are used similarly and have the same necessary absurdities
as the term consciousness when they are considered as absolutely
singular references to reality.  Each of these terms have different
meaningful uses when they are not used in this absolutely singular
sense.  For example, in this dissertation an awareness of the
potential confusions these terms introduce when they are used
singularly in abstract isolation as reference to reality will be
maintained.  Thus, I will not argue that these conceptions of reality
are incorrect, but in order to avoid confusion, I will never use these
terms in their abstracted sense.  Instead, I will conflate all of
these uses of these terms as all being the same as my conception of
reality, the ongoing activity that my model accepts, symbolizes, and
thinks about as given.

\section{Actual Reflection}

The qualities of Duration are the given dynamic ongoing activity.
These qualities are the inseparable actions that are currently
available to reflection.  Reflection is the ability to symbolize this
ongoing activity.  For example, the activity of perceiving a table can
be symbolized as ``table''.  Similarly, the activity of walking, can
be symbolized as ``walking''.  Note that my model of reflective
thinking assumes that there is ongoing activity before any symbols
exist for the actions of thinking to manipulate.  Thinking is an
auxiliary activity that manipulates artificial symbolic constructions
that refer to the primary ongoing activity of existence.  Thus, the
activity of perceiving a table requires neither that this activity be
reflectively symbolized nor that a secondary thinking activity
manipulates the resulting symbols.  I will use the phrase \emph{actual
  reflection} to refer to the activity of symbolizing ongoing
activity.

\section{Order of Space}

Qualities of Duration may be symbolized in a homogeneous medium
without quality.  Bergson calls this medium Space, allowing symbolic
references to qualities of Duration to be related.  Of major
importance is the fact that one does not derive the other, but quality
and order coexist with neither Duration being a derivative of Space
nor Space containing the fundamental qualities of Duration.
Obviously, Bergson has gone through painstaking efforts to leave room
in his description for reality as some independent, non-contingent
quality not derived from either Space or Duration.  He makes the
necessary concession that a description of something real is only
meaningful in that it leaves room for reality to exist independently
of its own symbolization.  This work is meant to be similarly
respectful in the description of reflectively learning to accomplish
goals.

\section{The Artificiality of Duration and Space}
\label{section:the_artificiality_of_duration_and_space}

Duration and Space are the fundamental duality that forms the basis of
my model of reality.  Duration and Space exist, pre-reflectively,
prior to symbols being created from the qualities of Duration and
prior to these symbolic references being related in Space.  Note the
fundamental contradiction of intentionally describing reality only in
artificial terms.  For example, if Space has no quality and is not
itself composed of qualities of Duration, to what is the symbol
``Space'' referring?  Symbol ``Space'' refers only to the logical
potential for an absence of quality, the potential for itself as not
really existing; thus, it is the logical necessity of the negative
implication of positive existence, to which no real alternative can be
assigned.  Again, my model will explicitly maintain an awareness of
this fundamental and irresolvable contradiction.  Thus, I begin my
description in the middle, the \emph{in medias res} of an underived
time and space, maintaining an explicit awareness that descriptions
are only tools for reflection.  Duration and Space are not
placeholders for reality; they are useful, intermediate fictions that
I use here to describe an ongoing reality, the activity of being.

\section{Spatial Reflection}

The qualities of Duration are the real activity of being in the
present, actively dynamic and ongoing.  Symbols are artificial static
references to the dynamic present.  Actions are the qualities of
Duration, whether or not they are symbolized.  There is a fundamental
distinction here between the dynamic and the static.  The ongoing
dynamic is actual, while the arrested static is artificial.  In this
way, spatial relationships between symbols can actually exist in
reality as an actively maintained artificial construction.  The
qualities of the ongoing activities that hold symbols in a specific
Spatial arrangement can be symbolically reified.  I will refer to the
activity of symbolizing the qualities of a Spatial arrangement of
symbols as \emph{Spatial reflection} or \emph{reification}.  Spatial
reflection is a form of actual reflection.

\section{The Physical Layer}

My model includes three layers of ongoing dynamic activity.  I will
use the phrase \emph{physical layer} to refer to the pre-reflective
layer of activities that exists prior to any subsequent reflective
thinking layers necessarily existing.  The physical layer is the only
pre-reflective layer of activity in my model.  Therefore, a clear line
is drawn between those activities that symbolize and Spatially arrange
as being circumscribed and not included in what I will refer to as
physical activity.  Note that in most AI models, there is a
distinction between the mind and the body or the environment.  In my
model, all activities that exist are part of the mind.  The real
mental activities that I am referring to by my model are not an object
that is viewed subjectively.  Critical to an understanding of the
goals of objective science is that the creation of object and subject
distinctions is part of the willful activities of the mind; because
this understanding is so critical to an advancement of scientific
understanding, I will clarify the mental derivation of objective
science in \autoref{chapter:education_and_science}.  As the creation
of object and subject relationships is a relatively advanced mental
activity and not key to my thesis, I will discuss this as future
research in \autoref{chapter:future}, which would be a promising place
to extend my model to include ideas like physical bodies and other
subjective and objective environmental perspectives.

\section{Orders of Reflective Layers}

In addition to the pre-reflective physical layer of activity, my model
includes two additional layers of reflective thinking activity that
use two different classes of causal models in order to accomplish two
different classes of goals.  In my conclusion, I will describe how the
implications of my model lead to an arbitrary number of subsequent
layers of reflection.  I will use the phrase \emph{first-order
  reflective layer} to refer to the reflective layer of thinking that
creates symbols and Spatial arrangements that refer to physical
activities.  I will refer to reflectively symbolizing physical
activities as \emph{physical reflection}.  Physical reflection is a
form of actual reflection and is one of the primary reflective
thinking activities; remember that physical activities are
pre-reflective, so physical reflection is not itself a physical
activity.

The first-order reflective layer does not create symbols or Spatial
arrangements that refer to its own activities with one exception:
Spatial reflection.  The first-order reflective layer has the
\emph{a~priori} ability to reify its own actively maintained Spatial
relationships between symbols that refer to physical activities.  The
lack of the ability for the first-order reflective layer to refer to
its own symbolically creative or Spatially manipulative activities
clarifies my model by keeping clear distinctions between each class of
causal knowledge in each subsequent reflective layer of thinking.
Before describing distinctions between classes of causal models, I
will spend the next few sections describing the composition of
temporal transitions and causal hypotheses.

Reflection over the activity of the first-order reflective layer
creates a second layer of reflection and thinking.  Clearly deriving
and describing this second layer of reflective thinking, which implies
an arbitrary number of layers, is the focus of this dissertation.
\cite{minsky:2006} describes a six-layer reflective model of thinking
about thinking.  The first three layers of my model and
implementation, including these two layers of reflective thinking, can
be compared to the first four layers of Minsky's model.  There are
many close parallels between my model and Minsky's model, and I will
describe these as implementation parallels in
\autoref{part:the_implementation}.

\section{Intensities of Qualities}
\label{section:intensities_of_qualities}

\cite{bergson:1910} makes a key distinction between comparisons of
``extensive'' quantities and comparisons of ``intensive'' quantities.
Extensive quantities are Spatial comparisons.  Extensive quantities
are between two symbolic references that share a containment
relationship in terms of their referents: one symbolic referent is
contained by another symbolic referent in Space.  For example, one
body may be said to be larger than another in terms of its Spatial
extent.  Intensive quantities are more subtle and an example is in
terms of pain, which can be said to be more or less intense.
Intensive quantities are similar to neither numerically nor Spatially
extensive containment relationships but, instead, refer to different
fundamental qualities of Duration.  For example, a pain that has no
intensity is not a pain at all; a pain that has a mild intensity may
be equivalently referred to as an irritation or an itch; a pain of
high intensity may be equivalently referred to as a sharp pain or a
throbbing pain.  The point is that these intensities of quality are
actually very different from extensive quantities in that they are not
greater than or less than one another in the sense of a containment
relationship.  For example, a very intense pain, such as a shooting
pain or a throbbing pain, does not in any sense contain a lesser mild
pain, such as an irritation or an itch.  This same pattern exists with
detailed descriptions of other qualities that are often thought of as
having comparable intensities: heat and cold; light; pressure; sound;
pitch; aesthetic feelings, such as grace and beauty in music, poetry
and art; emotions, such as rage and fear; moral feelings, such as
pity; affective sensations, including pleasure, pain and disgust.
Each of these examples points out that the apparent relationships of
greater than or less than with respect to intensity are actually
artificial Spatial arrangements of different fundamental qualities of
Duration.

\section{Number}
\label{section:number}

\cite{bergson:1910} shows the derivation of all numerical
representations as requiring an accompanying extension in Space.  For
example, in referring to the individual sheep in a flock of sheep, he
explains counting:

% pg. 75-77
\begin{quote}
Number may be defined in general as a collection of units, or,
speaking more exactly, as the synthesis of the one and the
many\ldots

No doubt we can count the sheep in a flock and say that there are
fifty, although they are all different from one another and are easily
recognized by the shepherd: but the reason is that we agree in that
case to neglect their individual differences and to take into account
only what they have in common.  On the other hand, as soon as we fix
our attention on the particular features of objects or individuals, we
can of course make an enumeration of them, but not a total\ldots

Hence we may conclude that the idea of number implies the simple
intuition of a multiplicity of parts or units, which are absolutely
alike.

And yet they must be somehow distinct from one another, since
otherwise they would merge into a single unit.  Let us assume that all
the sheep in the flock are identical; they differ at least by the
position which they occupy in space, otherwise they would not form a
flock.
\end{quote}

The last point here illustrates the idea that numbers are always
derived from symbols situated in Space.  Therefore, a model for
thinking based directly on the symbols required cannot be more
complicated than a model based on a numerical representation.  A
number is, after all, the reified constructive result of a Spatial
organization of symbolic references to reality, specific real sheep.
The explicit implication is, therefore, that, in building an AI that
is capable of reflective thought, model building and the use of
numerical representations must be explicitly available for inspection
by the AI itself.  Here, I employ Occam's razor to simplify the loop
of reflective representation in order to ease building the first
proof-of-concept examples of AIs capable of reflective thinking.

\section{Time as Space}

Symbolic references can be ordered spatially.  These symbolic
relationships can be reified and treated symbolically by further
ordering spatial relationships.  Thus, Bergson explains time as a form
of Space:

% pg. 98
\begin{quote}
Now, if space is to be defined as the homogeneous, it seems that
inversely every homogeneous and unbounded medium will be space.  For,
homogeneity here consisting in the absence of every quality, it is
hard to see how two forms of the homogeneous could be distinguished
from one another.  Nevertheless it is generally agreed to regard time
as an unbounded medium, different from space but homogeneous like the
latter: the homogeneous is thus supposed to take two forms, according
as its contents co-exist or follow one another.  It is true that, when
we make time a homogeneous medium in which conscious states unfold
themselves, we take it to be given all at once, which amounts to
saying that we abstract it from duration.  This simple consideration
ought to warn us that we are thus unwittingly falling back upon space,
and really giving up time.
\end{quote}

Time exists only as an artificial creation, an arrangement of symbols
in Space.  Understanding time to be artificial is important because
this allows the real heterogeneous qualities of Duration and a real
homogeneous unqualified Space to remain, respectfully, neither
distinct nor separate.  By prohibiting this separation, and, thereby,
their independent existences, I avoid defining the real qualities of
Duration in terms of artificial symbols, avoiding the creation of a
self-referential cycle, thus sidestepping the potential illusion of
meaning by being clear in our understanding of time used intentionally
as an artificial tool of thought.

In my model, symbols are put into an artificial Spatial arrangement
that I refer to as time.  The actions that create symbols and put
these symbols into Space are basic parts of my model of reflective
thinking.

\section{Temporal Reflection}

Spatial arrangements of symbols can be used to represent a past
through memory and subsequently project a ``future'', through
imaginative action, thus constructing a conception of time.  The past
and the future, therefore, remain artificial constructions that are
actively maintained as Spatial relationships in the present.  I will
refer to the reification of Spatial relationships that order the past
and the future as \emph{temporal reflection}.  Sequential time is
created through temporal reflection.  Temporal reflection is a form of
Spatial reflection.  Like Spatial reflection, temporal reflection is
also a form of actual reflection, symbolizing the active maintenance
of a temporal Spatial relationship between symbolic references.

\section{Simultaneity and Transition}

Correlation requires two symbols to be related in Space.  As symbols
are correlated in the Space called time, these temporal correlations
are referred to as either simultaneities or transitions.
Simultaneities represent two symbols that are both actively symbolized
in Duration.  Transitions represent the change from the past to the
future.  Transitions can be extended in time in order to create
counterfactual temporal extensions, called inferences.  Inferences of
the past and the future can be created by matching and repeating these
transitions.  A transition implies the removal of one symbol and the
addition of another as a progression is made through a temporal
sequence.  If correlations are counted and compared in ratios, then
these are called probabilistic correlations, and the resulting
inference would be a probabilistic inference.

\section{Resources}

Activities in Duration can be referred to symbolically by the
reflective thinking layers.  While all activities in all layers are
generally willful qualities of Duration, these activities can be
symbolized as potentially actionable parts of plans by the reflective
thinking layers.  I will refer to a symbolic reference to activities
that can be put into plans as a \emph{resource}.  Resources exist in
the thinking layers as symbolic references to activities in the layers
below.

\section{Activation and Suppression}

In my model, I've included a logical idea that I refer to as
\emph{suppression}.  Suppression is a symbolic relationship to a
resource that can be put into plans.  The idea of suppression is a
subtle point with respect to the qualities of Duration.  The basic
problem is that the qualities of Duration are the willful activities
that exist without thinking necessarily existing.  Further, the
activities in Duration cannot be inactive, by definition; a symbolic
reference to something inactive would imply a symbol that refers to
something that does not exist, a contradiction.  Therefore, the
logical idea of suppression is an artificial tool of thought, part of
the thinking layers, separate from the physical activities entirely.
The subtle point here is that suppression does not disable the
potential for willful activities.  In my model, suppression is a
logical block for a thinking activity that I will refer to as
\emph{activation}.  Activation and suppression refer to types of
Spatial relationships that are maintained between symbolic resources.
Therefore, it would not be correct to say that physical activities
have been activated or suppressed, but alternatively, it would be
correct to say that a resource is in a Spatial relationship that has
activated or suppressed qualities.  The activation and suppression of
resources occurs in the sense of actively creating Spatial qualities.

Therefore, in my model, a resource can be both either activated or
suppressed, which does not, by itself, imply a resultant activity in
the layer below; logically, if a resource is activated and is not
otherwise suppressed, then the resource is considered logically to be
activated; however, when a resource is both activated and suppressed
simultaneously, this is a logical failure that is cause for a plan to
halt execution.  Before discussing potential responses to plans
failing in this way, let me first discuss the basics of the planning
process.

\section{Cause and Effect}

Two symbols correlated in time are not enough to compose a causal
relationship.  Two symbols correlated in time are simply a transition
from the past to the future.  A causal relationship supposes an
additional component, a necessary connective symbolic reference to the
activities that are ongoing during the transition.  Thus, the effect
of the causal component is the transition.  A causal relationship,
therefore, has three parts: the symbolized qualities of Duration
active in the present, the symbolized qualities of Duration in the
past, and the symbolized qualities of Duration in the future.  I will
sometimes refer to these three parts of the causal relationship more
succinctly as (1) the cause, (2) the necessity, and (3) the result.
When causal models are used for planning, the symbolic reference that
is the cause is referred to as a resource.

\section{Goal and Failure Activities}

It is important to understand that, in my model, goals and failures
are not derivatives of symbolized perceptions.  Fundamentally,
symbolized goals and failures refer to the activities in Duration that
give direction to the activities of thinking.  Goals and failures can
thus be arranged in Spatial orders, according to preferential
qualities.

Bergson refers to activities in Duration as generally \emph{willful}.
In my model, a goal is simply a reference to these activities that
symbolically either exist or do not exist in Duration.  Qualities of
Duration can be willfully reflectively symbolized as goals or
failures.  A symbolized goal is a reference to activities that are to
be sought, while a symbolized failure is a reference to activities
that are to be avoided.

The first-order reflective layer willfully symbolizes goals and
failures and this creation of a symbolic reference to goal or failure
activities becomes the reason for planning and acting toward or away
from the associated perceptions.  Note that because symbolic goals,
failures, perceptions, and resources in my model are artificial and do
not derive from one another, a process of refining symbolic references
to perceptions and resources can be undertaken in the pursuit of
creating more accurate causal models that predict the activities in
Duration that symbolic goals and failures reference.

\section{Causal Hypothesis}

When goal or failure activities are symbolized, causal hypotheses are
created for predicting the goal or failure activities in the future.
For example, if there are ongoing activities happening simultaneously
with the symbolization of the goal or failure, these can be
hypothesized as causes of the future symbolized goal or failure.
Remember that a causal model has three parts: present cause, past, and
future.  In this case, the goal or failure is placed in a future
context, the symbolized current activity is in the present, and there
must also be a past symbol to fill the last slot in the model.

\section{Limitations of Logical Goals and Failures}

Logical approaches to AI have seen the value in considering goals and
failures to be relative arrangements of perceptual symbols.  These
sorts of symbolic relationships in my model are thought of as
occurring simultaneously with, before or after goal or failure
activities.  Artificial constructions that are actively maintained in
Spatial arrangement can be reified as symbolic perceptions, but
fundamentally the goal or failure does not refer to an artificial
construction, despite the possible correlation of goals and failures
with such constructions.  In this way, my model gains the ability to
think about and refine symbolization itself with respect to goals that
are not fundamentally in terms of artificial constructions.
Artificial constructions are useful tools for accomplishing goal
activities; however, reflectively, all artificial constructions,
including symbols, are caused by the AI itself and the responsible
activities can change.  In other words, in purely logical approaches
to AI, symbols are not understood to be artificial and, thus, cannot
be reflectively debugged when they are wrong.  In order to allow these
logical approaches to adapt, I see no alternative but to fundamentally
reinvent all logical approaches to problem solving in reflective terms
that acknowledge a dynamic reality and the static artificiality of
symbols, allowing the potential for debugging meaningful and useful
symbolic constructions.  AI systems must use symbols in full awareness
of their artificial construction as reference to an real unstated
existence, \emph{which can be refined}.

\section{Physical Goals and Failures}

The primary class of goals and failures that can be symbolized is
called the physical goal or failure.  Physical goals and failures
refer to activities in Duration that are the cause for refinement and
initial creation of symbolic perceptions and resources.  These initial
symbolic perceptions and resources are thus physical as well.  The
primary and most fundamental class of causal hypothesis is the
physical causal hypothesis, composed of physical perceptions,
resources, and goals.  All of this initial knowledge stems from the
most fundamental class of goal, the physical goal.

\section{Reflective Classes of Causal Models}

The first-order reflective layer creates causal models from physical
perceptions, resources, goals, and failures.  I've described
previously how goals can cause reflective thinking to create causal
hypotheses.  Another way to state this more succinctly is as follows:
\emph{goals and failures are the cause of causal hypotheses}.  Note
two different meanings of cause in the previous sentence.  In the
latter case, I'm referring to causal models relating physical
perceptions and resources, and in the former case, I'm referring to
the first-order reflective activity that causes the creation of the
physical model.  Thus, the creation of the physical model is a
transition from nothing to existence.

Reflecting over the activities of first-order reflection in Duration
allows my model to represent the transitions caused by first-order
reflective activities.  These transitions are knowledge level changes.
By using this reflective technique, a new class of causal model is
created, categorically different from the physical causal models that
the first-order reflective layer manipulates.  When the activities of
the first-order reflective layer are reflectively symbolized, my model
can then be motivated to learn to accomplish or avoid knowledge level
goals and failures, using knowledge level causal hypotheses.

\section{Probabilistic Causal Models}

Probabilistic models are an advanced and important thinking tool.
Building a probabilistic causal model involves counting symbolic
perceptions, resources, goals, and failures.  For example, let us
consider that two causal hypotheses have been created that reference
the same symbolic cause and the same symbolic perception in the past;
let us say that the only difference between these two hypotheses is
that they reference two different symbolic goals in the future.  This
situation gives my model a reason to further distinguish symbolic
representations for perceptions and resources, introducing more
refined symbols for these perceptions and resources in order to lead
to causal models that are more useful for correctly predicting these
two different symbolic goals.  This would be the appropriate course of
action if we wanted to correctly predict the symbolic goals; however,
there is an opportunity here to build a probabilistic model that does
not refine the symbolization of perceptions or resources.  For
example, both of the two causal hypotheses could be considered
together to compose a probabilistic causal hypothesis.  Using this
probabilistic hypothesis, a future inference could be created that
includes both symbolic goals, each with one half of a potential
existence.  Probabilistic causal hypotheses are useful for predicting
the average number of times that a symbolic event will occur.  Note
that probabilistic causal hypotheses require counting and creating
ratios from the more fundamental non-probabilistic causal hypotheses
from which they are constructed.

\section{First-order Reflective Thinking}

After constructing causal hypotheses that predict the necessities and
results of activating resources, these causal models are used to
construct larger structures called plans.  Plans are combinations of
causal models that contain inferences of past necessities and future
results.  For example, one could imagine a sequence of resources that
leads through a counterfactual future sequence of perceptions,
resources, goals, and failures.  In the literature, these
counterfactual constructions are referred to by a number of names,
including: case, explanation, fiction, narrative, and story.  Stories
and narratives often include a social self-reflective representation,
which I see as future work in the development of my model, which I
will discuss in \autoref{chapter:future}.  I will simply use the term
plan to refer to all of these constructions that combine causal models
into counterfactual views of the past and future.

First-order reflective thinking is the ongoing activity in Duration
that is limited to manipulating symbolic references to physical
activities.  First-order reflective thinking symbolizes physical
activities, creates physical causal hypotheses, and uses these
hypotheses to create plans toward goals or away from failures.
Further, first-order reflective thinking activities execute these
plans composed of physical resource causal symbolic references.

\section{$n^\text{th}$-Order Reflective Thinking}

I've previously described three layers of activity in my model: the
physical layer, the first-order reflective thinking layer, and the
second-order reflective thinking layer.  For brevity, and to emphasize
the layered nature of my model, I will sometimes use a superscript
notation ``$\text{reflective}^n$'' to refer to the $n^\text{th}$-order
of a layer in my model.  For example, the $\text{reflective}^1$ layer
will refer to first-order reflective activities, and the
$\text{reflective}^2$ layer will refer to second-order reflective
activities.  Although the physical layer is not a reflective thinking
layer at all, its place as the prior reference for the first-order
reflective thinking activities moves me to inductively extend this
notation to allow referring to the physical layer as the
$\text{reflective}^0$ layer or the \emph{zeroth-order reflective
  layer}.  In this sense, the implementation of my model of reflective
thinking includes activities of reflective order zero through
reflective order two.

\section{Limitations of First-order Reflective Thinking}

The act of creating a symbolic reference to physical activity is not
accessible to the first-order reflective layer; it is critical to
understand this.  A first-order reflective layer action, such as
creating a plan toward a physical goal, is limited to creating and
Spatially arranging symbols that refer to physical activities.  The
first-order reflective layer does not manipulate symbols that refer to
its own activities.

Another critical point to understand is that first-order reflective
thinking does not directly manipulate physical activities.  Neither
symbols nor Space are physically actual; instead, symbolic references
to physical activity and the relationships that order them in Space
are artificial creations and constructions of the first-order
reflective layer.  Thinking is about physical activities, but is
limited to the terms of its own willfully created artificial symbolic
references to the actual physical layer.  In contrast, the physical
layer does not include any of the artificial constructions of
thinking, neither symbols nor Spatial arrangements.  This is because
the physical layer does not include thinking, which is fundamentally a
reflective activity.

\section{Second-order Reflection on Symbolization}

First-order reflective thinking activities create, Spatially arrange,
and otherwise manipulate symbolic references to physical activities.
The activity of symbolization is a primary aspect of every reflective
thinking layer; however, given the logical limitation that reflective
thinking layers do not refer to their own activities with the
exception of active Spatial arrangements, a second-order reflective
thinking layer is necessarily required for creating symbolic
references to the activity of symbolization.  Note that the activity
of reflecting on symbolization appears in all layers above the
first-order reflective thinking layer, but this activity of reflecting
on the activity of symbolization necessarily first appears in the
second-order reflective thinking layer.

\section{Non-existence of Symbolic References}

Reflecting on symbolization is one the primary activities of the
second-order reflective layer, resulting in the primary second-order
symbols that refer to first-order activity.  Having a symbolic
reference to first-order symbolization allows considering this
activity to be symbolized as a symbolic second-order resource.
Considering first-order symbolization as a second-order resource
allows a second-order causal hypothesis to be created with this
resource as the ongoing cause of a transition.  In this case, the
transition is from the ``non-existence'' to the existence of the
first-order symbolic reference to physical activity.  For example, the
first-order reflective layer can create the perceptual symbol
``green'' to refer a current physical activity in Duration.

Note that non-existence is a logical placeholder in the causal model,
which only refers to the logical alternative to willful creation and
existence itself.  I've begun my description with the assumption that
what exists is my reference, what exists does not require symbols to
be created to refer to it or describe it.  However, in reflecting on
and symbolizing the activity of symbolization, I have implicitly
assumed this placeholder, ``non-existence'', in the causal model that
allows second-order reflective thinking about the existence of
first-order symbolic references.  Non-existence refers to the past
slot in the second-order causal model of first-order symbolization.
Causal hypotheses and their symbolic parts are artificial
constructions, so non-existence is a reference to part of an
artificial construction.  Non-existence is a modelling tool for
second-order reflective thinking.  Non-existence does not refer to a
primary quality of Duration, non-existence refers to an artificial
Spatial arrangement of symbols in the second-order reflective layer.

\section{Planning Symbolic Refinement}

There is a very useful opportunity here to extend the causal model
further into the past and predict the creation of new symbols based on
previous symbolized perceptions or resources.  This could lead to
second-order reflective thinking creating plans for the first-order
thinking layer to create new and more refined symbols that may be
useful for either accomplishing physical goals or avoiding physical
failures.  This is a useful and powerful type of second-order
reflective planning.

\section{The Distractive Nature of Non-existence}

Non-existence as a tool of thought is as dangerous as it is powerful.
The danger is that non-existence can be incorrectly considered to be
\emph{the preexisting reality}.  In my model, I have defined reality
to be everything that exists as ongoing in Duration, maintaining an
awareness of the absurdity of describing something that is prior to
symbols.  Said another way, the danger is in forgetting that
non-existence is a reference to the past slot in an artificial causal
construction of at least order two.  This danger of mistaking
non-existence for the prior reality is insidious, distracting, and
tempting because of its promise of an explanation for creation and
existence itself.

For example, first-order reflective thinking can create the perceptual
symbol ``green'' to refer to the ongoing physical activities.  These
ongoing physical activities do not contain symbols and I have been
very careful to not attempt to provide a description of that which
cannot be defined in terms of artificial symbols, the activities
ongoing in Duration, which are neither artificial nor in artificial
terms.  Therefore, the symbol ``green'' is an artificial reference to
these undefined activities, which are the given qualities of Duration,
which do not require symbolization nor thinking to exist.

Now, a second-order reflective thinking activity can create a causal
model of this first-order symbolization, the creation of the symbol
``green''.  This second-order activity implies that before the
creation of the symbol ``green'' there was a non-existence of the
symbol ``green''.  The problem is that the non-existence of the symbol
is actually the result of the creation of the symbol, so we must not
be confused when the second-order causal model places the
non-existence of the symbol in the past slot.  Non-existence of the
symbol is an artificial tool of thought.  ``Active non-existence'' is
just as absurd as ``inactive existence''; actual non-existence is a
tempting but simple contradiction.

\section{Third-order Reflection on Existence}

The second-order reflective layer creates causal models of the
creation of symbolic references.  These second-order artificial
constructions contain the primary references to existence and
non-existence.  Although my implementation does not include a
third-order reflective layer, my model includes the potential for an
arbitrary number of layers.  In order to briefly describe the
potential for third-order reflective thinking, let us consider a
causal model that would be the creation of the third-order reflective
layer.

For example, let the second-order activity of creating a causal
hypothesis be symbolized as a symbolic resource in the third-order
reflective layer.  This symbolic resource could refer to the
second-order creation of a causal hypothesis describing the
first-order creation of a symbolic reference to physical activity.  In
the past slot of this third-order causal model is the symbolic
resource that refers to the first-order creation of the symbol.  In
the future slot of this third-order causal model is a symbolic
resource that refers to the second-order creation of the causal model
that introduces the concept of existence being created out of
non-existence.  Note that the power of third-order reflection allows
the creation of the causal hypothesis that correctly places the
creation of the symbol prior to the simultaneous creation of the
knowledge of the existence and non-existence of the symbol.

\section{``Sub-symbolic Thinking''}

I have previously discussed the danger of having the second-order
reflective ability to create a causal model that predicts the
existence of symbolic perceptions from non-existence.  The danger is
in believing that the non-existence of a symbolic perception
represents some kind of pre-reflective mechanism that produces
symbols.  There is no such thing as a pre-reflective symbol or model.
Reflection creates symbols and models that refer to reality.  It is
important to understand that a causal model that predicts the creation
of symbols is the result of a second-order reflective activity, more
abstract and artificial than the first-order creation of the symbol.
Now, there have been many causal models that have been developed to
describe the creation of perceptual symbols.  Many of these are based
on physically objective assumptions that allow one to dissect brains
and build models of perceptions based on neural networks.  These
neural network models are sometimes said to be ``sub-symbolic''.  It
is important to realize that the idea of a model being sub-symbolic is
a contradiction because all models are built from symbols.  These so
called sub-symbolic models are actually at least at the level of
second-order reflection, which is the first level of reflection that
is able to model the creation of a symbolic perception.  I think that
referring to the types of models that the second-order reflective
thinking layer creates that predict perceptions as sub-symbolic is a
confusing and distracting phrase in the field of AI.  How to think
reflectively about neural networks has not yet been described in the
field of AI because neural networks are very complex numerical models
that require at least a second-order level of reflective thinking, if
they are to be understood as a model of the creation of symbols.

\section{Leftovers...}

\section{Failure in Expected Results}

When a symbolic reference to a physical activity is actualized in the
context of a plan, the causal model predicts expected symbolic
physical perceptions, goals, and failures.

\section{First-order Reflective Goals and Failures}

First-order reflective activity creates, forgets, and manipulates
causal models of physical activities.  Physical goals and failures are
the initial first-order reflective activity that causes the creation
of physical causal models, combining these causal models into larger
structures called plans.  First-order reflective goals and failures
are a fundamentally different class of goal and failure from the
physical class of goal and failure.  While physical goals and failures
serve to preferentially organize symbolic physical perceptions and
resources, first-order reflective goals and failures serve to
preferentially organize first-order reflective plan perceptions and
planning resources.

\section{First-order Reflective Causal Hypotheses}

First-order reflective goals and failures are the reason for the
creation of first-order reflective causal models.



