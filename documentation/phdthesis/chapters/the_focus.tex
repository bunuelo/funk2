%************************************************
\chapter{The Focus}
\label{chapter:the_focus}
%************************************************

In this dissertation I will focus on a description of reasoning that
reflectively learns to accomplish goals.  I will not attempt to give a
complete representation for a general problem domain nor a complete
description of a general problem solver.  My work here leaves the
unmodellable aspects of reality as known to be just that.  Reasoning
uses models as useful tools with the clear understanding that these
models are not fundamentally real.  Thus, my dissertation will focus
on positive qualities of being, perceiving, acting, creating causal
models, deliberating about plans, and reflectively debugging causal
models when failures result from their use.  I will give clear
examples in terms of a working model implementation, aware of the
explicit distinction between a model and reality.  I have categorized
my focus only as an example of the computational description.
Further, and critically important for the advancement of the field of
AI, when describing reflective thinking, it is imperative to have an
explicit awareness of reason being limited to manipulating the
artificial.  I will therefore refer to an AI model as simply an ``AI''
with the shared understanding that I am referring to a model.

\section{Meaningful Description}

A meaningful description references reality.  Words and language have
meaning only in this way.  The further ability of words to also
reference other words allows the additional possibility of creating a
referential circularity in language.  An illusion of a primitive
meaning is possible in such a cycle because all of the words have a
referent; however, because all references within such a cycle are kept
within language and do not refer to anything real, there is no meaning
other than circularity to such a purely artificial construct.  Again,
symbols are tools for modelling and are useful only if the illusion
inherent in the circularity is not forgotten as being artificial.
Thus, I avoid presenting such illusions of meaning as primitive in
this dissertation.  To allow for meaningful descriptions, I will first
outline my conception of reality, to which my description ultimately
must refer.  A description of reflective thinking referring to an
\emph{a~priori} concept that has not been derived in terms of symbols
allows for something to remain outside of language as the real
referent.  This last step avoids the creation of a cyclical,
self-referential illusion of meaning in the model.  In the next
section I will describe the necessary conception of reality to which
my description of reflective thinking will refer.

\section{Qualities of Duration}

\cite{bergson:1910}, in his seminal \underline{Time and Free Will},
discusses reality as heterogeneous qualities of Duration.  Qualities
of Duration will serve as my conception of reality for my description
of reflective thinking.  Qualities of Duration are the dynamic ongoing
activity, heterogeneous qualities that are neither distinct nor
separate.  There is a trick, a sort of slight of hand, required here
in describing Bergson's conception of the qualities of Duration
because these qualities exist independently of symbolization.  For
example, when I look at a tree, I may say that the tree has green
leaves and a brown trunk; the symbols, green and brown, refer to the
qualities of Duration, which I am really seeing.  When I focus
carefully on the reality of my situation, I see that the leaves and
trunk could be described by a greater combination of various greens,
yellows, grays, browns and reds.  Symbols limit perception to a sort
of projective presupposition.  Therefore, as I introduce more symbolic
references in my description of reality, a growing danger of losing
focus on something real emerges, creating a cycle of symbols referring
exclusively to symbols.  For example, the motivation to tangentially
define more clearly what I mean by the symbol green or brown by using
more symbols results in a circularity from which there is no exit, and
therefore, within which I will never again be able to see or talk
about any real tree as the referent.  In order to avoid the danger of
cyclic symbolic reference in my model, I will continue to relate the
parts of my model to real reflective thinking, which exists as
qualities of Duration.

\section{Consciousness, Awareness, and Experience}

There are many terms that are used in the literature when referring to
the concept of reality.  Each of these terms has a different
associated collection of necessary absurdities.  For example, the term
consciousness begs the question \emph{of what} one is conscious.
There is an implied subject and object relationship when using this
term, which is not meant when the term is used as an absolute
reference to reality.  The terms awareness and experience are used
similarly and have the same necessary absurdities as the term
consciousness when they are considered as absolutely singular
references to reality.  Each of these terms have different meaningful
uses when they are not used in this absolutely singular sense.  For
example, in this dissertation an awareness of the potential confusions
these terms introduce when they are used as absolute singular
references for reality will be maintained.  Thus, I will not argue
that these conceptions of reality are incorrect, but in order to avoid
confusion, I will never use these terms in their absolute sense.
Instead, I will conflate all of these uses of these terms as all being
the same as my conception of reality, the ongoing activity that my
model accepts, symbolizes, and reasons about as given.

\section{Actual Reflection}

The qualities of Duration are an active dynamic ongoing activity.
These qualities are the inseparable actions that are currently
available to reflection.  Reflection is the ability to symbolize this
ongoing activity.  For example, the activity of perceiving a table can
be symbolized as ``table''.  Similarly, the activity of walking, can
be symbolized as ``walking''.  Note that my model of reflective
reasoning assumes that there is ongoing activity before any symbols
exist for the actions of reason to manipulate.  Reason is an auxiliary
activity that manipulates artificial symbolic constructions that refer
to the primary ongoing activity of existence.  Thus, the activity of
perceiving a table requires neither that this activity be reflectively
symbolized nor that a process of reason manipulates the resulting
symbols.  Note that, in the literature, the term reflection is often
used to refer to a given activity that is manipulating relationships
between symbols.  I will use the phrase \emph{actual reflection} to
distinguish the activity of symbolizing ongoing actions from other
uses of the term.

\section{A Layered Reflective Model of Reasoning}

The computational implementation of my model includes two layers of
reasoning that use two different classes of causal models in order to
accomplish two different classes of goals.  In my conclusion, I will
describe how the implications of my model lead to an arbitrary number
of subsequent layers of reflection.  The deliberative layer is the
first reflective layer of reasoning in my model.  The deliberative
layer manipulates symbols that refer to activities in order to
accomplish goals.  Reflection over the activity of deliberation in
order to accomplish deliberative reasoning goals is the focus of this
dissertation.  \cite{minsky:2006} describes a six-layer reflective
model of thinking about thinking.  My model and implementation can be
compared to the first four layers of Minsky's model.  There are many
close parallels between my model and Minsky's model, and my thesis
will describe these parallels in detail.  In order to do this requires
a description of how the relations between symbols come to exist in my
model.

\section{Order of Space}

Qualities of Duration may be symbolized in a homogeneous medium
without quality.  Bergson calls this medium Space, allowing symbolic
references to qualities of Duration to be related.  Of major
importance is the fact that one does not derive the other, but quality
and order coexist with neither Duration being a derivative of Space
nor Space containing the fundamental qualities of Duration.
Obviously, Bergson has gone through painstaking efforts to leave room
in his description for reality as some quality not derived from either
Space or Duration.  He makes the necessary concession that a
description of something real is only meaningful in that it leaves
room for reality to exist independently of its own symbolization.
This work is meant to be similarly respectful in the description of
reflectively learning to accomplish goals.

\section{The Artificiality of Duration and Space}
\label{section:the_artificiality_of_duration_and_space}

Duration and Space are the fundamental duality that forms the basis of
my model of reality.  Duration and Space exist, pre-reflectively,
prior to symbols being created from the qualities of Duration and
prior to these symbolic references being related in Space.  Note the
fundamental contradiction of intentionally describing reality only in
artificial terms.  For example, if Space has no quality and is not
itself composed of qualities of Duration, to what is the symbol
``Space'' referring?  Symbol ``Space'' refers only to the logical
potential for an absence of quality, the potential for itself as not
really existing; thus, it is the logical necessity of the negative
implication of positive existence, to which no real alternative can be
assigned.  Again, my model will explicitly maintain an awareness of
this fundamental contradiction.  I begin my description in the middle,
\emph{in medias res}, maintaining an explicit awareness that
descriptions are only tools for reflection.  Duration and Space are
not placeholders for reality; they are useful fictions that I use here
to describe an ongoing reality, the activity of being.

\section{Intensities of Qualities}
\label{section:intensities_of_qualities}

\cite{bergson:1910} makes a key distinction between comparisons of
``extensive'' quantities and comparisons of ``intensive'' quantities.
Extensive quantities are Spatial comparisons.  Extensive quantities
are between two symbolic references that share a containment
relationship in terms of their referents: one symbolic referent is
contained by another symbolic referent in Space.  For example, one
body may be said to be larger than another in terms of its Spatial
extent.  Intensive quantities are more subtle and an example is in
terms of pain, which can be said to be more or less intense.
Intensive quantities are similar to neither numerically nor Spatially
extensive containment relationships but, instead, refer to different
fundamental qualities of Duration.  For example, a pain that has no
intensity is not a pain at all; a pain that has a mild intensity may
be equivalently referred to as an irritation or an itch; a pain of
high intensity may be equivalently referred to as a sharp pain or a
throbbing pain.  The point is that these intensities of quality are
actually very different from extensive quantities in that they are not
greater than or less than one another in the sense of a containment
relationship.  For example, a very intense pain, such as a shooting
pain or a throbbing pain, does not in any sense contain a lesser mild
pain, such as an irritation or an itch.  This same pattern exists with
detailed descriptions of other qualities that are often thought of as
having comparable intensities: heat and cold; light; pressure; sound;
pitch; aesthetic feelings, such as grace and beauty in music, poetry
and art; emotions, such as rage and fear; moral feelings, such as
pity; affective sensations, including pleasure, pain and disgust.
Each of these examples points out that the apparent relationships of
greater than or less than with respect to intensity are actually
artificial Spatial arrangements of different fundamental qualities of
Duration.

\section{Limitations of Symbolic Thought}

The act of creating a symbol is not accessible to deliberation; it is
critical to understand this.  A deliberative action, such as using
models to create a plan toward a goal, is limited to manipulating
symbols in Space.  Neither symbols nor Space are real; instead,
symbols and the relationships that order them in Space are artificial
creations and constructions of an intelligent mind.  An intelligent
mind deliberates about reality; it does not reason in terms of
reality.

There are some approaches to AI that become confused at this point
because of the seemingly primitive quality of numerical
representations.  Many current approaches consider probabilistic
representations as fundamental to modelling.  In the same way, others
insist that artificial neural networks are more realistic descriptions
of thinking.  Arguments against symbolic representations for thinking
are orthogonal in aim.  I see these approaches to AI as sub-symbolic.
My focus will not be to describe sub-symbolic actions in a
probabilistic or connectionist approach, but rather to describe
deliberative thinking and reflective control in terms of symbols as
modelling tools.

\section{Time as Space}

Symbolic references can be ordered spatially.  These symbolic
relationships can be reified and treated symbolically by further
ordering spatial relationships.  Thus, Bergson explains time as a form
of Space:

% pg. 98
\begin{quote}
Now, if space is to be defined as the homogeneous, it seems that
inversely every homogeneous and unbounded medium will be space.  For,
homogeneity here consisting in the absence of every quality, it is
hard to see how two forms of the homogeneous could be distinguished
from one another.  Nevertheless it is generally agreed to regard time
as an unbounded medium, different from space but homogeneous like the
latter: the homogeneous is thus supposed to take two forms, according
as its contents co-exist or follow one another.  It is true that, when
we make time a homogeneous medium in which conscious states unfold
themselves, we take it to be given all at once, which amounts to
saying that we abstract it from duration.  This simple consideration
ought to warn us that we are thus unwittingly falling back upon space,
and really giving up time.
\end{quote}

Time exists only as an artificial creation, an arrangement of symbols
in Space.  Understanding time to be artificial is important because
this allows the real heterogeneous qualities of Duration and a real
homogeneous unqualified Space to remain, respectfully, neither
distinct nor separate.  By prohibiting this separation, and, thereby,
their independent existences, I avoid defining the real qualities of
Duration in terms of artificial symbols, avoiding the creation of a
self-referential cycle, thus sidestepping the potential illusion of
meaning by being clear in our understanding of time used intentionally
as an artificial tool of thought.

In my model, symbols are put into an artificial Spatial arrangement
that I refer to as time.  The actions that create symbols and put
these symbols into Space are basic parts of my model of reflective
thinking.

\section{Temporal Reflection}

The qualities of Duration are the real activity of being in the
present, actively dynamic and ongoing.  Symbols are artificial static
references to the dynamic present.  Spatial arrangements of symbols
can be used to represent a past through memory and subsequently
project a ``future'', through imaginative action, thus constructing a
conception of time.  The past and the future, therefore, remain
artificial constructions that are actively maintained as Spatial
relationships in the present.  Actions are the qualities of Duration,
whether or not they are symbolized.  There is a fundamental
distinction here between the dynamic and the static.  The ongoing
dynamic is actual, while the arrested static is artificial.  In this
way, time, including the past and the future, can actually exist in
reality as an artificial construction.  I will use the phrase
\emph{temporal reflection} to distinguish the deliberative use of the
artificial construction called time from \emph{actual reflection},
which creates symbolic references to ongoing activities in Duration.

\section{Simultaneity and Transition}

Correlation requires two symbols to be related in Space.  As symbols
are correlated in the Space called time, these temporal correlations
are referred to as either simultaneities or transitions.
Simultaneities represent two symbols that are both actively symbolized
in Duration.  Transitions represent the change from the past to the
future.  Transitions can be extended in time in order to create
counterfactual temporal extensions, called inferences.  Inferences of
the past and the future can be created by matching and repeating these
transitions.  A transition implies the removal of one symbol and the
addition of another as a progression is made through a temporal
sequence.  If correlations are counted and compared in ratios, then
these are called probabilistic correlations, and the resulting
inference would be a probabilistic inference.

\section{Cause and Effect}

Two symbols correlated in time are not enough to compose a causal
relationship.  Two symbols correlated in time are simply a transition
from the past to the future.  A causal relationship supposes an
additional component, a necessary connective symbolic reference to the
action that is ongoing during the transition.  Thus, the effect of the
causal component is the transition.  A causal relationship, therefore,
has three parts: the symbolized qualities of Duration active in the
present, the symbolized qualities of Duration in the past, and the
symbolized qualities of Duration in the future.  I will sometimes
refer to these three parts of the causal relationship more succinctly
as (1) the cause, (2) the necessity, and (3) the result.

\section{Goal Activity}

It is important to understand that, in my model, goals are not a
derivative of symbolized perceptions or symbolized actions.
Fundamentally, symbolized goals refer to the activities in Duration
that cause the activities of deliberation.  Goals can thus be arranged
in Spatial orders, according to preferential qualities.

Qualities of Duration can be reflectively symbolized as goals.  The
symbolization of a goal is the initial deliberative action.
Reflecting on deliberation, goals are the initial cause for the
existence of all causal models.  Symbolized goals in my AI can be
either positive or negative, referring to desirable or undesirable
qualities of the ongoing activities in Duration.

Bergson refers to activities in Duration as generally \emph{willful}.
In my model, a goal is simply a reference to these activities that
symbolically either exist or do not exist in Duration.  This willful
activity can be symbolized by the deliberative layer and this willful
creation of a symbolic reference to goal activities becomes the reason
for acting toward or away from specific associated perceptions or
actions.  Note that because symbolic goals, perceptions, and actions
in my model are artificial and do not derive from one another, a
process of refining symbolic references to perceptions and actions can
be undertaken in the pursuit of creating more accurate causal models
that predict goal activities.

\section{Goal-oriented Causal Hypothesis}

When goal activities are symbolized, causal hypotheses are created for
predicting the goal activities in the future.  For example, if there
are ongoing activities happening simultaneously with the symbolization
of the goal, these can be hypothesized as causes of the future
symbolized goal.  Remember that a causal model has three parts:
present cause, past, and future.  In this case, the goal is placed in
a future context, the symbolized current activity is in the present,
and there must also be a past symbol to fill the last slot in the
model.

\section{Symbolic Relationships as Perceptions}

Logical approaches to AI have seen the value in considering goals to
be relative arrangements of perceptual symbols.  These sorts of
symbolic relationships in my model are thought of as occurring
simultaneously with, before or after goal activities.  Artificial
constructions that are actively maintained in Spatial arrangement can
be reified as symbolic perceptions.  In this way, my model gains the
ability to reason about and refine symbolization itself with respect
to goals that are not in terms of artificial constructions.
Artificial constructions are useful tools for accomplishing goal
activities; however, reflectively, all artificial constructions,
including symbols, are caused by the AI itself and the responsible
activities can willfully change.  In other words, in purely logical
approaches to AI, symbols are not understood to be artificial and,
thus, cannot be debugged when they are wrong.  In order to allow these
logical approaches to adapt, I see no alternative but to fundamentally
reinvent them in reflective terms that acknowledge a dynamic reality
and the static artificiality of symbols, allowing the potential for
debugging meaningful and useful symbolic constructions.

\section{Physical Activity}

My model is layered into different reflective classes of activity.
The primary class of activity is called ``physical''.  Physical
activity is what can be willfully reflectively symbolized without any
reasoning activity, such as deliberation, necessarily existing.
Physical activity does not necessitate the prior existence of
artificial Spatial arrangements of symbols.  Physical activities may
be symbolized, but they do not require symbols to already exist at
all.  Physical activity is the fundamental class of activity in my
model that can be reflectively symbolized.

The primary class of goal that can be symbolized is called a physical
goal.  Physical goals refer to activities in Duration that are the
cause of the initial symbolic representations for perceptions and
actions.  These initial symbolic perceptions and actions are thus
physical as well.  Since all causal hypotheses are created with
respect to goals, the primary and most fundamental class of causal
hypothesis is the physical causal hypothesis, composed of physical
perceptions, actions, and goals.  All of this initial knowledge stems
from the most fundamental class of goal, the physical goal.

\section{Reflective Classes of Causal Models}

The deliberative layer creates causal models from physical
perceptions, actions, and goals.  I've described previously how goals
cause deliberation to create goal-oriented causal hypotheses.  Another
way to state this more succinctly is as follows: \emph{goals are the
  cause of causal hypotheses}.  Note two different meanings of cause
in the previous sentence.  In the latter case, I'm referring to causal
models relating physical perceptions and actions, and in the former
case, I'm referring to the deliberative activity that causes the
creation of the physical model.  Thus, the creation of the physical
model is a transition from nothing to existence.

Reflecting over the activities of deliberation in Duration allows my
model to represent the transitions caused by deliberative actions.
These transitions are knowledge level changes.  By using this
reflective technique, a new class of causal model is created,
categorically different from the physical causal models that the
deliberative layer manipulates.  When the activities of the
deliberative layer are reflectively symbolized, my model can then be
motivated to learn to accomplish knowledge level goals, using
knowledge level causal hypotheses.

\section{Number}
\label{section:number}

\cite{bergson:1910} shows the derivation of all numerical
representations as requiring an accompanying extension in Space.  For
example, in referring to the individual sheep in a flock of sheep, he
explains counting:

% pg. 75-77
\begin{quote}
Number may be defined in general as a collection of units, or,
speaking more exactly, as the synthesis of the one and the
many\ldots

No doubt we can count the sheep in a flock and say that there are
fifty, although they are all different from one another and are easily
recognized by the shepherd: but the reason is that we agree in that
case to neglect their individual differences and to take into account
only what they have in common.  On the other hand, as soon as we fix
our attention on the particular features of objects or individuals, we
can of course make an enumeration of them, but not a total\ldots

Hence we may conclude that the idea of number implies the simple
intuition of a multiplicity of parts or units, which are absolutely
alike.

And yet they must be somehow distinct from one another, since
otherwise they would merge into a single unit.  Let us assume that all
the sheep in the flock are identical; they differ at least by the
position which they occupy in space, otherwise they would not form a
flock.
\end{quote}

The last point here illustrates the idea that numbers are always
derived from symbols situated in Space.  Therefore, a model for
deliberative thinking based directly on the symbols required cannot be
more complicated than a model based on a numerical representation.  A
number is, after all, the reified constructive result of a Spatial
organization of symbolic references to reality, specific real sheep.
The explicit implication is, therefore, that, in building an AI that
is capable of reflective thought, model building and the use of
numerical representations must be explicitly available for inspection
by the AI itself.  Here, I employ Occam's razor to simplify the loop
of reflective representation in order to ease building the first
proof-of-concept examples of AIs capable of reflective thinking.

\section{Probability}

Probabilistic models are an advanced and important reasoning tool.
Building a probabilistic goal-oriented causal model involves counting
symbolic perceptions, actions, and goals.  For example, let us
consider that two causal hypotheses have been created that reference
the same symbolic cause and the same symbolic perception in the past;
let us say that the only difference between these two hypotheses is
that they reference two different symbolic goals in the future.  This
situation gives my model a reason to further distinguish symbolic
representations for perceptions and actions, introducing more refined
symbols for these perceptions and actions in order to lead to causal
models that are more useful for correctly predicting these two
different symbolic goals.  This would be the appropriate course of
action if we wanted to correctly predict the symbolic goals; however,
there is an opportunity here to build a probabilistic model that does
not refine the symbolization of perceptions or actions.  For example,
both of the two goal-oriented causal hypotheses could be considered
together to compose a probabilistic goal-oriented causal hypothesis.
Using this probabilistic hypothesis, a future inference could be
created that includes both symbolic goals, each with one half of a
potential existence.  Probabilistic causal hypotheses are useful for
predicting the average number of times that a symbolic event will
occur.  Note that probabilistic causal hypotheses require counting and
creating ratios from the more fundamental non-probabilistic causal
hypotheses from which they are constructed.

\section{Deliberative Planning}

After constructing goal-oriented hypotheses that predict the
necessities and results of actions, these causal models are used to
construct larger structures called plans.  Plans are combinations of
causal models that contain inferences of past necessities and future
results.  For example, one could imagine a sequence of actions that
leads through a counterfactual future sequence of perceptions,
actions, and goals.  In the literature, these counterfactual
constructions are referred to by a number of names, including: case,
narrative, and story.  Stories and narratives often include a social
self-reflective representation, which I see as future work in the
development of my model, which I will discuss in
Chapter~\ref{chapter:future}.  I will simply use the term plan to
refer to all of these constructions in order to emphasize that goals
are the reason for the initial existence of all of these artificial
constructions that combine causal models into counterfactual views of
the past and future.

\section{Object and Subject}

Causal hypotheses are used in order to predict the future occurrence
of symbolic perceptions and goals.  Deliberatively, plans are
constructed from causal hypotheses consistent with the past,
elaborating the necessities in the past and the results in the future.
Analogies between consistent plans can be abstracted into models of
objects.  For example, a plan that begins and ends with the same
symbolic perception could be considered an example of a ``circular''
object; thus, an analogical plan abstraction would represent a type of
object, that has multiple sides to perceive depending on the subject's
position in the plan.  Circular objects may be an important type of
object to analogically recognize because a circular object allows one
to perform actions while being able to get back to a known perceptual
symbol.  If one is in a known circular object, then one is not lost in
the sense that one can always follow the circle in order to get back
to any perceptual symbol contained within the circular plan object.
My implementation includes tools for performing analogical
abstractions and constructions of plans, but this machinery is not key
to my thesis of explaining causal reflective learning to accomplish
goals.  I include this section in the dissertation to eliminate a
potential confusion that would conflate the concept of symbol with
that of object.  Symbols are the most primitive elements available to
reasoning, while objects are more complicated in that they are
composed of analogical consistencies between plans that are themselves
composed of many symbols.

\section{Deliberative Actions}

Deliberative activity reflects over physical activities and creates
plans




\section{Leftovers...}



