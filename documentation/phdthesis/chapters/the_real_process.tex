%************************************************
\chapter{The Real Process}
\label{ch:the_real_process}
%************************************************

In this dissertation I will focus on a description of the reasoning
process that reflectively learns to accomplish goals.  I will not
attempt to give a complete representation for a general problem domain
nor a complete description of a general problem solver.  My work here
leaves the unmodellable aspects of reality as known to be just that.
Reasoning is a process that uses models as useful tools with the clear
understanding that these models are not fundamentally real.  Thus, my
dissertation will focus on a positive description of the real
processes of being, perceiving, acting, creating causal models,
deliberating about plans, and reflectively debugging causal models
when failures result from their use.  I will give clear examples in
terms of a working model implementation, aware of the clear
distinction between a model and the fact of reality itself.  I have
categorized my focus only as an example of the computational
description.  Further, and critically important for the advancement of
the field of AI, when describing reflective thinking, it is imperative
to have an explicit awareness of reason being limited to manipulating
only the artificial.  I will therefore refer to an AI model as simply
an ``AI'' with the shared understanding that I am referring to a
model.

\section{Meaningful Description}

A meaningful description references something real.  Words and
language have meaning only in this way.  The ability of words to
reference other words allows the possibility of creating a referential
cycle in language.  An illusion of meaning is possible in such a cycle
because all of the words have a referent; however, because all
references within such a cycle are kept within language and do not
refer to anything real, there is no meaning to such a purely
artificial construct.  It is imperative that I avoid presenting such
an illusion of meaning in this dissertation.  To allow for a
meaningful description, I must first outline my conception of reality,
to which my description may refer.  I must ground my description of
reflective thinking in a concept that is not in terms of symbols so
that something outside of language may be the real referent.  This
last step is necessary in allowing me to avoid creating a cyclical
illusion of meaning.  In the next section I will describe my
conception of reality, to which my description of reflective thinking
will refer.

\section{Qualities of Duration}

\cite{bergson:1910}, in his seminal \underline{Time and Free Will},
discusses reality as heterogeneous qualities of Duration.  Qualities
of Duration will serve as my conception of reality for my description
of the real process of reflective thinking.  Qualities of Duration are
the dynamic experience or consciousness of being, heterogeneous
qualities that are neither distinct nor separate.  It is a trick to
describe Bergson's conception of the qualities of Duration because
these qualities exist, independent of their being symbolized.  For
example, when I look at a tree, I may say that the tree has green
leaves and a brown trunk; the symbols, green and brown, refer to the
qualities of Duration, which I am really seeing.  When I focus
carefully on the reality of my situation, I see that the leaves and
trunk could be described by a greater combination of various greens,
yellows, grays, browns and reds.  As I introduce more symbolic
references in my description of reality, there is a growing danger of
losing my focus on describing something real.  The danger is in
creating a meaningless cycle of symbols referring to symbols.  For
example, if I were to get stuck in a tangent, defining more clearly
what I mean by the symbol green or brown by using more symbols, then I
will never again see or talk about reality.  In order to avoid the
danger of cyclic symbolic reference in my model, I will continue to
relate the parts of my model to the real process of reflective
thinking, which exists as qualities of Duration.

\section{Order of Space}

Qualities of Duration may be symbolized in a homogeneous medium
without quality.  Bergson calls this medium Space, allowing symbolic
references to qualities of Duration to be related.  He does not use
one to derive the other, but quality and order coexist with neither
Duration being a derivative of Space nor Space containing the
fundamental qualities of Duration.  He goes through painstaking
efforts to leave room in his description for reality.  He makes the
necessary concession that a description of something real is only
meaningful in that its description leaves room for reality to exist
independently of its own description.  Our work is meant to be
similarly respectful in our description of the real process of
reflectively learning to accomplish goals.

\section{Limitations of Symbolic Thought}

The process of obtaining or creating a symbol is not accessible to
deliberation.  It is critical to understand that a deliberative
process is limited to manipulating symbols in Space.  Neither symbols
nor Space are real; instead, symbols and the relationships that order
them in Space are artificial creations and constructions of an
intelligent mind.  An intelligent mind deliberates about reality; it
does not reason in terms of reality.

There are some approaches to AI that are confused in their insistence
on numerical representations.  Many current popular approaches
consider probabilistic representations as fundamental to the process
of thinking.  Others insist that artificial neural networks are more
realistic descriptions of the real process of thinking.  Many
arguments against symbolic representations for thinking are orthogonal
in aim.  I see these approaches to AI as sub-symbolic.  My focus will
be, not to describe sub-symbolic processes as in probabilistic and
connectionist approaches, but to describe the process of deliberative
thinking and reflective control.

One defense of neural networks is that they can deal with numerical
intensities of stimuli, such as temperature, pressure, or weight.  In
the next section, (Section~\ref{section:intensities_of_qualities}), I
will describe my approach to how a deliberative process deals with
intensities of qualities of Duration.  Further, in order to dismiss
remaining confusion regarding how numerical representations relate to
the deliberative process in my model, the subsequent section
(Section~\ref{section:numerical_representation}) will explicitly
discuss my approach to thinking using numerical representations.
These issues of intensity and numerical representations are not
fundamental to my model but are included for the reader that
traditionally thinks of these issues as fundamental to thinking.  The
following two sections make explicit the relationship between my model
and those models.

\section{Intensities of Qualities}
\label{section:intensities_of_qualities}

\cite{bergson:1910} makes a key distinction between comparisons of
``extensive'' quantities and comparisons of ``intensive'' quantities.
Extensive quantities are Spatial comparisons.  Extensive quantities
are between two symbolic references that share a containment
relationship in terms of their referents: one symbolic referent is
contained by another symbolic referent in Space.  For example, one
body may be said to be larger than another in terms of its Spatial
extent.  Intensive quantities are more subtle and an example is in
terms of pain, which can be said to be more or less intense.  Bergson
makes the point that intensive quantities are similar to neither
numerically nor Spatially extensive containment relationships but,
instead, refer to different fundamental qualities of Duration.  For
example, a pain that has no intensity is not a pain at all; a pain
that has a mild intensity may be equivalently referred to as an
irritation or an itch; a pain of high intensity may be equivalently
referred to as a sharp pain or a throbbing pain.  Bergson argues that
these intensities of quality are actually very different from
extensive quantities in that they are not greater than or less than
one another in the sense of a containment relationship.  For example,
a very intense pain, such as a shooting pain or a throbbing pain, does
not in any sense contain a lesser mild pain, such as an irritation or
an itch.  He shows this same pattern with detailed descriptions of
other qualities that are often thought of as having comparable
intensities: heat and cold; light; pressure; sound; pitch; aesthetic
feelings, such as grace and beauty in music, poetry and art; emotions,
such as rage and fear; moral feelings, such as pity; affective
sensations, including pleasure, pain and disgust.  The point that
Bergson makes in each of these examples is that the apparent
relationships of greater than or less than with respect to intensity
are actually artificial Spatial arrangements of different fundamental
qualities of Duration.

There are some approaches to AI that consider numerical
representations to be primitive to the process of deliberation.  I
will discuss a few of these approaches that consider numerical
quantities as fundamental after my next section where I will first
introduce my understanding of counting and how counting leads to
numerical representations in a deliberate process.

\section{Numerical Representation}
\label{section:numerical_representation}

\cite{bergson:1910} show the derivation of all numerical
representations as requiring an accompanying extension in Space.  For
example, in referring to the individual sheep in a flock of sleep, he
explains the process of counting:

% pg. 75-77
\begin{quote}
Number may be defined in general as a collection of units, or,
speaking more exactly, as the synthesis of the one and the
many\ldots

No doubt we can count the sheep in a flock and say that there are
fifty, although they are all different from one another and are easily
recognized by the shepherd: but the reason is that we agree in that
case to neglect their individual differences and to take into account
only what they have in common.  On the other hand, as soon as we fix
our attention on the particular features of objects or individuals, we
can of course make an enumeration of them, but not a total\ldots

Hence we may conclude that the idea of number implies the simple
intuition of a multiplicity of parts or units, which are absolutely
alike.

And yet they must be somehow distinct from one another, since
otherwise they would merge into a single unit.  Let us assume that all
the sheep in the flock are identical; they differ at least by the
position which they occupy in space, otherwise they would not form a
flock.
\end{quote}

Bergson's last point here illustrates his idea that numbers are always
derived from symbols situated in Space.  Therefore, a model for
deliberative thinking based directly on the symbols required cannot be
more complicated than a model based on a numerical representation.  A
number is, after all, the reified constructive result of a Spatial
organization of symbolic references to reality, specific real sheep in
Bergson's explanation.  The explicit implication is, therefore, that,
in building an AI that is capable of reflective thought, the internal
constructive processes of model building and the internal use of
numerical representations must be explicitly available for inspection
by the AI itself.  Here, I employ Occam's razor to simplify the loop
of reflective representation in order to ease building the first
proof-of-concept examples of AIs capable of reflective thinking.

\section{Time as Space}

Symbolic references can be ordered spatially.  These symbolic
relationships can be reified and treated symbolically in further
ordered spatial relationships.  Bergson explains time as a form of
Space:

% pg. 98
\begin{quote}
Now, if space is to be defined as the homogeneous, it seems that
inversely every homogeneous and unbounded medium will be space.  For,
homogeneity here consisting in the absence of every quality, it is
hard to see how two forms of the homogeneous could be distinguished
from one another.  Nevertheless it is generally agreed to regard time
as an unbounded medium, different from space but homogeneous like the
latter: the homogeneous is thus supposed to take two forms, according
as its contents co-exist or follow one another.  It is true that, when
we make time a homogeneous medium in which conscious states unfold
themselves, we take it to be given all at once, which amounts to
saying that we abstract it from duration.  This simple consideration
ought to warn us that we are thus unwittingly falling back upon space,
and really giving up time.
\end{quote}

Time is an artificial creation, an arrangement of symbols in Space.
Understanding time to be artificial is important because this leaves
the real heterogeneous qualities of Duration and the real homogeneous
Space as, respectfully, neither distinct nor separate.  Here we avoid
defining the real qualities of Duration in terms of artificial
symbols, avoiding the creation of a referential cycle.  In avoiding
this cycle, we sidestep a potential illusion of meaning by being clear
in our understanding of time as an artificial tool of thought.

In my model, symbols are put into an artificial Spatial arrangement
that I refer to as time.  The processes that create symbols and put
these symbols into Space are both basic parts of my model of
reflective thinking.

\section{Cause and Correlation}

Correlation requires two symbols to be related in Space.  As symbols
are correlated in the space called time, these correlations can be
extended in time in order to create counterfactual temporal
extensions, called inferences.  Inferences of the past and the future
can be created by recognizing and using these symbolic correlations in
time.  If these correlations are counted and compared in ratios, then
these are probabilistic correlations, and the resulting inference is a
probabilistic inference.

Having two symbols correlated in Space does not create a causal
relationship.  Even having two symbols correlated in the Space called
time is simply a temporal correlation.  A causal relationship has an
additional level of complexity.



\section{Perceptions and Reflections}



\section{Implicit Processes and Reflective Symbols}



\section{Cause and Effect}

Paramount to an understanding of causal reflective reasoning is an
awareness of the fundamental limitations of symbolic thought.  Without
such an awareness, one is tempted to trace causal models of reality
back to the real cause of the model itself.  

\section{Goals}

Deliberative reasoning is the process of reasoning about how to
accomplish goals.  



\section{Leftovers...}

\section{Awareness of Effort}

Bergson would refer to efforts as fundamental qualities of Duration.
For example,What I am referring to as action is what Bergson would
call effort.  To Bergson, all effort

% pg. 20
\begin{quote}
If there is a phenomenon which seems to be presented immediately to
consciousness under the form of quantity or at least of magnitude, it
is undoubtedly muscular effort.  We picture to out minds a psychic
force imprisoned in the soul like the winds in the cave of Aeolus, and
only waiting for an opportunity to burst forth: our will is supposed
to watch over this force and from time to time to open a passage for
it, regulating the outflow by the effect which it is desired to
produce.  If we consider the matter carefully, we shall see that this
somewhat crude conception of effort plays a large part in our belief
in intensive magnitudes.
\end{quote}

He goes on to argue that an awareness of muscular effort is solely due
to proprioceptive feedback from muscles contracting.  Further, he
describes phantom limb patients that feel that they are expending
effort and he hypothesizes that they must be moving a limb
\emph{somewhere} and they must be getting sensory feedback from it,
even though it is obviously in this case not a physical limb.

What appears symbolically to the deliberative mind about willful
actions is similar in category to the afferent symbolic information
that is traditionally considered perceptual.







