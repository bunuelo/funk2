%*******************************************************
% Abstract
%*******************************************************
%\renewcommand{\abstractname}{Abstract}
\pdfbookmark[1]{Abstract}{Abstract}
\begingroup
\let\clearpage\relax
\let\cleardoublepage\relax
\let\cleardoublepage\relax

\chapter*{Abstract}

There have been two directions of research with the goal of building a
machine that explains intelligent human behavior.  The first approach
is to build a baby-machine that learns from scratch to accomplish
goals through interactions with its environment.  The second approach
is to give the machine an abundance of knowledge that represents
correct behavior.

Each of these solutions has benefits and drawbacks.  The baby-machine
approach is good for dealing with novel problems, but these problems
are necessarily simple because complex problems require a lot of
background knowledge.  The data abundance approach deals well with
complicated problems requiring a lot of background knowledge, but
fails to adapt to changing environments, for which the algorithm has
not already been trained.

We are working on an algorithm that benefits from both of these
approaches by learning from cultural language knowledge, while
reflectively monitoring and recognizing the failures of this knowledge
when it is used in a goal-oriented domain.

Toward this end we have developed a reflective programming language
allowing us the ability to monitor the execution and interactions
between large numbers of complicated lisp-like processes.  Further, we
have developed a cognitive architecture within our language that
provides structures for layering reflective processes, resulting in a
hierarchy of control algorithms that respond to failures in the layers
below.

Finally, we present an example of our cognitive architecture learning
in the context of a social commonsense reasoning domain with parents
that teach children as they attempt to accomplish cooking tasks in a
kitchen.

\endgroup

\vfill
