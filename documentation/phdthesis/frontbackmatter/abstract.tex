%*******************************************************
% Abstract
%*******************************************************
%\renewcommand{\abstractname}{Abstract}
\pdfbookmark[1]{Abstract}{Abstract}
\begingroup
\let\clearpage\relax
\let\cleardoublepage\relax
\let\cleardoublepage\relax

\chapter*{Abstract}
%Between the ages of 1-3 years old, children display primary emotions,
%such as joy, disappointment, and surprise.  These emotional processes
%have been hypothesized to be related to the process of failing or
%succeeding to accomplish a goal.  Around age 4, children begin to
%display emotions that involve the self, such as guilt and shame.  It
%has been hypothesized that these emotions relate to another person's
%evaluation of the child's goals as good or bad.
%
%We approach modelling this developmental process by applying Marvin
%Minsky's theory of the child-imprimer relationship.  According to
%Minsky's theory, at a young age, a human child becomes attached to a
%person that functions as a teacher.  The imprimer could be a parent or
%a caregiver or another person in the child's life, but the function of
%the imprimer is to provide feedback to the child in terms of what
%goals are good or bad for the child to pursue.




Recently, there have been two directions of research with the goal of
building a machine that explains intelligent human behavior.  The
first approach is the machine learning approach and the second is the
pattern recognition approach.  Each of these solutions has benefits
and drawbacks.  The machine learning approach attempts to build a
machine that learns to accomplish goals by learning the effects of its
actions by interacting with its environment.  The pattern recognition
approach is given large amounts of knowledge and finds statistical
regularities within this knowledge in order to generate more
knowledge.  Machine learning is good for dealing with novel problems,
but these problems are necessarily simple because complex problems
require background knowledge.  Pattern recognition deals well with
complicated problems requiring a lot of background knowledge, but
fails to adapt to changing environments, for which the algorithm has
not already been trained.

We are working on an algorithm that combines these two extremes into
an algorithm that inherits cultural language knowledge, while
recognizing the failures of this knowledge through failures and
successes when this knowledge is used.  We develop a definition of the
utility of cultural knowledge in a domain that is grounded in
goal-oriented action that corrects this knowledge by learning in the
context of failure and success.

\begin{quote}
Problem-sovlers must find relevant data.  How does the human mind retrieve what it needs from among so many millions of knowledge iterms?  Different AI systems have attempted to use a variety of different methods for this.  Some assign keywords, attributes, or descriptors to each item and tehn locate data by feature-matching or by using more sophisticated associative data-base methods.  Oythers use graaph-matching or analogical case-based adaptation.  Yet others try to find relevant information by threading their ways through systematic, usually hierarchical classifications of knowledge---sometimes called ``ontologies''.  But, to me, all such ideaas seem deficient because it is not enough to classify items of information simply in terms of the features or structures of those items themselves.  This is because we rarely use a representation in an intentional vaccuum, but we always have goals---and two objectss may seem similar for one purpose but different for another purpose.
\end{quote}




\endgroup

\vfill
