%*******************************************************
% Abstract
%*******************************************************
%\renewcommand{\abstractname}{Abstract}
\pdfbookmark[1]{Abstract}{Abstract}
\begingroup
\let\clearpage\relax
\let\cleardoublepage\relax
\let\cleardoublepage\relax

%\begin{addmargin}[1cm]{-2cm}
\begin{center}
    \begingroup
        \color{Maroon}
        {\Large\textsc{\myTitle}}
        %\myTitle
    \endgroup

    by

    \myName

    \vspace{4mm}

 Submitted to the program in Media Arts and Sciences, School of
 Architecture and Planning in partial fulfullment of the requirements
 for the degree of

    \vspace{2mm}

Doctor of Philosophy in Media Arts and Sciences

at the

Massachusetts Institute of Technology

    \vspace{2mm}

September 2012

\vspace{-4mm}

\end{center}

%\end{addmargin}

\chapter*{Abstract}

A system built on a layered reflective cognitive architecture presents
many novel and difficult software engineering problems.  Some of these
problems can be ameliorated by erecting the system on a substrate that
implicitly supports tracing of results and behavior of the system to
the data and through the procedures that produced those results and
that behavior.  Good traces make the system accountable; it enables
the analysis of success and failure, and thus enhances the ability to
learn from mistakes.

This constructed substrate provides for general parallelism and
concurrency, while supporting the automatic collection of audit trails
for all processes, including the processes that analyze audit trails.
My system natively supports a Lisp-like language.  In such a language,
as in machine language, a program is data that can be easily
manipulated by a program, making it easier for a user or an automatic
procedure to read, edit, and write programs as they are debugged.

Here, I build and demonstrate an example of reflective problem solving
in a block building problem domain.  In my demonstration an AI model
can learn from experience of success or failure.  The AI not only
learns about physical activities but also reflectively learns about
thinking activities, refining and learning the utility of built-in
knowledge.  Procedurally traced memory can be used to assign credit to
those thinking processes that are responsible for the failure,
facilitating learning how to better plan for these types of problems
in the future.

\vspace{4mm}
\noindent
\begin{tabular}{rl}
Thesis Supervisor:&Joseph Paradiso, PhD\\
                  &Associate Professor of Media Arts and Sciences
\end{tabular}

\endgroup

\vfill

