%*******************************************************
% Abstract
%*******************************************************
%\renewcommand{\abstractname}{Abstract}
\pdfbookmark[1]{Abstract}{Abstract}
\begingroup
\let\clearpage\relax
\let\cleardoublepage\relax
\let\cleardoublepage\relax

\chapter*{Abstract}

A system built on a layered reflective cognitive architecture presents
many novel and difficult software engineering problems.  Some of these
problems can be ameliorated by erecting the system on a substrate that
implicitly supports tracing of results and behavior of the system to
the data and through the procedures that produced those results and
that behavior.  Good traces make the system accountable; it enables
the analysis of success and failure, and thus enhances the ability to
learn from mistakes.

I have constructed just such a substrate.  It provides for general
parallelism and concurrency, while supporting the automatic collection
of audit trails for all processes, including the processes that
analyze audit trails.  My system natively supports a Lisp-like
language.  In such a language, as in machine language, a program is
data that can be easily manipulated by a program.  This makes it
easier for a user or an automatic procedure to read, edit, and write
programs as they are debugged.

Here, I build and demonstrate an example cognitive architecture
simulation of life in a rigidbody physical environment.  In my
demonstration multiple agents can learn from experience of success or
failure or by being explicitly taught by other agents, including the
user.  In my demonstration I show how procedurally traced memory can
be used to assign credit to those deliberative processes that are
responsible for the failure, facilitating learning how to better plan
for these types of problems in the future.

\endgroup

\vfill

