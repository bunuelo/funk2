%*******************************************************
% Acknowledgments
%*******************************************************
\pdfbookmark[1]{Acknowledgments}{acknowledgments}

%Don't do anything that isn't play. \\ \medskip
%    --- Joseph Campbell



\begin{flushright}{\slshape    
Don't do anything that isn't play.} \\ \medskip
    --- Joseph Campbell
\end{flushright}



\bigskip

\begingroup
\let\clearpage\relax
\let\cleardoublepage\relax
\let\cleardoublepage\relax
\chapter*{Acknowledgments}

I would first like to thank my memory of Push Singh for being a strong
memory of a heroic friend and advisor.

\vspace{5mm}

\noindent I would like to thank my committee:
Joe Paradiso for unfailing support and faith in my ability to do something well.
Marvin Minsky for consistently and patiently providing an inspiringly critical perspective and a wealth of important problems to solve.
Gerry Sussman for loving to program.
Mike Cox for providing critical direction for navigating the space of the contemporary meta-cognitive computational sciences.

\vspace{5mm}

\noindent I would like to thank my immediate family:
Greg Morgan and Carolyn Spinner for being willing to help me think about anything at any time.
Paul Bergman for teaching me how to program.
Virginia Barasch for painting my finger nails.
Leaf Morgan for playing Archon with me.

\vspace{5mm}

\noindent I'd like to thank those who worked on the project:
Dustin Smith for the physical social commonsense kitchen simulation.
Radu Raduta for the PID control loops for robot balance and movement.
Jon Spaulding for the EEG amplifiers and pasteless electrodes.
Gleb Kuznetsov for the first version of a 3D commonsense kitchen critic-selector layout planner.
Jing Jian for being so good at finding critical bugs in Funk2 and for writing all of the learned-reactive resources for following recipes in the kitchen.
Panupong Pasupat for the error-correcting subgraph isomorphism algorithm and the Winston difference-of-differrence analogy algorithm.
Mika Braginsky for the visualization of mental resources, agencies, and layers.

\vspace{5mm}

\noindent I'd like to thank the following professionals:
Walter Bender,
Henry Lieberman,
Patrick Winston,
Whitman Richards,
Ed Boyden,
Sebastian Seung,
Hugh Herr,
Ted Selker, and
Rebecca Saxe.

\vspace{5mm}

\noindent I'd like to thank the Media Lab commonsense researchers:
Barbara Barry,
Catherine Havasi,
Rob Speer,
Hugo Liu,
Ian Eslick,
Jason Alonso, and
Dustin Smith.

\vspace{5mm}

\noindent I would like to thank the popes of GOAM:
Scotty Vercoe,
Dane Scalise,
Kemp Harris, and
Dustin Smith.

\vspace{5mm}

\noindent I would like to thank the researchers of the Mind Machine Project:
Newton Howard, Forrest Green, Kenneth Arnold, Dustin Smith, Catherine Havasi, Scott Greenwald, Rob Speer, Peter Schmidt-Nielsen, Sue Felshin, David Dalrymple, and Jason Alonso.

\vspace{5mm}

\noindent I would be remiss if I were not to thank, again in some cases, the following individuals,
whom have kept me sane and happy through the PhD process:
Dustin Smith, % for being a rock.
Grant Kristofek, % for loving music, staying true to the blues, and keeping my soul alive by organizing jams.
Mako Hill, % for helping me get Funk2 out into the public open software development communities.
Josh Lifton, % for inspiring me to write my first garbage collector and ultimately my own programming language.
Mat Laibowitz, % for disagreeing with me about almost everything, being the hardest-core engineer, and refusing ``to party''.
Mark Feldmeier, % for being there to ask questions, listen, and talk.
Matt Aldrich, % for jamming, teaching me about metal, learning technically intricate music.
Nan-Wei Gong, % for singing, playing keys, and generally rocking out.
Mike Lapinski, % for being a healthy athlete.
Clio Andris, % for going through the sometimes painful PhD process with me.
Hannah Perner-Wilson, % for discussing the limitations of categorical thinking.
Jeff Lieberman, % for practicing simply being.
David Cranor, and % for organizing thesis crew.
Em\"{o}ke-\'{A}gnes Horv\'{a}t. % for being magical.

\endgroup

