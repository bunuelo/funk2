\pagestyle{empty}

\hfill

\vfill


\pdfbookmark[0]{Colophon}{colophon}
\section*{Colophon}
This thesis was typeset with \LaTeXe\ using Hermann Zapf's
\emph{Palatino}
and \emph{Euler} type faces (Type~1 PostScript fonts \emph{URW
Palladio L}
and \emph{FPL} were used). The listings are typeset in \emph{Bera
Mono}, originally developed by Bitstream, Inc. as ``Bitstream Vera''.
(Type~1 PostScript fonts were made available by Malte Rosenau and
Ulrich Dirr.)

The typographic style was inspired by Bringhurst's genius as
presented in \emph{The Elements of Typographic Style} 
(Bringhurst 2002). It is available for \LaTeX\ via \textsmaller{CTAN} as 
``\href{http://www.ctan.org/tex-archive/macros/latex/contrib/classicthesis/}%
{\texttt{classicthesis}}''.

\paragraph{note:} The custom size of the textblock was calculated
using the directions given by Mr. Bringhurst (pages 26--29 and
175/176). 10~pt Palatino needs  133.21~pt for the string
``abcdefghijklmnopqrstuvwxyz''. This yields a good line length between
24--26~pc (288--312~pt). Using a ``\emph{double square textblock}''
with a 1:2 ratio this results in a textblock of 312:624~pt (which
includes the headline in this design). A good alternative would be the
``\emph{golden section textblock}'' with a ratio of 1:1.62, here
312:505.44~pt. For comparison, \texttt{DIV9} of the \texttt{typearea}
package results in a line length of 389~pt (32.4~pc), which is by far
too long. However, this information will only be of interest for
hardcore pseudo-typographers like me.%

To make your own calculations, use the following commands and look up
the corresponding lengths in the book:
\begin{verbatim}
    \settowidth{\abcd}{abcdefghijklmnopqrstuvwxyz}
    \the\abcd\ % prints the value of the length
\end{verbatim}
Please see the file \texttt{classicthesis.sty} for some precalculated 
values for Palatino and Minion.

    \settowidth{\abcd}{abcdefghijklmnopqrstuvwxyz}
    \the\abcd\ % prints the value of the length


\bigskip

\noindent\finalVersionString



