%\manualmark
%\markboth{\spacedlowsmallcaps{Preface}}{\spacedlowsmallcaps{Preface}} % work-around to have small caps also
\refstepcounter{dummy}

%************************************************
\addtocontents{toc}{\protect\vspace{\beforebibskip}} % to have the bib a bit from the rest in the toc
\addcontentsline{toc}{chapter}{\tocEntry{Preface}}
%************************************************

\chapter*{Preface}
\chaptermark{Preface}

Artificial intelligence is closely related to philosophy of mind.
Philosophers have worked carefully to describe consistent models of
reality.  Artificial intelligence theoreticians and philosophers of
mind have the shared goal of understanding what it means to think.
While philosophers of mind must think about and simulate their models
by using their human brains, artificial intelligence theoreticians
have the advantage of computer memories and processors that can
simulate models with more different parts than any human brain could
ever hope to remember, let alone simulate.

While artificial intelligence theoreticians can quickly experiment
with the behavior of complex theories, these theories must still be
tied to a fundamental consistent philosophical basis in reality.  When
experimenting with complex theories of mind, there is a dangerous
potential for the model to lose a clear reference to reality, making
it meaningless.  Theoretical artificial intelligence is a modern
computational incarnation of philosophy of mind, pursuing greater
realism, complexity, and internal consistency than was ever possible
before computers.

As a theoretical field of research, artificial intelligence does not
accept a description of a problem to solve nor a clearly defined
metric of evaluation as a standard for success.  The reference for
developing a theory is reality itself.  A new theory is a new
description of reality.  Reality exists whether or not it is
symbolized or described.  A new description gives a new set of symbols
and working parts that can be used to define new dimensions of metric
analysis.

Goals are not relationships between symbols; goals are the initial
deliberative activity.  All knowledge exists because of this initial
willful act.  Like all goals, my goal exists as a real activity,
whether or not it is reflectively symbolized.  In this dissertation I
will describe my goal and what I have discovered by pursuing it.  I
will ground my description in a clear philosophy of mind, and I will
relate my work to the work of my peers in the field of artificial
intelligence.

